%check leadership and seconds, esp WP4,5


\documentclass[a4paper,11pt]{article}

\usepackage[top=2cm, bottom=2cm, left=2cm, right=2cm]{geometry}

\usepackage{mathptmx}
\usepackage{epsf}           %\input{epsf}
\usepackage{amsfonts}
\usepackage{amstext}
\usepackage{amssymb}
\usepackage{url}
%\usepackage[dvips]{graphics}
\usepackage[dvips,pdftex]{graphicx}
\usepackage[dvips,all]{xy}
\usepackage{multicol}
\usepackage{natbib}
\setlength{\bibsep}{0.0pt}

\usepackage{hyperref}

%\newlength{\extraplusheight}
%\newlength{\extrapluswidth}
%\setlength{\extraplusheight}{4.7cm}
%\setlength{\extrapluswidth}{4.7cm}
%\addtolength{\textwidth}{\extrapluswidth}
%\addtolength{\textheight}{\extraplusheight}
%\addtolength{\oddsidemargin}{-.5\extrapluswidth}
%\addtolength{\evensidemargin}{-.5\extrapluswidth}
%\addtolength{\topmargin}{-0.5\extraplusheight}
\setlength{\parindent}{0 pt}
\setlength{\parskip}{.5ex}
%\documentclass[a4paper]{article}

\usepackage{pgfgantt}
  

\begin{document}
%\hspace{-3cm}
\thispagestyle{empty}
{
\centering
{\Large HoTT : Programming and Verification Gantt Chart}
\medskip

\setganttlinklabel{s-s}{}
\setganttlinklabel{f-f}{}
\setganttlinklabel{s-f}{}

\begin{tikzpicture}[y=0.5cm]  
\begin{ganttchart}[x unit=1.75cm,y unit chart=.5cm,y unit title=.5cm,vgrid,
canvas/.append style={fill=none, draw=black!5, line width=.75pt},
hgrid style/.style={draw=black!5, line width=.75pt},
vgrid={*1{draw=black!4,dashed,line width=.5pt},*1{draw=black!5, line width=.75pt}},
  title/.style={draw=none, fill=black!5},
%  title label node/.append style={below=0pt},
  include title in canvas=false,
  bar label font=\mdseries\small\color{black!70},
  bar label node/.append style={left=.5cm},
  bar/.append style={draw=none, fill=magenta!63},
%  bar incomplete/.append style={fill=barblue},
%  bar progress label font=\mdseries\footnotesize\color{black!70},
%  group incomplete/.append style={fill=groupblue},
%  group left shift=0,
%  group right shift=0,
%  group height=.5,
%  group peaks tip position=0,
%  group label node/.append style={left=.6cm},
%  group progress label font=\bfseries\small,
 link/.style={-latex, line width=0.5pt, black},
 link label font=\scriptsize\bfseries,
 link label node/.append style={below left=-2pt and 0pt},
 title height=.75, title top shift=0,
% title label anchor={below=-1.5ex},
%,
% bar top shift=-0.1, 
 bar height=0.5
  ]{1}{8}
  \gantttitle{2015}{2}
  \gantttitle{2016}{2}
  \gantttitle{2017}{2}
  \gantttitle{2018}{2}\\
  \ganttbar{WP1}{1}{3}\\
  \ganttbar{WP2}{2}{5}\\
  \ganttbar{WP3}{4}{6}\\
  \ganttbar{WP4}{6}{8}\\
  \ganttbar{WP5}{1}{4}\ganttbar{WP5}{7}{8}\\
  \ganttbar{WP6}{3}{6}\\
  \ganttbar{WP7}{5}{8}\\
  \ganttbar{WP8}{7}{8}
  % \ganttlink[link type=s-s]{elem0}{elem1}
  % \ganttlink[link type=s-s]{elem0}{elem2}
  % \ganttlink[link type=s-s]{elem0}{elem4}
  % \ganttlink[link type=s-s]{elem1}{elem2}
  % \ganttlink[link type=s-s]{elem1}{elem4}
  % \ganttlink[link type=s-s]{elem1}{elem5}
  % \ganttlink[link type=s-s]{elem2}{elem3}
  % \ganttlink[link type=s-s]{elem5}{elem4}
\end{ganttchart}
\end{tikzpicture}

}

\bigskip

{\bf Relationships between work packages:} The following direct
relationships between work packages feed into the design of the above
Gantt chart. WP1,5 start immediately as both have no prerequisites ---
the latter will begin by formalising the cubical set model. We want to
start WP2 early as this is one of the most important work packages and
we already have ideas to pursue within Altenkirch's Cubical Type
Theory. An early start to WP2 minimises the risk that unforseen
difficulties within this key work package holds
up progress elsewhere. As WP1 progresses, its results
will feed into WP2. Similarly, another key work package is our core
language of WP6, so we start this as soon as possible --- since WP2 feeds
into WP6, we therefore minimise risk by starting WP6 only six months after WP2. WP3 needs to be integrated with our syntactic
and semantic foundations and so starts after WP1,2 have had time to
deliver results; 
%however, these three work packages can progress in
%parallel. 
In the same way, WP7 builds upon WP6 and WP3 and so starts
after those. % work packages have had time to deliver results.
WP4 builds upon WP3 as is reflected in
its start date. Finally, once most of our tools are developed, we
start WP8.  

{\bf Duration:} The longest work package is WP5 as this is where our
formal verification takes place and hence is our broadest in terms of
scope. We want to formally verify as many results as possible about
our foundations, programming language tools, and impact generating
case studies.  As a result this is the longest work package which
(with a break for language development) will be undertaken at both the
beginning and the end of the project. WP2,6,7 while less broad, are
equally fundamental to our overall goals and so they are allocated 2
years each. The other work packages are more focussed and
so are given 1.5 years apart from WP8 which we think will only need 1
year as WP8 is the most tightly scoped work package.

Nevertheless, there is flexibility in these lengths to either extend
or shorten work packages as the pace of progress within them becomes
evident. For example, if our work on WP6,7 delivers the required
platform, we will start work on WP8 earlier and look for extensions
to the research proposed. The timing and duration of these work
packages means we are in a fortunate position in that both the more
theoretical RA at Nottingham (WP1--4) and the more applied RA (WP5--8)
will not have to work on more than one work package at a time.

The above durations have also been devised to reduce overall risk in
that the largest risk in the overall proposal is the amount of
engineering work required by programming language and verification
tools. Although, this risk is mitigated by McBride's presence in the
team, it is also mitigated by separating the ambitious goal of a fully
fledged system (WP7) from the more modest goal of a core system (WP6)
and thereby allocating about 50$\%$ of the time of RA2 to WP6 and
WP7.






{\bf Leadership:} WP1 will be jointly led by Nottingham and Leeds as
they have considerable experience there. Having joint leadership will
not be a problem due to the proximity these sites and because Gambino
and Altenkirch successfully collaborated on the HoTT book during the
Special Year on HoTT in Princeton. Indeed, this will be advantageous
as the Nottingham RA will get to work closely with Gambino. Ghani's
experience in category theory will make him a secondary contributor on
WP1. WP2--4 will be led at Nottingham as Altenkirch has significant
experience in all of these areas including a track record of
collaborating with Microsoft. All investigators have experience in
type theory and data types, and hence are secondary contributors to
WP2,3. Ghani's experience in logical relations will also be deployed
here. Ghani and McBride will similarly contribute
to WP4 because of their work on effects and with Microsoft. WP5--8
will be led at Strathclyde and Altenkirch will be a secondary
contributor to WP5--7 given his experience in programming language
implementation. McBride will take the lead on WP5--7 as he has the
most expertise in these areas, while Ghani will take the lead on WP8
because of his experience working with Kennedy on Units of Measure.

\end{document}

%relationships between WPs
%length
%lead and seconds
%WP5 in Yr4, WP8 = 1.5 yrs

\documentclass[a4paper,11pt]{article}

\usepackage[top=2cm, bottom=2cm, left=2cm, right=2cm]{geometry}

\usepackage{mathptmx}
\usepackage{epsf}           %\input{epsf}
\usepackage{amsfonts}
\usepackage{amstext}
\usepackage{amssymb}
\usepackage{url}
%\usepackage[dvips]{graphics}
\usepackage[dvips,pdftex]{graphicx}
\usepackage[dvips,all]{xy}
\usepackage{multicol}
\usepackage{natbib}
\setlength{\bibsep}{0.0pt}

\usepackage{hyperref}

%\newlength{\extraplusheight}
%\newlength{\extrapluswidth}
%\setlength{\extraplusheight}{4.7cm}
%\setlength{\extrapluswidth}{4.7cm}
%\addtolength{\textwidth}{\extrapluswidth}
%\addtolength{\textheight}{\extraplusheight}
%\addtolength{\oddsidemargin}{-.5\extrapluswidth}
%\addtolength{\evensidemargin}{-.5\extrapluswidth}
%\addtolength{\topmargin}{-0.5\extraplusheight}
\setlength{\parindent}{0 pt}
\setlength{\parskip}{.5ex}
%\documentclass[a4paper]{article}

\usepackage{pgfgantt}
  

\begin{document}
%\hspace{-3cm}
{
\centering
{\Large HoTT : Programming and Verification Gantt Chart}
\medskip

\setganttlinklabel{s-s}{}
\setganttlinklabel{f-f}{}
\setganttlinklabel{s-f}{}

\begin{tikzpicture}[y=0.5cm]  
\begin{ganttchart}[x unit=3.5cm,y unit chart=.5cm,y unit title=.5cm,vgrid,
canvas/.append style={fill=none, draw=black!5, line width=.75pt},
hgrid style/.style={draw=black!5, line width=.75pt},
vgrid={*1{draw=black!5, line width=.75pt}},
  title/.style={draw=none, fill=black!5},
%  title label node/.append style={below=0pt},
  include title in canvas=false,
  bar label font=\mdseries\small\color{black!70},
  bar label node/.append style={left=.5cm},
  bar/.append style={draw=none, fill=magenta!63},
%  bar incomplete/.append style={fill=barblue},
%  bar progress label font=\mdseries\footnotesize\color{black!70},
%  group incomplete/.append style={fill=groupblue},
%  group left shift=0,
%  group right shift=0,
%  group height=.5,
%  group peaks tip position=0,
%  group label node/.append style={left=.6cm},
%  group progress label font=\bfseries\small,
 link/.style={-latex, line width=0.5pt, black},
 link label font=\scriptsize\bfseries,
 link label node/.append style={below left=-2pt and 0pt},
 title height=.75, title top shift=0,
% title label anchor={below=-1.5ex},
%,
% bar top shift=-0.1, 
 bar height=0.5
  ]{1}{4}
  \gantttitle{2015}{1}
  \gantttitle{2016}{1}
  \gantttitle{2017}{1}
  \gantttitle{2018}{1}\\
  \ganttbar{WP1}{1}{2}\\
  \ganttbar{WP2}{1}{2}\\
  \ganttbar{WP3}{2}{3}\\
  \ganttbar{WP4}{3}{4}\\
  \ganttbar{WP5}{1}{2}\\
  \ganttbar{WP6}{2}{3}\\
  \ganttbar{WP7}{3}{4}\\
  \ganttbar{WP8}{3}{4}
  % \ganttlink[link type=s-s]{elem0}{elem1}
  % \ganttlink[link type=s-s]{elem0}{elem2}
  % \ganttlink[link type=s-s]{elem0}{elem4}
  % \ganttlink[link type=s-s]{elem1}{elem2}
  % \ganttlink[link type=s-s]{elem1}{elem4}
  % \ganttlink[link type=s-s]{elem1}{elem5}
  % \ganttlink[link type=s-s]{elem2}{elem3}
  % \ganttlink[link type=s-s]{elem5}{elem4}
\end{ganttchart}
\end{tikzpicture}

}

\bigskip

{\bf Relationships between workpackages:} The following direct
relationships between workpackages feed into the design of the above
Gantt Chart. WP1 and WP2 can start immediately because, while WP2 is
influenced by WP1, we can begin WP2 by developing Altenkirch's Cubical
type theory. WP5 can start immeidately as it starts with the
formalisation of the cubical set model before proceeding to formalise
the results of WP1 and WP2. WP3 and WP6 start later so that the results of
WP1 and WP2 can feed into it. WP4, WP7 start even later so that the
results of WP3 and WP6 can feed into them. Finally, WP8 starts last so
that the maximum results will 


{\bf Workloads:} RA1 will over the more theoretical research of WP


 feeds directly into WP2 and WP3 and hence preceeds
them. WP2 feeds dirctly into 




The work will start simultaneously in Leeds on Semantic
Foundations (WP1), in Nottingham on Univalent Type Theory (WP2) and in
Strathclyde by work on Formalisation of the required Meta-Theory
(WP5). We expect this work to be heavily intertwined and proceeding in
parallel. After progress is made on WP2, Nottingham will start looking
into Higher Inductive Types (WP3) as we expect these to be useful in
WP2 as well. As the work on WP1 progresses and comes to a successfull
end, the attention of Nicola Gambino in Leeds will move to HITs as
well, given his expertise in the area. In the meantime, Strathclyde,
having laid the matfoundations in WP5 will begin formalising the Type
Theory produced in WP2 in Nottingham. As the work on implementation
progresses, higher level will be considered (WP6) with the findings
back propagating both to WP5 and WP3. In the final stage Nottingham
will seek apply the type theory to formalising effectful programming
(WP4) while Strathclyde gradually move onto case studies. However the
work on WP7 is not finished until it's utility is tested in WP8. Both
teams will depend on Leeds support in foundational matters in this period. 
\end{document}

%relationships between WPs
%length
%lead and seconds
%WP5 in Yr4, WP8 = 1.5 yrs

\documentclass[a4paper,11pt]{article}

\usepackage[top=2cm, bottom=2cm, left=2cm, right=2cm]{geometry}

\usepackage{mathptmx}
\usepackage{epsf}           %\input{epsf}
\usepackage{amsfonts}
\usepackage{amstext}
\usepackage{amssymb}
\usepackage{url}
%\usepackage[dvips]{graphics}
\usepackage[dvips,pdftex]{graphicx}
\usepackage[dvips,all]{xy}
\usepackage{multicol}
\usepackage{natbib}
\setlength{\bibsep}{0.0pt}

\usepackage{hyperref}

%\newlength{\extraplusheight}
%\newlength{\extrapluswidth}
%\setlength{\extraplusheight}{4.7cm}
%\setlength{\extrapluswidth}{4.7cm}
%\addtolength{\textwidth}{\extrapluswidth}
%\addtolength{\textheight}{\extraplusheight}
%\addtolength{\oddsidemargin}{-.5\extrapluswidth}
%\addtolength{\evensidemargin}{-.5\extrapluswidth}
%\addtolength{\topmargin}{-0.5\extraplusheight}
\setlength{\parindent}{0 pt}
\setlength{\parskip}{.5ex}
%\documentclass[a4paper]{article}

\usepackage{pgfgantt}
  

\begin{document}
%\hspace{-3cm}
\thispagestyle{empty}
{
\centering
{\Large HoTT : Programming and Verification Gantt Chart}
\medskip

\setganttlinklabel{s-s}{}
\setganttlinklabel{f-f}{}
\setganttlinklabel{s-f}{}

\begin{tikzpicture}[y=0.5cm]  
\begin{ganttchart}[x unit=1.75cm,y unit chart=.5cm,y unit title=.5cm,vgrid,
canvas/.append style={fill=none, draw=black!5, line width=.75pt},
hgrid style/.style={draw=black!5, line width=.75pt},
vgrid={*1{draw=black!4,dashed,line width=.5pt},*1{draw=black!5, line width=.75pt}},
  title/.style={draw=none, fill=black!5},
%  title label node/.append style={below=0pt},
  include title in canvas=false,
  bar label font=\mdseries\small\color{black!70},
  bar label node/.append style={left=.5cm},
  bar/.append style={draw=none, fill=magenta!63},
%  bar incomplete/.append style={fill=barblue},
%  bar progress label font=\mdseries\footnotesize\color{black!70},
%  group incomplete/.append style={fill=groupblue},
%  group left shift=0,
%  group right shift=0,
%  group height=.5,
%  group peaks tip position=0,
%  group label node/.append style={left=.6cm},
%  group progress label font=\bfseries\small,
 link/.style={-latex, line width=0.5pt, black},
 link label font=\scriptsize\bfseries,
 link label node/.append style={below left=-2pt and 0pt},
 title height=.75, title top shift=0,
% title label anchor={below=-1.5ex},
%,
% bar top shift=-0.1, 
 bar height=0.5
  ]{1}{8}
  \gantttitle{2015}{2}
  \gantttitle{2016}{2}
  \gantttitle{2017}{2}
  \gantttitle{2018}{2}\\
  \ganttbar{WP1}{1}{3}\\
  \ganttbar{WP2}{2}{5}\\
  \ganttbar{WP3}{4}{6}\\
  \ganttbar{WP4}{6}{8}\\
  \ganttbar{WP5}{1}{4}\ganttbar{WP5}{7}{8}\\
  \ganttbar{WP6}{3}{6}\\
  \ganttbar{WP7}{5}{8}\\
  \ganttbar{WP8}{7}{8}
  % \ganttlink[link type=s-s]{elem0}{elem1}
  % \ganttlink[link type=s-s]{elem0}{elem2}
  % \ganttlink[link type=s-s]{elem0}{elem4}
  % \ganttlink[link type=s-s]{elem1}{elem2}
  % \ganttlink[link type=s-s]{elem1}{elem4}
  % \ganttlink[link type=s-s]{elem1}{elem5}
  % \ganttlink[link type=s-s]{elem2}{elem3}
  % \ganttlink[link type=s-s]{elem5}{elem4}
\end{ganttchart}
\end{tikzpicture}

}

\bigskip

{\bf Relationships between work packages:} The following direct
relationships between work packages feed into the design of the above
Gantt chart. WP1,5 start immediately as both have no prerequisites ---
the latter will begin by formalising the cubical set model. We want to
start WP2 early as this is one of the most important work packages and
we already have ideas to pursue within Altenkirch's Cubical Type
Theory. This minimises the risk that this key work package will hold
up the progress of the other packages. As WP1 progresses, its results
will feed into WP2. Similarly, another key work package is our core
language of WP6, so we start this after one year, enabling the results
of WP2 to feed into it. WP3 needs to be integrated with our syntactic
and semantic foundations and so starts after WP1,2 have had time to
deliver results; however, these three work packages can progress in
parallel. In the same way, WP7 builds upon WP6 and WP3 and so starts
after those. % work packages have had time to deliver results.
The largest risk here is the amount of engineering work required by
programming language and verification tools --- this risk is mitigated
by McBride's presence in the team and our separation of the
ambitious goal of a fully fledged system (WP7) from the more modest
goal of a core system (WP6).  WP4 builds upon WP3 as is reflected in
its start date. Finally, once most of our tools are developed, we
start WP8.  Risk throughout the project is reduced by our plans to
collaborate with experts on each work package --- see {\em Pathways to
  Impact}.


{\bf Duration:} The longest work package is WP5 as this is where our
formal verification takes place. We want to formally verify as many
results as possible, particularly about our model theory, about our
type theory, about our programming language, and about our application
to Units of Measure. As a result this is the longest work package which
(with a break for language development) will be undertaken at both the
beginning and the end of the project. The next most important
work packages in terms of overall goals and impact are WP2,6,7 and so
they are allocated 2 years each. The other work packages are slightly
less fundamental and so are given 1.5 years apart from WP8 which we think
will only need 1 year as WP8 is more tightly scoped than the other
work packages and so probably requires only one year.

Nevertheless, there is flexibility in these lengths to either extend
or shorten work packages as the pace of progress within them becomes
evident. For example, if our work on WP6,7 delivers the required
platform, we will start work on WP8 earlier and look for extensions
to the research proposed. The timing and duration of these work
packages means we are in a fortunate position in that both the more
theoretical RA at Nottingham (WP1--4) and the more applied RA (WP5--8)
will not have to work on more than one work package at a time.

{\bf Leadership:} WP1 will be jointly led by Nottingham and Leeds as
both sites have considerable experience there. Having joint leadership
in this work package will not be a problem due to the close proximity
and the experience we have gained of working together over this
proposal. Indeed, this will be advantageous as the Nottingham RA will
get to work closely with Gambino. Ghani's experience in category
theory will make him a secondary contributor on WP1. WP2--4 will be led
at Nottingham as Altenkirch has significant experience in all of these
areas. Nevertheless, McBride's experience in type theory and Ghani's
experience in logical relations will mean they are secondary
contributors to WP2. All investigators have experience in, and hence
will contribute to, WP3. Ghani and McBride will similarly contribute
to WP4 because of their work on effects. WP5--8 will be led at
Strathclyde but Altenkirch will be a secondary contributor to
WP5--7. McBride will take the lead on WP5--7 while Ghani will take
the lead on WP8 because of his experience working with Kennedy on
Units of Measure.

\end{document}

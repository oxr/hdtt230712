\documentclass[twocolumn,a4paper,11pt]{article}

\usepackage{pstcol}
\usepackage{pst-text}
\usepackage{color}

% \usepackage{multibib}
% \newcites{track}{Publications}
% \newcites{main}{References}

\usepackage{mathptmx}
%\usepackage{epsf}      
%\usepackage{url}
%\usepackage[dvips]{graphics}
%\usepackage[dvips,all]{xy}
%\usepackage[round,authoryear]{natbib}
%\usepackage{multicol}

% \newlength{\extraplusheight}
% \newlength{\extrapluswidth}
% \setlength{\extraplusheight}{4.7cm}
% \setlength{\extrapluswidth}{4.7cm}
% \addtolength{\textwidth}{\extrapluswidth}
% \addtolength{\textheight}{\extraplusheight}
% \addtolength{\oddsidemargin}{-.5\extrapluswidth}
% \addtolength{\evensidemargin}{-.5\extrapluswidth}
% \addtolength{\topmargin}{-0.5\extraplusheight}
\setlength{\parindent}{0.1in}
\setlength{\parskip}{0.4ex}
\setlength{\topmargin}{-2cm}
\setlength{\textheight}{26cm}
\setlength{\oddsidemargin}{-0.68cm}
\setlength{\textwidth}{17cm}

\usepackage{color}
\newcommand{\txa}[1]{\textcolor{red}{\textbf{Thorsten:~}#1}}

%\renewcommand{\cite}[1]{{\tt[#1]}}

\newcommand{\citetrack}[1]{\cite{#1}}
\newcommand{\citemain}[1]{\cite{#1}}

%\input{prelude.tex}


\title{Programming and Reasoning\\ in Higher Dimensional Type Theory \\
\LARGE (Case for Support)}

\author{Thorsten Altenkirch, Eugenia Chang, Neil Ghani and Ondrej Rypacek}
\date{}

\begin{document}
\raggedright
\sffamily

%\twocolumn[
\maketitle
\section*{Summary}
Type Theory is the one of the most promising approaches to formally
certified software and mathematics, being the base of tools like the
proof assistant Coq \cite{CoqArt} and the dependently typed functional
programming language Agda \cite{Agda}. In the research proposed here
we are going to investigate and develop further a novel extension of
the type-theoretic approach based on a new connection between geometry
(Homotopy theory) and reasoning proposed by Field medallist Vladimir
Voevodsky. In a nutshell this \emph{higher dimensional type theory}
allows us to treat structures as first class citizens by viewing
equivalence of structures as equality. We anticipate that this
innovation will have substantial impact on the feasibility of large scale
formal development by supporting the replaceability of components
without affecting the rest of the development. Our work will draw on
and create new connections between fundamental mathematical theories
like Homotopy theory and higher dimensional category theory on the one
side and formally supported software engineering on the other.


\section*{Part 1: Track Record}

\subsection*{Thorsten Altenkirch}
Thorsten Altenkirch received his PhD from the University of
Edinburgh in 1993 since October 2006 he has been a Reader in
Computer Science at the University of Nottingham and in October
2008 he founded  the Functional Programming Laboratory together with
Graham Hutton.

Altenkirch is well known for his work on Type Theory and applications
of category theory in computer science, and has published about 50
research papers (all available online via google scholar) which are
frequently cited (h-index $\geq$ 23). During his work at Nottingham
\pounds 1M in research funding, comprising \pounds 650,903 as PI in 4
EPSRC grants, \pounds 241,075 as CoI in 2 EPSRC grants, \pounds
159,038 in 1 fellowship and 1 studentship. Especially relevant for the
current project is Observational Equality For Dependently Typed Programming
(EP/C512022/1), Theory And Applications of Induction Recursion (EP/G03298X/1),
Reusability and Dependent Types (EP/G034109/1) with the last two still
ongoing. Altenkirch and Ghani have been or still are collaborating on
three research grants.

Altenkirch has has recently given an invited lecture at the Workshop on
Higher Dimensional Algebra, Categories and Types, Lubljana, in 2012
and he is going to be an invited reasearch fellow at the
Institutute for Advanced Study in the Spring term 2013 to work with
Voevodsky and others on topics related to this research grant.
 
% n particular his work on
% normalisation
% \cite{alti:tlca93,alti:types94,alti:ctcs95,alti:lics96,txa:jtait},
% extensionality \cite{alti:lics99,alti:ott-conf}, containers
% \cite{alti:fossacs03,alti:cont-tcs,alti:lics09,txa:cie10} and quantum
% programming languages \cite{alti:qml,alti:qarrows,alti:qio}.
% He has been the principal investigator on 4 EPSRC grants: 
% Modelling Irreversible Quantum Computation (GR/S30818/01),
% Observational Equality For Dependently Typed Programming
% (EP/C512022/1), Theory And Applications of Induction Recursion (EP/G03298X/1),
% Reusability and Dependent Types (EP/G034109/1) with the last two still
% ongoing and the coinvestigator on two further research grants. He has
% applied and hosted one Marie Curie fellowship and one Microsoft PhD
% studentship and is active in the European TYPES community. 

% Altenkirch has recently given an invited lecture at the Workshop on
% Higher Dimensional Algebra, Categories and Types, Lubljana, in 2012
% and was invited to the institute for Advance Study in Princetopn by
% Vladimir Voevodsky and also by Robert Harper and Steve
% Awodey and their colleagues at the Carnegie Mellon University in 2011.
% ALtenkirch is going to be an invited reasearch fellow at the
% Institutute for Advanced Study in the Spring term 2013 to work with
% Voevodsky and others on topics related to this research grant.


\subsection*{Eugenia Chang}

\subsection*{Neil Ghani}

\subsection*{Ondrej Rypacek}

\subsection*{Nottingham - host organisation}

The School of Computer Science at the University of Nottingham
is a research-led School in one of the leading Universities in
the UK. The School was ranked 8th in the last Research Assessment
Exercise, and the Functional Programming Lab within the School is
one of four major research groups, with an international reputation
for its work on formally-based approaches to software construction
and verification.  The FP lab currently comprises 4 academic staff
(Thorsten Altenkirch, Venanzio Capretta, Graham Hutton, and Henrik
Nilsson) and 9 PhD students.  To date the
group has received \pounds 1.5M of EPSRC funding over 14 projects,
and has 12 completed PhD students.

The Functional Programming Lab provides a highly stimulating
research environment for researchers and PhD students with weekly
research meetings and frequent seminars. 

%]
% {\small 
% \bibliographystyletrack{abbrv}
% \bibliographytrack{proposal} 
% }

%\newpage

\newpage

\section*{Part 2: Proposed Research}

Recently, Field medallist Vladimir Voevodsky of the Institute of
Advanced Study in Princeton became one of the latest proponents of
formally developed, computer checked Mathematics. Voevodsky is now
using the interactive proof system Coq to support his work and
encourages his colleagues to follow suit. However, while impressed
with the potential of systems like Coq based on Type Theory Voevodsky
also proposed an important extension of the type theoretic approach
based on his background in Homotopy theory: Univalent Type Theory ---
an instance of a higher dimensional theory where the structure of
equality proofs in non-trivial.

Higher dimensional Type Theory with univalence enables us to view
mathematical structures as first class citizens and identify
equivalent structures as if they were equal. While this is clearly
important for the development of Mathematics it also is essential for
the development of a large corpus of reusable and certified software
allowing us to replace one abstract module by another equivalent one
without having to repeat the effort of certification. While this is a
long standing issue in the use of abstract datatypes the novelty of
Higher dimensional Type Theory lies in the possibility to view equivalent
structures as equal which is not supported by any existing approach.

\section{Background}

Type Theory ala Martin-L\"of is at the same time a programming
language and a logical system based on the propositions as as types
principle. The basic notions are $\Pi$-types generalizing the notion
of a function type from functional programming to a situation where
the domain type can \emph{depend} on the actual input and on the
logical side covering the intuitionistic  explanation of the notions of 
implication and universal quantification. On the other hand
$\Sigma$-types generalize the notion of a product type (or record) in
functional programming and on the logical side allow us to model
conjunction and existential quantification. A 3rd central component are
equality types which assign to any two values the type of proofs that 
these values are equal. Unlike in conventional logic, Type Theory
allows us to talk about properties of proofs, e.g. we can ask the
question whether two propositions (i.e. types) aren't only logically
equivalent but whether they are actually isomorphic
(i.e. computationally equivalent). Other components of Type Theory are
inductive and coinductive types which allow us to construct trees
with finite or potentially infinite depth and notion of a universe,
such as the universe of small sets corresponding to inaccessible
cardinals in set theory.

The notion of equality in Type Theory raises some fundamental questions. 
% Many interesting questions in relation to Type Theory center around
% the notion of equality. 
Can we prove the principle of functional extensionality, i.e. that two
functions are equal if they are pointwise equal? Maybe surprisingly,
this principle is not provable in Intensional Type Theory which is the
basis of most implementations of Type Theory (e.g. Agda,
Coq). However, this shortcoming can be partly addressed using
Observational Type Theory under the assumption of uniqueness of
identity proofs, which is one of the main outcomes of EPSRC project
XXX based on earlier work by Altenkirch \cite{altenkirch:extSetoids}.
The question whether two proofs of equality are themselves equal
(uniqueness of equality proofs) was open until it was shown by Hofmann
and Streicher that this is not provable in standard Type Theory using
a groupoid interpretation of Type Theory
\cite{hofmannStreicher:groupoids}. Later Lumsdaine and independently
Garner and van den Berg
\cite{lumsdaine:omegaCatsFromTT,bergGarner:typesAreWeakOG} showed that
equality in Type Theory gives rise to a weak $\omega$-groupoid, a
higher-categorical structure well known in homotopy theory. These
results suggest a novel interpretation of Type Theory based on the
interplay of Type Theory on the logical side, homotopy theory on the
geometrical side and higher-dimensional category theory on the
algebraic side.

% on homotopy theory
Voevodsky and Awodey developed an interpretation of Type Theory using
homotopy theory
\cite{voevodsky:equivalenceAndUnivalence,awodey:tth}. In this
interpretation types are viewed as (special) topological spaces,
elements as points and equality proofs as paths or homotopies between
elements. The homotopy interpretation gives a very intuitive geometric
explanation for the unprovability of uniqueness of equality proofs. It
also lead Voevodsky to postulate the univalence axiom, which states
that weakly equivalent types should be equal. In particular isomorphic
sets such as natural numbers and lists of natural numbers are equated
as a consequence of the univalence axiom. Clearly, the univalence
axiom is incompatible with uniqueness of equality proofs since in
general there is more than one way to show that two isomorphic sets
are equal. The univalence axiom implies the principle of
extensionality --- indeed it can be viewed as a strong extensionality
principle which identifies indistinguishable types.

% higher category theory
The development of algebraic topology and homotopy theory in
particular has driven the development of higher-dimensional category
theory as its algebraic counterpart. A higher(-dimensional) category
is a category where the arrows between two objects form not just a set
but again a higher category. One can assign a higher category to every
space by considering the points in the space as objects (0-cells), the
paths between points as arrows (1-cells), the deformations of paths as
arrows between arrows (2-cells), etc. This is analogous to the
formation of higher identity types in Type Theory. It is a property
not only of higher categories arising in this way that the axioms of a
category don't hold on the nose but only up to higher cells. This
weakening of equality to cells gives rise to an exponential blowup in
the complexity of naive formulations of higher-dimensional categories
\cite{gordonPowerStreet:tricategories,trimble:tetracategories} which
has ultimately led to the development of more elaborate approaches to
higher-dimensional categories, most notably based on higher operads
\cite{batanin:monoidal,cheng:comparingOperadicTheories}, weak
enrichment \cite{leinster:survey}, simplicial and opetopic sets
\cite{joyal02:quasicategories,baezDolan:opetopes,cheng:opetopicAndMultitopicFoundations}
among many others \cite{leinster:HOHC,leinster:survey,chengLauda:guidebook}.  These various formulations need yet to be
refined, compared and further developed in order to arrive at a
tractable and workable theory of higher categories, in particular when
the goal is a formally verified and computable formulation.

Type Theory has an increasing influence on the development of
certified software and mathematical theories in particular through the
Coq system. While Coq in practice maintains a separation of logic and
proof, newer developments like the Agda system introduce Type Theory
as a total functional programming language with a particular
expressive type system and thus makes Type Theory accessible to
interested programmers. The availability of extensionality principles
(such as functional extensionality and univalence)
is here of practical importance because it is essential for an
structured development of a complex deliverables allowing us to 
replace one module by another, behaviourially equivalent one.
Unlike in conventional logic where it is enough just to postulate an
axiom this is not sufficient in Type Theory because this may stop
computation. Hence extensionality principles come with a canonicity
problem which needs to be addressed before these principles can be used
in practice. 


% - Type Theory
% - Equality in Type Theory, UIP, Groupoid model
% - Models of Type Theory (CWFs, LCCCs)
% - Higher categories & groupoids
% - Problem of extensionality in Type Theory
% - Notion of univalence...
% - Implementations of Type Theory (Agda, Coq)
% - the problem of canonicity 


% \subsection{Higher dimensional category theory}
% \label{sec:high-dimens-categ}

% \subsection{Homotopy Type Theory}
% \label{sec:homotopy-type-theory}


\section{Programme and Methodology}

In the research proposed here we are going to explore
Higher-dimensional Type Theory in order to investigate its potential
impact on Computer Science in particular the efficient development of
certified software. Identifying equivalence of structures with
equality raises well known coherence issues which have been
investigated in the context of higher dimensional category theory
\cite{eugenia} which forms one of the foundations of our research
(WP1). This background is essential when we study the syntax and
semantics of higher dimensional type theory (WP2) --- hoping to
address thorny issues such as the canonicity problem in presence of
the univalence principle. The Agda system \cite{Agda}
which is at the same time a programming language and a interactive
proof system is ideally suited to make these concepts available to
interested researchers. We will use the Agda system both as a tool to
develop the theory (WP3) but also as a target to turn theory into
practice by developing software tools based on Agda to support the use
of Higher-dimensional Type Theory for certification (WP4). Finally we
plan to conduct a number of case studies evaluating the potential
impact of Higher-dimensional Type Theory in Computer Science (WP5) and
Computer Aided Mathematics (WP6).

\subsection*{WP1 : Higher dimensional category theory}
\label{sec:wp:qio}

In this work package we aim at laying the mathematical foundations for
the rest of the project. A central component is a better understanding
of $\omega$ groupoids which can be viewed as the higher dimensional
generalisation of equivalence relations. There are a number of
different approaches \cite{leinster:survey,batanin:monoidal,joyal02:quasicategories,baezDolan:opetopes} including our own
\cite{altenRypacek:weakOmegaGrp} and Coquand's recent work on a constructive reformulation
of simplicial sets\cite{??}. We need to understand better how these approaches
are related and what are their respective advantages and
disadvantages. While our main emphasis is on groupoids we also need to
relate them to the more general case of higher dimensional categories
and in particular investigate $(n,\infty)$ categories.

\txa{Say something about Opetopes?}

\txa{Need lots of input on this WP. Eugenia? Neil?}

\subsubsection*{Research challenges}
\label{sec:rsearch-challenges}
\begin{itemize}
\item Identify a workable, constructive definition of a weak
$\omega$-groupoid.
\item Relate this to existing approaches to $\omega$-categories and
  groupoids based on contractible operads.
\item Relate the notion the simplicial approach to $\omega$-groupoid.
\end{itemize}


\subsection*{WP2 : Foundations of Higher Dimensional Type Theory}

Building on the work in WP1 we are going to investigate higher
dimensional type theory from a semantic perspective. A starting point
are 2-dimensional locally cartesian closed categories and
higher dimensional generalisations of categories with families. We
need to address the issue that weak 2-groupoids do not form a model of
first order type theory. 

On the other hand we need a syntactic representation of higher
dimensional type theory, i.e. a collection of
judgements. \cite{licataHarper:canonicity2d} present such a system but assume that
the lower level is extensional and ignore the role of explicit
weakenings. Inspired by our semantical analysis we plan to develop
alternatives based on the groupoid interpretation of type theory.
Having done this for lower dimensions will inspire the design of
the $\omega$-dimensional case.

A central question is wether the extension of type theory by
univalence preserves canonicity, i.e. wether any closed term of a
first-order datatype is reducible (or at least provably equal) to a
term in constructor form. Harper \cite{licataHarper:canonicity2d} have
a first result for the 2-dimensional case but we believe we can
improve on this based on our previous work on the elimination of
extensionality \cite{altenkirch:extSetoids,altenMcBSwier07:beast} by
presenting a more general and simpler construction based on the
groupoid interpretation. We also envisage that this will lay the
foundation for the higher-dimensional case.

\subsubsection*{Research challenges}
\label{sec:rsearch-challenges}

\begin{itemize}
\item Investigate models of Type Theory (CWFs, locally cartesian
  closed categories) in a higher dimensional setting,

\item Develop a feasible judgemental presentation of higher order,
  higher dimensional Type Theory

\item Identify a notion of higher dimensional model and show that
  standard constructions give rise or are closed under this notion,

\item Investigate the problem of weak and strong canonicity
  in the presence of extensionality and univalence.
\end{itemize}

\subsection*{WP3 : Formalisation} 
\label{sec:wp:qio}

Since we are working in the context of formally verified, machine
checked Mathematics it is clear that we are going to apply these ideas
to our own development. This is also essential due to the complexity
and dependency on rather subtle details which makes a purely paper
based development doubtful. Our goal is to formalize aspects of higher
dimensional category theory and type theory in Agda. One major
limitation of this may be the computational feasibility of this
endeavour putting new demands on the implementation. We would also
like to understand the metatheoretical requirements for the
formalisation of higher dimensional structures - here we hope to be
able to build on results from out recent EPSRC project XXX in
induction-recursion. 

\subsubsection*{Research challenges}

\begin{itemize}
\item Make the notions of WP1 and WP2 sufficiently precise so that they
  can be formalized and mechanically checked in Agda.

\item Formalize core aspects of higher dimensional category theory in
  Agda.

\item Formalize core aspects of higher dimensional type theory in Agda.

\item Identify the requirements on the Metatheory we need to represent
  our notion (e.g. induction-recursion or induction-induction related
  to project XXX).

\end{itemize}

\subsection*{WP4 : Implementation} 

To make higher dimensional Type Theory usable we need an
implementation. Realistically we cannot expect a full scale
implementation or modification of an existing system. Moreover,
Voevodsky has already proposed to start such a development at
Princeton but focusing on application in Mathematics. Within the
present project we are focusing on the implementation of a higher
dimensional type theory as a proof of concept. 

\subsubsection*{Research challenges}

\begin{itemize}
\item Develop the implementation of a higher dimensional core language
  in a functional language (e.g. Haskell).

\item Study the potential of modifying the Agda system to support
  higher dimensional concepts.
  
\end{itemize}


\subsection*{WP5 : Applications in Computer Science} 
Recently higher dimensional inductive types have been proposed
\cite{shulman:blog} which provide a type-theoretic account of the
notion of the fundamental groupoid of a space. This notion has also
interesting applications in computer science for datastructures whose
equality can be inductively defined. Going beyond this we consider the
notion of a higher dimensional quotients which arise as the left
adjoint of the embedding of types into the category of $n$-groupoids
(or more general $\omega$-groupoids). Even 2-dimensional quotients
provide interesting applications to the theory of quotient containers
\cite{abottAltenGhaniMcB:quotientContainers} developed in EPSRC
project XXX which has recently aquired new attention
\cite{gylterud:thesis,kock:groupoids}: we can reduce quotient containers to ordinary
containers in a higher dimensional setting.

Rypacek \cite{rypacek:thesis} found exciting applications of higher
dimensional reasoning in programming language theory (distributive
laws) in particular he was able to analyze the relation between
functional and object oriented programming. This approach is ideally
suited to be developed formally within a higher dimensional type theory


\subsubsection*{Research challenges}

\begin{itemize}
\item Give a precise syntactic and semantic description of higher
  dimensional inductive (and coinductive) types.
\item Develop the notion of higher dimensional quotient types and
  investigate their relation to higher dimensional datatypes.
\item Investigate the application of higher dimensional quotient to
  the theory of quotient containers.
\item Develop Rypacek's distributive laws in the setting of higher
  dimensional type theory.
\end{itemize}

\subsection*{WP7 : Applications in Formal Mathematics} 

\txa{Need help with this. Clearly we are looking for an area where
  reasoning upto equivalence is important without being too reflective
  (i.e. higher dimensional cats or semantics of type theory)}

\subsubsection*{Research challenges}

\section{Relevance to Beneficiaries}

Our work will create new connections between pure Mathematics and
formally justified software engineering, benefiting researchers in
both areas: Mathematicians who are becoming aware of new applications
of the theories which will also raise new questions and raise further
research interest and on the other hand researchers in formal software
engineering who are using tools like Coq and Agda. In the long run a
development of a higher dimensional theory is essential for the
feasibility of larger scale modular development of certified software
and will in the long run have an important impact on the practice of
software engineering.

\txa{Say more ?}

\section{Dissemination and Exploitation}

Our dissemination strategy exploits both modern interactive forms of
communication with the community eg using twitter, facebook, blogs and
mailing lists. We will also provide a webpage making all research
output immediately available to fellow researchers and the interested
public. At the same time we will also publish research papers at
conferences (such as LICS, POPL, CSL, ITP) and in journals such as JFP,
TCS, etc. We will continue to participate in the regular Agda
Implementers meeting and make our developments available to the Type
Theory community.  

\txa{Say more ?}

%{ \small 
% \bibliographystylemain{abbrv}
% \bibliographymain{proposal} 
% }

{\small
\bibliographystyle{plain}
\bibliography{proposal} 
}

\end{document}

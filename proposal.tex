\documentclass[a4paper]{article}

\usepackage{pstcol}
\usepackage{pst-text}
\usepackage{color}

% \usepackage{multibib}
% \newcites{track}{Publications}
% \newcites{main}{References}

\renewcommand{\cite}[1]{{\tt[#1]}}

\newcommand{\citetrack}[1]{\cite{#1}}
\newcommand{\citemain}[1]{\cite{#1}}

%\input{prelude.tex}


\title{Programming and Reasoning\\ in Higher Dimensional Type Theory \\
\LARGE (Case for Support)}

\author{Thorsten Altenkirch, Eugenia Chang, Neil Ghani and Ondrej Rypacek}
\date{}

\begin{document}

%\twocolumn[
\maketitle
\section*{Summary}
Type Theory is the one of the most promising approaches to formally
certified software and mathematics, being the base of tools like the
proof assistant Coq \cite{coq} and the dependently typed functional
programming language Agda \cite{Agda}. In the research proposed here
we are going to investigate and develop further a novel extension of
the type-theoretic approach based on a new connection between geometry
(Homotopy theory) and reasoning proposed by Field medallist Vladimir
Voevodsky. In a nutshell this \emph{higher dimensional type theory}
allows us to treat structures as first class citizens by viewing
equivalence of structures as equality. We anticipate that this
innovation will have substantial impact on the feasibility of large scale
formal development by supporting the replaceability of components
without affecting the rest of the development. Our work will draw on
and create new connections between fundamental mathematical theories
like Homotopy theory and higher dimensional category theory on the one
side and formally supported software engineering on the other.


\section*{Part 1: Track Record}

\subsection*{Thorsten Altenkirch}
Thorsten Altenkirch received his PhD from the University of
Edinburgh in 1993. He has been a research assistant at the University of
Edinburgh and at Chalmers University in Gothenburg, Sweden and has
held lecturing positions at the University of Munich, Germany, and the
University of Nottingham. Since October 2006 he has been a Reader in
Computer Science at the University of Nottingham and in October
2008 he founded the Functional Programming Laboratory together with
Graham Hutton. Altenkirch and Hutton are now jointly leading the
laboratory.


\subsection*{Eugenia Chang}

\subsection*{Neil Ghani}

\subsection*{Ondrej Rypacek}

% in the School of Computer Science, at The University of
% Nottingham.% \citetrack{g53nsc:g54nsc}.

\subsection*{Nottingham - host organisation}

The School of Computer Science at the University of Nottingham
is a research-led School in one of the leading Universities in
the UK.	 The School was ranked 8th in the last Research Assessment
Exercise, and the Functional Programming Lab within the School is
one of four major research groups, with an international reputation
for its work on formally-based approaches to software construction
and verification.  The FP lab currently comprises 4 academic staff
(Thorsten Altenkirch, Venanzio Capretta, Graham Hutton, and Henrik
Nilsson), 2 postdoctoral fellows, and 9 PhD students.  To date the
group has received \pounds 1.5M of EPSRC funding over 14 projects,
and has 12 completed PhD students.

The Functional Programming Lab provides a highly stimulating
research environment for researchers and PhD students with weekly
research meetings and frequent seminars. 

%]
% {\small 
% \bibliographystyletrack{abbrv}
% \bibliographytrack{proposal} 
% }

%\newpage

\newpage

\section*{Part 2: Proposed Research}

Recently, Field medallist Vladimir Voevodsky of the Institute of
Advanced Study in Princeton became one of the latest proponents of
formally developed, computer checked Mathematics. Voevodsky is now
using the interactive proof system Coq to support his work and
encourages his colleagues to follow suit. However, while impressed
with the potential of systems like Coq based on Type Theory Voevodsky
also proposed an important extension of the type theoretic approach
based on his background in Homotopy theory: Univalent Type Theory.
\cite{voevodsky,awodey}. 
\begin{quote}{\footnotesize
Should we say something like: We will build on Univalent Type Theory and take it
further to a .... theory, which we call Higher-dimensional Type
Theory.}
\end{quote}
Univalent Type Theory enables us to view mathematical structures as
first class citizens and identify equivalent structures as if they
were equal. While this is clearly important for the development of
Mathematics it also is essential for the development of a large corpus
of reusable and certified software allowing us to replace one abstract
module by another equivalent one without having to repeat the effort
of certification. While this is a long standing issue in the use of
abstract datatypes the novelty of Univalent Type Theory lies in the
possibility to view equivalent structures as equal which is not
supported by any existing approach.


\section{Background}

Type Theory ala Martin-L\"of is at the same time a programming
language and a logical system based on the propositions as as types
principle. The basic notions are $\Pi$-types generalizing the notion
of a function type from functional programming to a situation where
the domain type can \emph{depend} on the actual input and on the
logical side covering the intuitionistic  explanation of the notions of 
implication and universal quantification. On the other hand
$\Sigma$-types generalize the notion of a product type (or record) in
functional programming and on the logical side allow us to model
conjunction and existential quantification. A 3rd central component are
equality types which assign to any two values the type of proofs that 
these values are equal. Unlike in conventional logic, Type Theory
allows us to talk about properties of proofs, e.g. we can ask the
question whether two propositions (i.e. types) aren't only logically
equivalent but whether they are actually isomorphic
(i.e. computationally equivalent). Other components of Type Theory are
inductive and coinductive types which allow us to construct trees
with finite or potentially infinite depth and notion of a universe,
such as the universe of small sets corresponding to inaccessible
cardinals in set theory.

The notion of equality in Type Theory raises some fundamental questions. 
% Many interesting questions in relation to Type Theory center around
% the notion of equality. 
Can we prove the principle of functional extensionality,
i.e. that two functions are equal if they are pointwise equal? Maybe
surprisingly, this principle is not provable in Intensional Type
Theory which is the basis of most implementations of Type Theory
(e.g. Agda, Coq). However, this shortcoming can be partly addressed
using Observational Type Theory under the assumption of uniqueness of
identity proofs, which is one of the main outcomes of EPSRC project
XXX based on earlier work by Altenkirch \cite{alti:lics99}.
The question whether two proofs of
equality are themselves equal (uniqueness of equality proofs) 
was open until it was shown by Hofmann and Streicher that this is not
provable in standard Type Theory using a groupoid interpretation of
Type Theory \cite{groupoid-model}. Later Lumsdaine and independently
Garner and van den Berg \cite{lumsdaine,garner-deberg} showed that
equality in Type Theory gives rise to a weak $\omega$-groupoid, a
structure well known in higher-dimensional category theory. These results
suggest a novel interpretation of Type Theory. 

Voevodsky and Awodey developed an interpretation of Type Theory using
homotopy theory \cite{voevodsky,awodey}. In this interpretation types
are viewed as (special) topological spaces, elements as points and
equality proofs as paths or homotopies between elements. The homotopy
interpretation gives a very intuitive geometric explanation for the
unprovability of uniqueness of equality proofs. It also lead Voevodsky
to postulate the univalence axiom, which states that weakly equivalent
types should be equal. In particular isomorphic sets such as natural
numbers and lists of natural numbers are equated as a consequence of
the univalence axiom. Clearly, the univalence axiom is incompatible
with uniqueness of equality proofs since in general there is more than
one way to show that two isomorphic sets are equal. The univalence
axiom implies the principle of extensionality --- indeed it can be viewed
as a strong extensionality principle which identifies
indistinguishable types.  

Type Theory has an increasing influence on the development of
certified software and mathematical theories in particular through the
Coq system. While Coq in practice maintains a separation of logic and
proof, newer developments like the Agda system introduce Type Theory
as a total functional programming language with a particular
expressive type system and thus makes Type Theory accessible to
interested programmers. The availability of extensionality principles
(such as functional extensionality and univalence)
is here of practical importance because it is essential for an
structured development of a complex deliverables allowing us to 
replace one module by another, behaviourially equivalent one.
Unlike in conventional logic where it is enough just to postulate an
axiom this is not sufficient in Type Theory because this may stop
computation. Hence extensionality principles come with a canonicity
problem which needs to be addressed before these principles can be used
in practice. 



% - Type Theory
% - Equality in Type Theory, UIP, Groupoid model
% - Models of Type Theory (CWFs, LCCCs)
% - Higher categories & groupoids
% - Problem of extensionality in Type Theory
% - Notion of univalence...
% - Implementations of Type Theory (Agda, Coq)
% - the problem of canonicity 


% \subsection{Higher dimensional category theory}
% \label{sec:high-dimens-categ}

% \subsection{Homotopy Type Theory}
% \label{sec:homotopy-type-theory}


\section{Programme and Methodology}

In the research proposed here we are going to explore
Higher-dimensional Type Theory in order to investigate its potential
impact on Computer Science in particular the efficient development of
certified software. Identifying equivalence of structures with
equality raises well known coherence issues which have been
investigated in the context of higher dimensional category theory
\cite{eugenia} which forms one of the foundations of our research
(WP1). We conjecture that this background is useful when developing
Higher-dimensional Type Theory addressing thorny issues such as the canonicity
problem \cite{harper,voevodsky,coquand} and the formulation of higher
dimensional quotients (WP2). The Agda system \cite{agda} which is at
the same time a programming language and a interactive proof system is
ideally suited to make these concepts available to interested
researchers. We will use the Agda system both as a tool to develop the
theory (WP3) but also as a target to turn theory into practice by
developing software tools based on Agda to support the use of
Higher-dimensional Type Theory for certification (WP4). Finally we plan to
conduct a number of case studies evaluating the potential impact of
Higher-dimensional Type Theory in Computer Science (WP5) and Computer Aided
Mathematics (WP6).

\subsection*{WP1 : Higher dimensional category theory}
\label{sec:wp:qio}

In this work package we aim at laying the mathematical foundations for
the rest of the project. A central component is a better understanding
of $\omega$ groupoids which can be viewed as the higher dimensional
generalisation of equivalence relations. There are a number of
different approaches \cite{operads, etc} including own own
\cite{csl12} and Coquand's recent work on a constructive reformulation
of simplicial sets. We need to understand better how these approaches
are related and what are their respective advantages and
disadvantages. While our main emphasis is on groupoids we also need to
relate them to the more general case of higher dimensional categories
and in particular investigate $(n,\infty)$ categories.

Opetopes?

\subsubsection*{Research challenges}
\label{sec:rsearch-challenges}
\begin{itemize}
\item Identify a workable, constructive definition of a weak
$\omega$-groupoid.
\item Relate this to existing approaches to $\omega$-categories and
  groupoids based on contractible operads.
\item Relate the notion the simplicial approach to $\omega$-groupoid.
\end{itemize}


\subsection*{WP2 : Foundations of Higher Dimensional Type Theory}

Building on the work in WP1 we are going to investigate higher
dimensional type theory from a semantic perspective. A starting point
are 2-dimensional locally cartesian closed categories (2LCCCs) and
higher dimensional generalisations of categories with families. We
need to address the issue that weak 2-groupoids do not form a model of
first order type theory. 

On the other hand we need a syntactic representation of higher
dimensional type theory, i.e. a collection of
judgements. \cite{harper:popl} present such a system but assume that
the lower level is extensional and ignore the role of explicit
weakenings. Inspired by our semantical analysis we plan to develop
alternatives based on the groupoid interpretation of type theory.
Having done this for lower dimensions will inspire the design of
the $\omega$-dimensional case.

A central question is wether the extension of type theory by
univalence preserves canonicity, i.e. wether any closed term of a
first-order datatype is reducible (or at least provably equal) to a
term in constructor form. Harper \cite{harper:popl} have a first
result for the 2-dimensional case but we believe we can improve on
this based on our previous work on the elimination of extensionality
\cite{alti99,plpv08} by presenting a more general and simpler
construction based on the groupoid interpretation. We also envisage
that this will lay the foundation for the higher-dimensional case.

\subsubsection*{Research challenges}
\label{sec:rsearch-challenges}

\begin{itemize}
\item Investigate models of Type Theory (CWFs, locally cartesian
  closed categories) in a higher dimensional setting,

\item Develop a feasible judgemental presentation of higher order,
  higher dimensional Type Theory

\item Identify a notion of higher dimensional model and show that
  standard constructions give rise or are closed under this notion,

\item Investigate the problem of weak and strong canonicity
  in the presence of extensionality and univalence.
\end{itemize}

\subsection*{WP3 : HIgher dimensional datastructures}

Recently higher dimensional inductive types have been proposed
\cite{shulman} which provide a type-theoretic account of the notion of
the fundamental groupoid of a space. This notion has also interesting
applications in computer science for datastructures whose equality can
be inductively defined. Going beyond this we consider the notion of a
higher dimensional quotients which arise as the left adjoint of the
embedding of types into the category of $n$-groupoids (or more general
$\omega$-groupoids). Even 2-dimensional quotients provide interesting
applications to the theory of quotient containers \cite{qcont}
developed in EPSRC project XXX which has recently aquired new
attention \cite{gylterud,kock}: we can reduce quotient containers to
ordinary containers in a higher dimensional setting. 

\subsubsection*{Research challenges}

\begin{itemize}
\item Give a precise syntactic and semantic description of higher
  dimensional inductive (and coinductive) types.
\item Develop the notion of higher dimensional quotient types and
  investigate their relation to higher dimensional datatypes.
\item Investigate the application of higher dimensional quotient to
  the theory of quotient containers.
\end{itemize}


\subsection*{WP4 : Formalizing higher dimensional structures} 
\label{sec:wp:qio}

Since we are working in the context of formally verified, machine
checked Mathematics in is clear that we are going to apply these ideas
to our own development. This is also essential due to the complexity
and dependency on rather subtle details which makes a purely paper
based development doubtful. Our goal is to formalize aspects of higher
dimensional category theory and type theory in Agda. One major
limitation of this may be the computational feasibility of this
endeavour putting new demands on the implementation. We would also
like to understand the metatheoretical requirements for the
formalisation of higher dimensional structures - here we hope to be
able to build on results from out recent EPSRC project XXX in
induction-recursion. 

\subsubsection*{Research challenges}

\begin{itemize}
\item Make the notions of WP1 and WP2 sufficiently precise so that they
  can be formalized and mechanically checked in Agda.

\item Formalize core aspects of higher dimensional category theory in
  Agda.

\item Formalize core aspects of higher dimensional type theory in Agda.

\item Identify the requirements on the Metatheory we need to represent
  our notion (e.g. induction-recursion or induction-induction related
  to project XXX).

\end{itemize}

\subsection*{WP5 : Implementation of higher-dimensional calculi} 



\subsubsection*{Research challenges}

\begin{itemize}
\item Develop the implementation of a higher dimensional core language
  in a functional language (e.g. Haskell).

\item Study the potential of modifying the Agda system to support
  higher dimensional concepts.
  
\end{itemize}


\subsection*{WP6 : Applications in Computer Science} 
\label{sec:wp:qio}

\subsubsection*{Research challenges}
\label{sec:rsearch-challenges}

\subsection*{WP7 : Applications in Formal Mathematics} 
\label{sec:wp:qio}

\subsubsection*{Research challenges}
\label{sec:rsearch-challenges}

\section{Relevance to Beneficiaries}



\section{Dissemination and Exploitation}



{% \small 
% \bibliographystylemain{abbrv}
% \bibliographymain{proposal} 
% }

{\small
\bibliographystyle{abbrv}
\bibliography{proposal} 
}
\end{document}

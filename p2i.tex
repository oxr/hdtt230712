

  \documentclass[a4paper,11pt]{article}



\usepackage[top=2cm, bottom=2cm, left=2cm, right=2cm]{geometry}


\setlength{\parindent}{0 pt}
\setlength{\parskip}{.5ex}

\usepackage{mathptmx}

\usepackage{epsf}           %\input{epsf}

\usepackage{amsfonts}

\usepackage{amstext}
  
\usepackage{amssymb}

\usepackage{url}

\usepackage[dvips]{graphics}

\usepackage[dvips,all]{xy}

\usepackage{multicol}



%\newlength{\extraplusheight}

%\newlength{\extrapluswidth}

%\setlength{\extraplusheight}{4.7cm}

%\setlength{\extrapluswidth}{4.7cm}

%\addtolength{\textwidth}{\extrapluswidth}

%\addtolength{\textheight}{\extraplusheight}

%\addtolength{\oddsidemargin}{-.5\extrapluswidth}

%\addtolength{\evensidemargin}{-.5\extrapluswidth}

%\addtolength{\topmargin}{-0.5\extraplusheight}

%\setlength{\parindent}{0 pt}

%\setlength{\parskip}{1ex}



\newcommand{\Int}[1]{[\![ #1 ]\!]}

\newcommand{\malign}[1]{\begin{array}[t]{@{}l@{\;}l@{}l@{}} #1 \end{array}}

\newcommand{\logrel}[2]{\Delta_{#1,#2}}

\newsavebox{\fminibox}

\newenvironment{fminipage}

% {\begin{lrbox}{\fminibox}\begin{minipage}{8cm}\vspace*{-2ex}}
% {\\[-2ex]\vspace*{-2ex}\end{minipage}\end{lrbox}\noindent\centerline{\fbox{\usebox{\fminibox}}}\vspace{0.5ex}}   



%\setlength{\parindent}{0.15in}

%\setlength{\parskip}{0.3ex}



% Discourage unnecessary hyphenation.

\sloppy\hyphenpenalty 4000



\newcommand{\ra}{\rightarrow}

\newcommand{\A}{\mathcal{A}}

\newcommand{\E}{\mathcal{E}}

\newcommand{\C}{\mathcal{C}}

\newcommand{\B}{\mathcal{B}}

\newcommand{\Set}{\mbox{{\sf Set}}}

\newcommand{\Nat}{\mathit{Nat}}

\newcommand{\Alge}[1]{\mathit{Alg}_{#1}}

\newcommand{\hash}{\#}



\begin{document}



\thispagestyle{plain}

\begin{center}

  {\Large {\bf Homotopy Type Theory: Programming and Verification \\
\vspace{0.2in} 
Pathways to Impact}}\\[1ex]   



\vspace*{-0.1in}



  \rule{150mm}{.5mm}\\[2ex]

\end{center}



\noindent



\vspace*{-0.1in}

{\bf Cost of Software Failure.} The cost of software failure is truly
staggering. Well known individual cases include the Mars Climate
Orbiter failure ($\pounds 80$ million), Ariane Rocket disaster
($\pounds 350$ million), Pentium Chip Division failure ($\pounds 300$
million), and more recently the heartbleed bug (est.\ $\pounds 400$
million).  There are many, many more examples. Even worse, 
failures such as one in the Patriot Missile System and
another in the Therac-25 radiation system have cost lives. More
generally, a 2008 study by the US government estimated that faulty
software costs the US economy $\pounds 100$ billion
annually~\cite{cnet08}.  A
major approach to software correctness is {\em language-based
  verification}, where successful compilation provides machine
certification of correctness. This is the logical conclusion of the
persistent trend in software engineering toward ever earlier program
verification. Thus, while the proposal is foundational in nature, the
economic importance of the problem it addresses means that our
Pathways to Impact document is stronger than many other similarly
foundational proposals, e.g. including summer schools, collaboration
with world leaders, dedicated impact generating work packages --- each
with industrial collaborators --- and plans for Impact Accelerator
Accounts. Our pathways include:

\vspace*{0.02in}

$\bullet$ {\bf Publication:} We will, of course, pursue the kinds of
publication venues all good scientists pursue. We will
aim for high-quality (REF 4*/3*) papers and endeavour to publish them in the best
journals. Publication in leading archival journals 
confers validation by the community of the correctness and importance
our our results, and allows them to serve as seminal references.
We expect each package to result in at least two
publications and several to result in more. This is feasible as we
have substantial publication track records.

\vspace*{0.02in} $\bullet$ {\bf Scientific Interaction:} We will
interact other researchers via conferences,
research visits, and other meetings as detailed elsewhere. More
generally, Scotland is an excellent place for interaction as there are
strong programming languages and verification groups at Edinburgh,
Glasgow, Heriot-Watt, Dundee and St Andrews. Similarly Nottingham is a
leading member of the MGS --- a collaboration with Birmingham and
Leicester providing training to young PhD students
from across Europe. Lecturing on our results there will also increase
impact. Strathclyde will also host a specialist
workshop in HoTT at Strathclyde in 2015 which has been partially
funded by the LMS. This workshop is exactly the kind of high-impact
event --- targeted at a small number of experts and in a very specific
area --- that truly aids and disseminates research. Because of the
value of such workshops, we will hold a follow-up workshop during
the project whose impact will be enhanced by inviting our
world leading collaborators. Similarly, to generate impact by
educating others about our results and methodologies, we will hold a
summer school at the end of the project to disseminate our results to
end-users.

\vspace*{0.02in}

$\bullet$ {\bf Collaboration:} To further maximise impact, an
internationally leading expert will collaborate with us on each work
package.  This is advantageous in several ways. First, working with
these collaborators will enhance the scientific quality of our
research. Secondly, their intimate involvement with the proposed
research will ensure our research is fit-for-purpose by ensuring it is
fit-for-purpose for their own high standards. Thirdly, thorough
dissemination will be occurring via our collaborators even as the work
is being done. And, finally, drawing on a variety of different
perspectives will ensure our research is outward-looking and solves
key problems.

Our project partners are Prof~Steve Awodey (CMU, WP1) who is the
leader of the team recently awarded $\$7.5$M to reformulate
mathematics within HoTT; Prof Vladimir Voevodsky (Princeton, WP2) is
regarded as the inventor of the HoTT program because of his Univalence
Axiom, Dr Mike Schulman (UCSD, WP3) invented Higher Inductive
Types, Dr Nick Benton (Microsoft, WP4) who is a leading industrial
user of foundational ideas, Dr Matthieu Sozeau (Paris 7, WP5) is a
leading member of the Coq team, Prof~Thierry Coquand (Chalmers, WP6)
developed the cubical set model of HoTT, Dr Edwin Brady
(St. Andrews, WP7) is the developer of Idris and Dr Andrew Kennedy
(Microsoft, WP8) who invented the Units of Measure type system. One
could hardly imagine a stronger set of collaborators --- in particular
we have an industrial partner for each of the impact generating work
packages while Coquand, Brady and Sozeau are members of the Agda,
Idris and Coq projects and their involvement will ensure impact of our
research via these projects.

\vspace*{0.02in}

$\bullet$ {\bf The Work Packages:} While the main consideration in
developing our work packages is the quality of the scientific ideas
underlying them as necessitated by high-impact research, our work
packages have also been designed to smooth the pathway to achieving
impact from them. For example i) our foundational work packages takes
HoTT as it exists and pushes it towards type theory thereby giving it
more impact amongst theoretical computer scientists who have little or
no knowledge of either homotopy theory or higher dimensional category
theory; ii) by developing HoTT-based programming language and
verification tools, our research opens the way for HoTT to gain impact
amongst these communities where HoTT currently has little impact; and
iii) applying HoTT to Effects in WP4 and to enhancing the
functionality of Units of Measure in WP8 is a concrete pathway to
impact for actual programmers. As WP8 is the most impactful of our
work packages, we describe it now:

\vspace*{0.02in}

$\bullet$ {\bf WP8: HoTT and Units of Measure} Andrew Kennedy of
Microsoft Research recently~\cite{aknn97} incorporated dimensional
analysis into type-checking, thus making it possible to detect
dimensional errors --- such as the units mismatch that caused the Mars
orbiter's famous \$125 million ``metric mishap''~\cite{wp99} --- at
compile time. This feature, dubbed {\em units of measure}, has been
integrated into Microsoft's commercial programming language $F\#$.
Kennedy's key observation is that units of measure form a free abelian
group (i.e., can be multiplied and divided to get, e.g., square metres
or metres per second). Technically, this means Kennedy's type system
was particularly novel in that types are indexed by abelian groups and
therefore that equational theory of abelian groups
had to be lifted to an equational theory of types. Fortunately for
Kennedy, the equational theory of abelian groups has particularly
simple normal forms which allowed him to program up to this equational
theory.

Recently, the scope of units of measure has been expanded to include
programming with coordinate free geometry and programming with
security constraints. This was achieved by so called {\em
  algebraically indexed types}~\cite{ajk} which generalise units of
measure by allowing types to be indexed by other, specific, equational
theories. In this work package we want to lay new foundations for
algebraically indexed types by using HoTT to programming with types
indexed by {\em arbitrary} equational theories. The likelihood of
producing results is high as, conceptually, HoTT contains both the
capacity to define arbitrary equational theories via HITs as well as
the usual type formers present in Units of Measure. However, we have a
more ambitious goal: the problem with algebraically indexed types is
that the equational theory present in types makes type checking an
open problem. We aim to solve this problem by extracting, from the proof
relevant nature of HoTT, a language of evidence to drive type checking
for types indexed by arbitrary equational theories. Even more
excitingly, if we find ways in WP7 of interfacing HoTT with more
mainstream programming languages, this might --- just --- allow us to port
results from HoTT to $F\#$.  For a foundational project, this is a
thrilling prospect!


\vspace*{0.02in}

$\bullet$ {\bf Impact Accelerator Accounts:} IAA grants are funded by
EPSRC to explore the possibility of
developing the impact of EPSRC funded research with non-academic
partners. We will increase potential impact 
by applying for IAA grants at both Nottingham and Strathclyde. Prof
Ghani already has an IAA grant, demonstrating the feasibility of
this. Further, the main prerequisite for IAA grants is the involvement 
of industrial collaborators but as WP4,8 show, we already meet this
criteria. IAA grants are short term (a few months to a year)
and so this will occur towards the end of the grant once it is clear
which results most interest industrial partners.

\vspace*{0.02in}


{\bf Ambition and Realism.} Our {\em Pathways to Impact} are both ambitious
and realistic. They are ambitious because, while it is a long way from
fundamental advances in type theory to routine verification conducted
by software engineers, we have managed to push HoTT all the way to
programming languages and verification tools and even beyond into
actual verification. To expect more, e.g. significantly increasing
industrial involvement, would be both infeasible and unrealistic. That
we have gone this far is partially a testament to our vision
but also a testament to the potential of HoTT. The proposal certainly
has stronger pathways to impact than any previously written by the
investigators. Our plans are also realistic. Apart from our own
abilities, we have members of the Agda (Coquand), Coq (Sozeau) and
Idris (Brady) teams as collaborators, and industrial collaborators
(Benton, Kennedy) for each of our impact generating work
packages. Further, we have significant previous successful experiences
working with Microsoft, e.g.\ Dr McBride's previous Microsoft funded PhD
student, his three month placement at Microsoft during summer 2014,
and Prof Ghani's current IAA grant. Finally, as each collaborator is
an end-user of each work package, we are following EPSRC best-practice of
involving end-users in our impact generating workshops and within the
both the actual design and execution of our research.





% $\bullet$ {\bf The Stream Model:} One prevalent model of research is
% the {\em stream model}. Under this model, to solve a problem one looks
% both ``upstream'' for fundamental theoretical ideas to feed into the
% research, and ``downstream'' for validation of the research by those
% who stand to benefit as end-users. This model maximises impact by
% ensuring that research is picked up not only by those working on the
% same problem, but also by those upstream, who will be interested in
% how their own work is applied, and by those downstream, who will look
% to apply it to their work.  To help ensure that the proposed research
% has a high impact we will follow this stream model.

% Upstream of us there is a significant number of mathematicians and
% theoretical computer scientists working on the foundations of HoTT,
% and whose results we will enhance and apply to develop our environment
% for programming with HoTT. Downstream, those in the programming
% languages and program verification communities are already interested
% in applying HoTT to their work, but the lack of an actual programming
% language hinders them from doing so. Thus our tools will be exactly
% what is needed to help progress the uptake of HoTT by the broader
% programming languages community. 

% \vspace*{0.02in}


% $\bullet$ {\bf Hidden Foundations:} Although the language and
% verification tools produced will be based on HoTT, what users of our
% tools will see is a clean, modern programming language whose use
% requires no knowledge of the underlying foundations of HoTT. We have
% already applied this ``hidden foundations" approach to mathematical
% descriptions of programming language artefacts where the mathematics
% is used to organise the structure of a programming language but not
% used in programming itself. However, even in the foundational work
% packages, we will still use the hidden foundations approach, e.g., by
% comprehensively treating special models of interest and by showcasing
% our results via examples as much as possible. This will allow those
% with modest or no mathematical background to use and profit from
% all of our research to the highest extent possible.

% %\end{itemize}



\vspace{-0.15in}



{%\small



\bibliographystyle{plain}

\begin{thebibliography}{}



\vspace*{-0.25in}



\setlength{\parindent}{0pt}

\setlength{\columnsep}{0.3in}

\setlength{\parskip}{-0.1ex}



\begin{multicols}{2}

%
%
%\bibitem{galorath11}
%Cost of Independent Software Verification \&
%Validation. 2011. At {\tt http://www.galorath.com/wp/}\\
%{\tt cost-of-independent-software-verification}\\
%{\tt -validation-ivv.php}


\bibitem{cnet08}
Total economic cost of insecure software: \$180 billion a year in the
U.S. 2008. At {\tt http://news.}\\
{\tt cnet.com/8301-13846$\_$3-9978812-62.html}

\bibitem{wp99}
Mystery of Orbiter Crash Solved. Available at
{\tt http://www.tysknews.com/Depts/Metrication}\\
{\tt /mystery$\_$of$\_$orbiter$\_$crash$\_$solved.htm} 

\bibitem{aknn97}
A. Kennedy,  {\em Relational parametricity and units of measure},
Procs. of POPL'97, 442-455. ACM Press, 1997.


\bibitem{ajk}
R. Atkey, P. Johann and A. Kennedy, 
{\em 
Abstraction and invariance for algebraically indexed types}
Procs. of POPL'13, 87-100. ACM Press, 2013.



\end{multicols}

\end{thebibliography}



}





\end{document}


%WP8
%First safety critical, then more generally ...
%Balance, IAA, 3 monhs summer, Adam.
%Say we dont do Agda scale systems so Power-esque quotes possible.
%TIC
%Wkshops, Summer Scools, IAA + 2 New IAA
%WP8 does verification
% Changyu

%This project treads a
%fine balance between being well scoped so that we can achieve theirs
%goals within the allocated period of study while simultaneously being
%ambitious enough to have a major and lasting impact upon how we
%compute.  



  \documentclass[a4paper,11pt]{article}



\usepackage[top=2cm, bottom=2cm, left=2cm, right=2cm]{geometry}



\usepackage{mathptmx}

\usepackage{epsf}           %\input{epsf}

\usepackage{amsfonts}

\usepackage{amstext}

\usepackage{amssymb}

\usepackage{url}

\usepackage[dvips]{graphics}

\usepackage[dvips,all]{xy}

\usepackage{multicol}



%\newlength{\extraplusheight}

%\newlength{\extrapluswidth}

%\setlength{\extraplusheight}{4.7cm}

%\setlength{\extrapluswidth}{4.7cm}

%\addtolength{\textwidth}{\extrapluswidth}

%\addtolength{\textheight}{\extraplusheight}

%\addtolength{\oddsidemargin}{-.5\extrapluswidth}

%\addtolength{\evensidemargin}{-.5\extrapluswidth}

%\addtolength{\topmargin}{-0.5\extraplusheight}

%\setlength{\parindent}{0 pt}

%\setlength{\parskip}{1ex}



\newcommand{\Int}[1]{[\![ #1 ]\!]}

\newcommand{\malign}[1]{\begin{array}[t]{@{}l@{\;}l@{}l@{}} #1 \end{array}}

\newcommand{\logrel}[2]{\Delta_{#1,#2}}

\newsavebox{\fminibox}

\newenvironment{fminipage}

% {\begin{lrbox}{\fminibox}\begin{minipage}{8cm}\vspace*{-2ex}}
% {\\[-2ex]\vspace*{-2ex}\end{minipage}\end{lrbox}\noindent\centerline{\fbox{\usebox{\fminibox}}}\vspace{0.5ex}}   



\setlength{\parindent}{0.15in}

%\setlength{\parskip}{0.3ex}



% Discourage unnecessary hyphenation.

\sloppy\hyphenpenalty 4000



\newcommand{\ra}{\rightarrow}

\newcommand{\A}{\mathcal{A}}

\newcommand{\E}{\mathcal{E}}

\newcommand{\C}{\mathcal{C}}

\newcommand{\B}{\mathcal{B}}

\newcommand{\Set}{\mbox{{\sf Set}}}

\newcommand{\Nat}{\mathit{Nat}}

\newcommand{\Alge}[1]{\mathit{Alg}_{#1}}

\newcommand{\hash}{\#}



\begin{document}



\thispagestyle{plain}

\begin{center}

  {\Large {\bf Homotopy Type Theory: Programming and Verification \\
\vspace{0.2in} 
Pathways to Impact}}\\[1ex]   



\vspace*{-0.1in}



  \rule{150mm}{.5mm}\\[2ex]

\end{center}



\noindent



\vspace*{-0.1in}

{\bf Cost of Software Failure:} The cost of software failure is truly
staggering. Well known individual cases include the Mars Climate
Orbiter failure ($\pounds 80$ million), Ariane Rocket disaster
($\pounds 350$ million), Pentium Chip Division failure ($\pounds 300$
million), and more recently the heartbleed bug (est.\ $\pounds 400$
million).  There are many, many more examples. Even worse, other
software failures such as one in the Patriot Missile System and
another in the Therac-25 radiation system have cost lives. More
generally, a 2008 study by the US government estimated that faulty
software costs the US economy $\pounds 100$ billion annually~\cite{cnet08}.
As a result, the human and economic importance of ensuring programs
run without error is hard to over-estimate. Further, this importance
will only grow significantly as software becomes ever more ubiquitous
in our lives and economy.

Formal verification uses mathematical techniques to prove that
programs actually perform the computations they are intended to
perform and/or avoid performing unintended ones. One major approach to
software correctness is {\em language-based verification}, in which
successful compilation of programs provides machine certification of
their correctness. Language-based verification thus supports the
development of software that is {\em correct by construction}, which
is the logical conclusion of the persistent trend in software
engineering toward ever earlier program verification. Thus although
the proposed research is definitely foundational in nature, the fact
that it addresses a problem of such economic importance means that our
Pathways to Impact document is stronger than many similarly
foundational proposals, including dedicated impact generating
workpackages --- each with industrial collaborators --- and plans for
Impact Accelerator Accounts.

\vspace*{0.02in}

$\bullet$ {\bf Publication:} We will, of course, pursue the kinds of
publication venues all good scientists pursue. Principally, we will
produce high-quality papers and endeavour to publish them in the best
journals. Publication of our papers in leading archival journals will
confer validation by the community of the correctness and importance
our our results, as well as allow them to serve as seminal references.
However, the tradition in computer science is also to aim for early
publication of important results to keep pace with the rapidly
changing nature of the discipline. We will therefore also seek to
publish our key results in top conferences. We expect each of our work
packages to result in at least two publications and several to result
in more. This is feasable as we have substantial track records of
doing this.

\vspace*{0.02in} $\bullet$ {\bf Scientific Interaction:} To increase
the impact of our research, we will continue to interact with our
research community via conferences, research visits, and other
meetings. For example, within the UK, Prof Ghani co-founded the
Scottish Category Theory Seminar, while Dr McBride co-founded Fun in
the Afternoon. They are both active members of the Scottish
Programming Languages Seminar and SICSA, the Scottish computer science
pooling body. More generally, Scotland is an excellent place to
interact with other researchers on a regular informal basis: there are
strong programming languages and verification research groups at
Edinburgh, Glasgow, Heriot Watt, and St Andrews, and there are also
Scottish Theorem Proving meetings. Similarly Nottingham is a leading
member of the MGS --- a collaboration with Birmingham and Leicester
which provides educational training to young PhD students from accross
Europe. Lcturing on our results there will increase impact. At a more
detailed level, Strathclyde will host a specialist workshop in HoTT at
Strathclyde in 2015 which has been partially funded by the LMS. This
workshop is exactly the kind of high-impact event --- targeted at a
small number of experts and in a very specific area --- that truly
aids and disseminates research. Our experience is that by careful
choice of time and location, we can attract significant numbers from
both Europe and further afield to such workshops and the
attractiveness of our workshops will be enhanced by inviting key
leaders in the field. Because of the value of such workshops, we
intend to hold a follow-up workshop during the project. Finally, to
generate impact by educating others about our results and
methodologies, we will hold a summer school at the end of the project
which will provide an excellent platform to disseminate our results.




%\item 

\vspace*{0.02in}

$\bullet$ {\bf Collaboration:} To further maximise impact, we have
invited a number of internationally leading project partners to work
with us on specific work packages of the proposed research (see
below). This is advantageous in several ways. First, working
with these project partners will enhance the scientific
quality of the proposed research. Secondly, working with these
partners will mean that they are intimately involved with the proposed
research; in effect, thorough dissemination of our ideas will be
occurring with our partners even as the work is being done, and this
will help get our ideas out into the broader
research community. Finally, drawing on a variety of different
perspectives will help ensure that the proposed research does not
become overly insular, but instead is outward-looking and
solves important problems in key application areas.

Our project partners are Prof~Steve Awodey (CMU, WP1), Prof Vladimir
Voevodsky (Princeton, WP2), Dr Mike Schulman (UCSD, WP3), Dr Nick
Benton (Microsoft, WP4), Dr Matthieu Sozeau (Paris 7, WP5),
Prof~Thierry Coquand (Chalmers, WP6), Dr Edwin Brady (St. Andrews,
WP7) and Dr Andrew Kennedy (Microsoft, WP8). One could not imagine a
stronger set of collaborators --- in particular we have secured the
involvement of an industrial partner for each of the impact generating
work packages while Brady and Sozeau are leaders of the Idris and Coq
projects and their involvement will ensure impact of our research via
these projects. In terms of maximising impact, the collaboration with
Kennedy is the closest we come to actual industrial crossover as
Kennedy's original units of measure system has been incorporated as
code in Microsoft's commercial $F\#$ language. Although it is not the
main focus of the proposed research, we will certainly investigate
this possibility.

\vspace*{0.02in}

$\bullet$ {\bf The Work Packages:} Our main consideration in
developing our work packages is the quality of the scientific ideas
underlying them as necessitated by high-impact research.
For the specific research proposed here it also requires
i) not just a much better understanding of the current
literature on HoTT but substantial contributions to it; ii) not just
theoretical ideas presented in a way that theoreticians can
understand, but also concrete program languages and verification tools
that programmers can understand and use on their own terms; and iii)
not just evidence of the overall power of our new tools, but also a
collection of examples outlining a methodology for deploying them.
These three requirements guided our designing the work packages. WP1-3
make significant contributions to the literature on HoTT while WP4
generates impact from these work packages by developing examples to
illustrate their results. Similarly, WP5-7 focusses on taking the
theoretical results we will achieve and using them to build our
tools. Finally, WP8 generates impact from these tools by developing
case studies where they are used thereby making them
directly usable by non-specialists.  Since WP8 focusses on 
applying our results to existing industrial software, we have the
capacity for generating actual industrial impact within this project
and so we describe it here

\vspace*{0.02in}

$\bullet$ {\bf WP8: HoTT and Units of Measure} Andrew Kennedy of
Microsoft Research recently~\cite{aknn97} incorporated dimensional
analysis into type-checking, thus making it possible to detect
dimensional errors --- such as the units mismatch that caused the Mars
orbiter's famous \$125 million ``metric mishap''~\cite{wp99} --- at
compile time. This feature, dubbed {\em units of measure}, has been
integrated into Microsoft's $F\#$ and, more recently has been
extended~\cite{ajk} to a more general notion of algebraically 

from dimensional analysis to cover geomteric
computation and security analysis.

Kennedy's key observation is that units of measure form a free abelian
group (i.e., can be multiplied and divided to get, e.g., square metres
or metres per second). But in fact the algebraic structure on units
can be generalised to arbitrary equational theories;~\cite{ajk}
develops such {\em generalised units of measure}. Fortunately, the
equational theory of abelian groups has a particualrly simple normal
form but this wont be the case in other algenbra




Significantly, 
each of~\cite{ajk} and~\cite{ken97} define exactly one logical relation
for the unit-indexed language it considers; this hardwires in the
notion of ``relatedness'' of programs, and so is too restrictive. To
remedy this, we propose to develop an axiomatic approach to logical
relations for units of measure. We will first use our axiomatic
framework from WP1 to construct logical relations for Kennedy's
original language; preliminary investigations show that, in addition
to the usual structure required of fibrations, its base category will
need an abelian group object. We will then use our framework to
construct logical relations for the language of~\cite{ajk}. This will
require that the base category has a representing object for the
particular algebraic structure given by the equational theory. The
flexibility of fibrations is crucial because different equational
theories on units require different base categories.



\vspace*{0.02in}

$\bullet$ {\bf Impact Accelerator Accounts:} 

\vspace*{0.02in}


$\bullet$ {\bf The Stream Model:} One prevalent model of research is
the {\em stream model}. Under this model, to solve a problem one looks
both ``upstream'' for fundamental theoretical ideas to feed into the
research, and ``downstream'' for validation of the research by those
who stand to benefit as end-users. This model maximises impact by
ensuring that research is picked up not only by those working on the
same problem, but also by those upstream, who will be interested in
how their own work is applied, and by those downstream, who will look
to apply it to their work.  To help ensure that the proposed research
has a high impact we will follow this stream model.

Upstream of us there is a significant number of mathematicians and
theoretical computer scientists working on the foundations of HoTT,
and whose results we will enhance and apply to develop our environment
for programming with HoTT. Downstream, those in the programming
languages and program verification communities are already interested
in applying HoTT to their work, but the lack of an actual programming
language hinders them from doing so. Thus our tools will be exactly
what is needed to help progress the uptake of HoTT by the broader
programming languages community. 

\vspace*{0.02in}


$\bullet$ {\bf Hidden Foundations:} Although the language and
verification tools produced will be based on HoTT, what users of our
tools will see is a clean, modern programming language whose use
requires no knowledge of the underlying foundations of HoTT. We have
already applied this ``hidden foundations" approach to mathematical
descriptions of programming language artefacts where the mathematics
is used to organise the structure of a programming language but not
used in programming itself. However, even in the foundational work
packages, we will still use the hidden foundations approach, e.g., by
comprehensively treating special models of interest and by showcasing
our results via examples as much as possible. This will allow those
with modest or no mathematical background to use and profit from
all of our research to the highest extent possible.

%\end{itemize}



\vspace{-0.15in}



{%\small



\bibliographystyle{plain}

\begin{thebibliography}{}



\vspace*{-0.25in}



\setlength{\parindent}{0pt}

\setlength{\columnsep}{0.3in}

\setlength{\parskip}{-0.1ex}



\begin{multicols}{2}

%
%
%\bibitem{galorath11}
%Cost of Independent Software Verification \&
%Validation. 2011. At {\tt http://www.galorath.com/wp/}\\
%{\tt cost-of-independent-software-verification}\\
%{\tt -validation-ivv.php}


\bibitem{cnet08}
Total economic cost of insecure software: \$180 billion a year in the
U.S. 2008. At {\tt http://news.}\\
{\tt cnet.com/8301-13846$\_$3-9978812-62.html}

\bibitem{wp99}
Mystery of Orbiter Crash Solved. Available at
{\tt http://www.tysknews.com/Depts/Metrication}\\
{\tt /mystery$\_$of$\_$orbiter$\_$crash$\_$solved.htm} 

\bibitem{aknn97}
A. Kennedy,  {\em Relational parametricity and units of measure},
Procs. of POPL'97, 442-455. ACM Press, 1997.


\bibitem{ajk}
R. Atkey, P. Johann and A. Kennedy, 
{\em 
Abstraction and invariance for algebraically indexed types}
Procs. of POPL'13, 87-100. ACM Press, 2013.



\end{multicols}

\end{thebibliography}



}





\end{document}


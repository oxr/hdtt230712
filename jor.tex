
\documentclass[a4paper,11pt]{article}

\usepackage[top=2cm, bottom=2cm, left=2cm, right=2cm]{geometry}

\usepackage{mathptmx}
\usepackage{epsf}           %\input{epsf}
\usepackage{amsfonts}
\usepackage{amstext}
\usepackage{amssymb}
\usepackage{url}
 \usepackage[dvips]{graphics}
\usepackage[dvips,all]{xy}
\usepackage{multicol}

%\newlength{\extraplusheight}
%\newlength{\extrapluswidth}
%\setlength{\extraplusheight}{4.7cm}
%\setlength{\extrapluswidth}{4.7cm}
%\addtolength{\textwidth}{\extrapluswidth}
%\addtolength{\textheight}{\extraplusheight}
%\addtolength{\oddsidemargin}{-.5\extrapluswidth}
%\addtolength{\evensidemargin}{-.5\extrapluswidth}
%\addtolength{\topmargin}{-0.5\extraplusheight}
\setlength{\parindent}{0 pt}
\setlength{\parskip}{1ex}

\newcommand{\Int}[1]{[\![ #1 ]\!]}
\newcommand{\malign}[1]{\begin{array}[t]{@{}l@{\;}l@{}l@{}} #1 \end{array}}
\newcommand{\logrel}[2]{\Delta_{#1,#2}}
\newsavebox{\fminibox}
\newenvironment{fminipage}
 {\begin{lrbox}{\fminibox}\begin{minipage}{8cm}\vspace*{-2ex}}
 {\\[-2ex]\vspace*{-2ex}\end{minipage}\end{lrbox}\noindent\centerline{\fbox{\usebox{\fminibox}}}\vspace{0.5ex}}   

\setlength{\parindent}{0.15in}
%\setlength{\parskip}{0.3ex}

% Discourage unnecessary hyphenation.
\sloppy\hyphenpenalty 4000

\newcommand{\ra}{\rightarrow}
\newcommand{\A}{\mathcal{A}}
\newcommand{\E}{\mathcal{E}}
\newcommand{\C}{\mathcal{C}}
\newcommand{\B}{\mathcal{B}}
\newcommand{\Set}{\mbox{{\sf Set}}}
\newcommand{\Nat}{\mathit{Nat}}
\newcommand{\Alge}[1]{\mathit{Alg}_{#1}}
\newcommand{\hash}{\#}



\begin{document}

\thispagestyle{plain}
\begin{center}
  {\Large \bf Homotopy Type Theory: Programming and Verification\\
\vspace{0.2in}
  Justification of Resources}\\[1ex]

\vspace*{-0.1in}

\rule{160mm}{.5mm}\\[2ex]
\end{center}

\noindent While EPSRC is principally interested in the quality of
research, costs must be justified. It is argued below that i) all four
investigators are required because they have
distinct-but-necessary skills; ii) the hours per week spent on
the grant by the investigators are required; iii)  two RAs are required because of the
breadth and volume of the work proposed; iv) the duration of the
grant is appropriate given the volume of the work involved and the
fast moving nature of HoTT research; v) the travel budget
is needed to underpin the planned collaborations, workshops, and
attendance at conferences; and vi) a small consumables budget is
needed. Extra value for money arises
as Strathclyde and Nottingham 
each fund and extra PhD student from the overheads of this grant.

\vspace{0.02in}

\noindent {\bf The Investigators:} All investigators have excellent
track records for this research: i) Altenkirch is an expert in type
theory and the syntactic presentation of HoTT, e.g. his work on
$\omega$-groupids, OTT, Cubical Type Theory, Containers, and the
experimental programming language $\Pi\Sigma$ has direct relevance to
WPs 2,3,5,6,7; ii) Gambino is an expert in category theory with his
work on the semantic foundations of HoTT and dependent polynomials
having direct relevance to WPs 1,3; iii) Ghani is an expert on the
categorical foundations of programming languages with his work on
logical relations, containers, effects and units of measure being
particularly relevant to WP1, 2, 3, 4 and 8; and iv) McBride is an
expert on the definition and implementation of programming languages
with his work on containers, OTT, effects and Epigram being
particularly relevant to WPs 3,4,6,7. These skills directly map onto
the management of the proposal as follows. Ghani's role --- apart from
the scientific contributions above --- will be to knit the theoretical
research at Nottingham and Leeds with the applied research at Strathclyde.

\noindent 
We are also a {\em very well balanced team} spanning the breadth of
the proposal from theory to practice and having two implementors of
programming languages --- Altenkirch ($\Pi\Sigma$) and McBride
(Epigram) --- will ensure robust discussions from different
perspectives which will help push the applied research forward.  This
balance makes us one of the few groups who could conduct this
research. Indeed, removing anyone from the project would
significantly reduce the strength of the team and the prospects for
completion. Within the team, Gambino is essential because of his
knowledge of the semantic foundations of HoTT and links with the grant
of Awodey, Altenkirch is essential for his work on the type theoretic
foundations of HoTT, Ghani is essential because of his work on logical
relations and McBride is essential for the depth of his understanding
of the implementation of functional programming languages.  As a
result, having each investigator on the proposed research is
indispensable. Of course, as we have argued, the investigators also
have other overlapping skills and hence all investigators will be
involved in all areas of the research so that we can contribute to,
and be advised of, all progress. In summary, we are an ambitious team
who want to change the world we live in and who believe we have the
mathematical and computational skills to do exactly that.



\vspace{0.02in}

\noindent {\bf Hours per Week:} 
Each investigator will spend 20\% of their time on the project.
This means that on average they will devote one
full day per week the project. These workloads are standard for such projects, and
our experience of research management suggests any less is 
unrealistic. This allows the investigators to devote the significant
time required by the proposed research proposal while still discharging
their other responsibilities.

\vspace{0.02in}



\noindent {\bf The RAs:} 
The volume and sophistication of work is clearly too much to be
undertaken by the investigators alone or by PhD students% .  cannot be
% trained in subjects like category theory, type theory, and the use and
% development of program language and verification tools, and still have
% enough time to conduct the proposed research. 
Full-time RAs are therefore needed on this project. We had originally
hoped that one RA would suffice, but both the amount of the work and
the highly distinct skills needed between the theoretical foundations
and the more practical aspects of the project means it would be
totally unrealistic to expect it to be undertaken by a sole RA. We
will aim for both RAs to have skills in the others area so that each
can benefit from the others work, producing a multiplier effect due to
the benefits of collaboration and cross-fertilisation between their
areas. The benefits will thus be greater than that of having
two RAs work on two separate projects. The RAs will certainly need to
have PhDs and, given the diverse and specialised nature of the
research, will ideally have postdoctoral experience as well.  Such
appointments are feasible because there are a number of PhD students
and RAs working in exactly these areas, e.g., in the Coq and Agda
development teams or who have been following the
numerous summer schools that interest in HoTT has sparked, e.g. at the
Trimester in Certified Proof in Paris and the Special Year on HoTT.  Although the RAs will be expected to be able to work
independently, our own strengths means the RAs will be closely managed
by the investigators. They will be asked to work with the project
partners and collaborators of the investigators, to push the project
ahead; to contribute to the writing of papers and grant proposals;
and, as part of their general training, to help mentor PhD students. A
Grade 7, Point 29 salary of $\pounds$28,839 is the minimum for
newly-minted PhDs, so a Grade 7, Point 35 salary of $\pounds$35,597
will give us the flexibility to employ the more senior RAs who have
indicated their interest in applying. The small extra cost is more
than outweighed by the benefits their greater breadth and depth that
they bring to the project. The balance of the team will be maintained
by hiring one RA on WP1-4 at Nottingham and one RA on
WP5-8 at Strathclyde. The named coinvestigator will be the RA at Nottingham.



\vspace{0.02in}

\noindent {\bf Duration of the Grant:} The appropriate duration for
this project is 4 years. We have proposed 8 very high quality work
packages and while each has a conceptually straightforward deliverable
(to ensure the project can proceed as a whole), it would be a mistake
to underestimate the amount of technical work required to formalise
these concepts to the level of rigour required. On top of this, we
want to tackle the more ambitious goals in our work packages as they
will produce better overall results in terms of the programming tools
developed and the case studies we carry out. Finally, HoTT is a
fast-moving subject and 4 years gives us time to absorb and make use
of breakthroughs likely to arise in the near future.  Moreover, the
time required to implement and formally prove properties within a
proof system is an order of magnitude greater than for equivalent
pencil-and-paper proofs. This is the price of extra rigour. Allocating
4 years to the project gives us reasonable amounts of time to conclude
the work in the various packages --- and also provides sufficient time
at the end of the project to serve as a safety cushion in case any of
the work packages run into unforeseen difficulties.


\noindent Each work package is necessary --- certainly WP1 is needed
to provide the semantic criteria against which the rest of our
research can be judged, WP2 is needed to turn the semantic model of
HoTT from WP1 into a syntactic presentation of HoTT, WP3 is needed
since HITs are one of the exciting new features of HoTT, and WP4 will
be used to generate impact from HITs by applying them to a significant
problem in computer science, namely the theory of effects. WP5 is
needed to formally verify the correctness of our foundations,
programming language tools and case studies, while WP6
and WP7 together form the programming language we aim to develop. The
value of them will be demonstrated with concrete examples in WP8.
Thus, dropping any of the work packages would clearly diminish the
project.

\vspace{0.02in}

\noindent {\bf Travel Budget:} The investigators and the RAs will
attend conferences and workshops relevant to the research:
international conferences such as LICS, ICFP, POPL, Types, TLCA, MPC,
MSFP, TFP, CSL, and ICALP. Each team member will attend two
conferences per year at an estimated average cost of $\pounds 1000$
per trip, for a total of $\pounds 48$K. Within CS, conference
publication is the principal means of assessing quality,
disseminating, and attracting feedback on results. They are also vital
to keeping abreast of the related research. The cost per conference is
accurate since most conferences will be in Europe and cost
approximately $\pounds 300$ for registration, $\pounds 250$ for
travel, $\pounds 200$ for subsistence, and $\pounds 250$ for
accommodation. The cost of the occasional conference further afield
will be offset by savings on occasional conferences in the UK. We have
also asked for $\pounds 32$K for four research visits to each of our
eight collaborators --- their active participation will provide
significant added value while the costs of the visits are minimal when
taken over 4 years. Each visit will be for approximately one week and
the costs are standard covering travel and basic subsistence. Finally,
Dr Gambino held a highly successful UK HoTT meeting this year and Prof
Ghani will host one in Jan 2015. We propose another
workshop half way through the grant and plan to invite each of our
collaborators as invited speakers --- this will increase the quality
of their formal review, the
attractiveness of the workshop and thus the possibilities for
disseminating our results and learning from other researchers. At the
end of the grant, we will hold a summer school to demonstrate and
disseminate our results. Altenkirch, Ghani and McBride have
significant experience of the educational benefits of such 
events through their work in the MGS. We request \pounds 6K for
the workshop and \pounds 6K for the summer school --- these
are very small costs when taken against the benefits of 
facilitating exchange of ideas and generating greater
impact. We also require money for travel within the 
team. We request funds for 2 group meetings per year and for 2
further trips per person per year. As these are all UK trips, we
estimate \pounds 500 per trip. Again these are minimal costs and
certainly required as face-to-face meetings are highly productive.

\vspace{0.02in}

\noindent {\bf Consumables:} The RAs require a PC and also a
laptop because this grant involves and unusually large amount of travel to take
advantage of the input of our collaborators. These are not provided by
our Universities and so one each is requested for each RA. The cost is
$\pounds 4x1000$ --- an entirely standard request with standard
pricing. 

\end{document}




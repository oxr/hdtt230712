\documentclass[a4paper,11pt]{article}

\usepackage[top=2cm, bottom=2cm, left=2cm, right=2cm]{geometry}

\usepackage{mathptmx}
\usepackage{epsf}           %\input{epsf}
\usepackage{amsfonts}
\usepackage{amstext}
\usepackage{amssymb}
\usepackage{url}
 \usepackage[dvips]{graphics}
\usepackage[dvips,all]{xy}
\usepackage{multicol}

%\newlength{\extraplusheight}
%\newlength{\extrapluswidth}
%\setlength{\extraplusheight}{4.7cm}
%\setlength{\extrapluswidth}{4.7cm}
%\addtolength{\textwidth}{\extrapluswidth}
%\addtolength{\textheight}{\extraplusheight}
%\addtolength{\oddsidemargin}{-.5\extrapluswidth}
%\addtolength{\evensidemargin}{-.5\extrapluswidth}
%\addtolength{\topmargin}{-0.5\extraplusheight}
\setlength{\parindent}{0 pt}
\setlength{\parskip}{1ex}

\newcommand{\Int}[1]{[\![ #1 ]\!]}
\newcommand{\malign}[1]{\begin{array}[t]{@{}l@{\;}l@{}l@{}} #1 \end{array}}
\newcommand{\logrel}[2]{\Delta_{#1,#2}}
\newsavebox{\fminibox}
\newenvironment{fminipage}
 {\begin{lrbox}{\fminibox}\begin{minipage}{8cm}\vspace*{-2ex}}
 {\\[-2ex]\vspace*{-2ex}\end{minipage}\end{lrbox}\noindent\centerline{\fbox{\usebox{\fminibox}}}\vspace{0.5ex}}   

\setlength{\parindent}{0.15in}
%\setlength{\parskip}{0.3ex}

% Discourage unnecessary hyphenation.
\sloppy\hyphenpenalty 4000

\newcommand{\ra}{\rightarrow}
\newcommand{\A}{\mathcal{A}}
\newcommand{\E}{\mathcal{E}}
\newcommand{\C}{\mathcal{C}}
\newcommand{\B}{\mathcal{B}}
\newcommand{\Set}{\mbox{{\sf Set}}}
\newcommand{\Nat}{\mathit{Nat}}
\newcommand{\Alge}[1]{\mathit{Alg}_{#1}}
\newcommand{\hash}{\#}



\begin{document}

\thispagestyle{plain}
\begin{center}
  {\Large \bf Homotopy Type Theory: Programming and Verification ---
  Justification of Resources}\\[1ex]

\vspace*{-0.1in}

\rule{160mm}{.5mm}\\[2ex]
\end{center}

\noindent While EPSRC is principally interested in the quality of research,
costs must be justified. It is argued below that i) all four investigators
are required for the proposed research because they have
distinct-but-necessary skills; ii) two RAs are required because of the
breadth and volume of the work proposed; iii) the duration of the
grant is appropriate given the volume of the work involved; iv) the
hours per week spent on the grant by the investigators are required;
v) the travel budget is needed to underpin the planned collaborations,
workshop, and attendance at conferences; and vi) the requested
consumables are both minimal and necessary.

\vspace{0.02in}

\noindent {\bf The Investigators:} All four investigators have
excellent track records in the areas of the proposed research: i)
Altenkirch is an expert in type theory and the syntactic presentation
of HoTT --- in particular his work on OTT, Cubical Type Theory,
Containers and the experimental programming language $\Pi\Sigma$ has
direct relevance to WPs 2,3,6,7; ii) Gambino is an expert in category
theory and the semantic foundations of HoTT with his work on dependent
polynomials having direct relevance to WPs 3; iii) Ghani
is an expert on the categorical foundations of programming languages
with his work on logical relations, containers and units of measure
being particularly relevant to WP1, 2, 3 and 8; and iv) McBride is an
expert on the definition and implementation of programming languages
with his work on containers, OTT, effects and Epigram being
particularly relevant to WPs 3,4,6,7. These skills directly
map onto the management of the proposed research as follows. WP1 will
be lead by Leeds as Gambino is the strongest researcher there, while
WPs 2,3 and 5 will be lead by Nottingham as they reflect strengths of
Dr Altenkirch. Finally WPs 4,6,7 and 8 will be lead by Strathclyde
given their expertise in dependently typed programming. Ghani's role -
apart from specific scientific contributions --- will be to knit the
theoretical research done at Nottingham and Leeds with the programming
languages research done at Strathclyde. 

Their general background and specific experience ranging from
theoretical foundations to concrete implementations mean that the
investigators are --- perhaps --- one of the few groups who could conduct
this research. Indeed, removing any of them from the project would
significantly reduce the strength of the team and the prospects for
completion.
% of the proposed research. 
As a result, having each
investigator on the proposed research is indispensable. Of course, all
investigators will work closely together and thus will in fact be
involved in all areas of the research, even if they are not leading
them, so that they can contribute to, and be advised of, their
progress.



\vspace{0.02in}

\noindent {\bf The RAs:} The volume of work in this substantial
research programme is clearly too much to be undertaken by the
investigators alone. Further, PhD students cannot be trained in
subjects like category theory, type theory, and the use and
development of program language and verification tools, and still have
enough time to conduct the proposed research. Full-time RAs are
therefore needed on this project. We had originally hoped that one RA
would suffice, but both the volume of the work and the highly distinct
skills needed between the theoretical foundations and the more
practical aspects of programming language implementation and
programming verification, mean that we need one RA with skills in each
of these two areas.  The RAs will certainly need to have PhDs and,
given the diverse and specialised nature of the research, will ideally
have postdoctoral experience as well.  Such appointments are feasible
because there are a number of PhD students and RAs working in exactly
these areas, e.g., in the Coq and Agda development teams and also PhD
students who have been following the numerous summer schools that
interest in HoTT has sparked.  Although the RA will be expected to be
able to work independently, our own strengths means the RAs will be
closely managed by the investigators. They will be asked to work with
the project partners and collaborators of the investigators, to push the project ahead; to
contribute to the writing of papers and grant proposals; and, as part
of their general training, to help mentor PhD students. A Grade 7,
Point 29 salary of $\pounds$28,839 is the minimum for newly-minted
PhDs, so a Grade 7, Point 31 salary of $\pounds$30,594 is appropriate
for researchers with the qualifications required. While it may be
possible to employ new PhDs who are appropriate for the project at
Point 29, advertising a Point 31 salary will allow us to attract
researchers with more experience to the position. 


%The position will be
%required for four years to have time to fully integrate the RAs into
%the project, make progress on the proposed research, and then
%evaluate and refine the results obtained. Given the fast moving nature
%of this area of research, this length of time will allow us to absorb
%progress as it is made across the world.

\vspace{0.02in}

\noindent {\bf Duration of the Grant:} We believe that the appropriate
time required to achieve the objectives of this research is 4
years. We have proposed 8 very high quality work packages. While we
believe that some of the work package are conceptually fairly
straightforward given our preliminary investigations and our specific
previous experience, it would be a mistake to underestimate amount the
technical work required to formalise these concepts to the level of
rigour required for a new foundations of programming languages,
robust implementations of those foundations, and case studies
detailing their use. On top of this, several work packages have
ambitious goals 
%we wish to strive for 
which may require conceptual
advances,  and the need to revise basic working assumptions. Finally,
HoTT is a fast-moving subject and four years will give us enough time
to absorb and make use of breakthroughs likely to come in the next few
years. Moreover, as anyone who has worked on an implementation will
confirm, the time required to implement and formally prove properties
within a proof system is an order of magnitude greater than for
equivalent pencil-and-paper proofs. Allocating four years to the
project gives us reasonable amounts of time to conclude the work in
the various packages
% --- this will be a stiff challenge, but we want to
%aim high and already have the ideas to make this feasible 
--- and also
provides sufficient time at the end of the project to serve as a
safety cushion in case any of the work packages run into unforeseen
difficulties.

%Given that four years is a reasonable amount of time to undertake the
%proposed research, one may wonder if 
We firmly believe that all the work packages we propose are
necessary. Certainly WP1 is needed to provide the semantic criteria
against which the correctness of the rest of our research can be
judged, WP2 is needed to turn the semantic model of HoTT from WP1 into
a syntactic presentation of HoTT, WP3 is needed since HITs are one of
the exciting new features of HoTT, and WP4 will be used to generate
impact from HITs by applying them to a significant problem in computer
science, namely the theory of effects. WP5 is needed to ensure the
correctness of our foundation can be machine checked, while WP6 and WP7 
together form the programming language we aim to develop. The value of
them will be demonstrated with concrete examples in WP8.
Thus, dropping any of the work packages would diminish the project as a whole.

\vspace{0.02in}

\noindent {\bf Hours per Week:} %We now address the expenditure
%concerning the time each of the investigators will spend on the grant
%per week. 
Each of the investigators will spend 20\% of their time on the project.
This means that on average they will devote one
full day per week the project. These workloads are standard for such projects, and
our experience of research management suggests that they are
realistic. They will allow the investigators to devote the significant
time required by the proposed research proposal while still fully
engaging with their other responsibilities.

\vspace{0.02in}

\noindent {\bf Travel Budget:} The investigators and the RA will
attend conferences and workshops relevant to the research:
international conferences such as LICS, ICFP, POPL, Types, TLCA, MPC,
MSFP, TFP, CSL, and ICALP, and national conferences such as BCTCS,
SPLS, ScotCats, and Fun in the Afternoon. Each team member will
attend two conferences per year at an estimated average cost of
$\pounds 1000$ per trip, for a total of $\pounds 48000$. Conference
presentation and publication are the principal
means of assessing quality and disseminating
research results in computer science, and are vital to
%attendance is vital to presenting the results of research, and to
keeping abreast of the related research of other researchers. It is
also a cost-effective mechanism for gaining high visibility for our
own research results, and for getting feedback from the community on
how to further meet the requirements of the users of the foundations
we will develop.
%Moreover, conference presentation and publication are the principal
%means of assessing the quality and disseminating the results of
%research in computer science.
%Two conferences per person year is reasonable: 
More than one person
will attend each conference to take maximal advantage of the
networking opportunities such meetings offer. The cost per conference
is accurate since most conferences will be in Europe and cost
approximately $\pounds 300$ for registration, $\pounds 250$ for
travel, $\pounds 200$ for subsistence, and $\pounds 250$ for
accommodation. The cost of the occasional conference further afield
%(for example, in the US) 
will be offset by savings on occasional conferences in the UK. We
estimate that 25\% of the meetings will be within the UK and 75\% will
be international.  We have also asked for $\pounds 16000$ for research
visits to our eight project partners; we believe the active
collaboration of other world leading experts in the proposed research
will provide significant added value, and the costs of the research
visits to enable this collaboration are relatively minimal when taken
over a four-year period. We envisage two visits overall to each of our
partners. Each visit will be for approximately one week, and two team
members will participate in each visit.
%a two-week visit per team member per year, with each visit costing
%$\pounds 1750$. Of these visits, roughly 80\% will be in the UK and
%20\% in Denmark.
These are entirely standard costs covering travel and basic
subsistence. Finally, Dr Gambino has held --- and Prof Ghani will organise --- a workshop on HoTT
this year expected to be highly successful. To engender greater impact and
facilitate the exchange of ideas, we propose to hold another such
workshop, and so request $\pounds 2000$ to pay for speaker costs and
$\pounds 750$ for catering.

\vspace{0.02in}

\noindent {\bf Consumables:} The RA will require a PC. However, PCs
for RAs are not provided by the University, so we request one
here. The cost for this is $\pounds 1000$. This is an entirely
standard request with standard pricing.

\end{document}


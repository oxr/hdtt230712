

\documentclass[a4paper,11pt]{article}

\usepackage[top=2cm, bottom=2cm, left=2cm, right=2cm]{geometry}

\usepackage{mathptmx}
\usepackage{epsf}           %\input{epsf}
\usepackage{amsfonts}
\usepackage{amstext}
\usepackage{amssymb}
\usepackage{url}
 \usepackage[dvips]{graphics}
\usepackage[dvips,all]{xy}
\usepackage{multicol}

\usepackage{microtype}
\usepackage{hyperref}

%\newlength{\extraplusheight}
%\newlength{\extrapluswidth}
%\setlength{\extraplusheight}{4.7cm}
%\setlength{\extrapluswidth}{4.7cm}
%\addtolength{\textwidth}{\extrapluswidth}
%\addtolength{\textheight}{\extraplusheight}
%\addtolength{\oddsidemargin}{-.5\extrapluswidth}
%\addtolength{\evensidemargin}{-.5\extrapluswidth}
%\addtolength{\topmargin}{-0.5\extraplusheight}
\setlength{\parindent}{0 pt}
\setlength{\parskip}{1ex}

\newcommand{\Int}[1]{[\![ #1 ]\!]}
\newcommand{\malign}[1]{\begin{array}[t]{@{}l@{\;}l@{}l@{}} #1 \end{array}}
\newcommand{\logrel}[2]{\Delta_{#1,#2}}
\newsavebox{\fminibox}
\newenvironment{fminipage}
 {\begin{lrbox}{\fminibox}\begin{minipage}{8cm}\vspace*{-2ex}}
 {\\[-2ex]\vspace*{-2ex}\end{minipage}\end{lrbox}\noindent\centerline{\fbox{\usebox{\fminibox}}}\vspace{0.5ex}}   

\setlength{\parindent}{0.15in}
%\setlength{\parskip}{0.3ex}

% Discourage unnecessary hyphenation.
\sloppy\hyphenpenalty 4000

\newcommand{\ra}{\rightarrow}
\newcommand{\A}{\mathcal{A}}
\newcommand{\E}{\mathcal{E}}
\newcommand{\C}{\mathcal{C}}
\newcommand{\B}{\mathcal{B}}
\newcommand{\Set}{\mbox{{\sf Set}}}
\newcommand{\Nat}{\mathit{Nat}}
\newcommand{\Alge}[1]{\mathit{Alg}_{#1}}
\newcommand{\hash}{\#}



\begin{document}

\thispagestyle{plain}
\begin{center}
  {\Large \bf Homotopy Type Theory: Programming and Verification\\
\vspace{0.2in}
  Justification of Resources}\\[1ex]

\vspace*{-0.1in}

\rule{160mm}{.5mm}\\[2ex]
\end{center}

\noindent While EPSRC is principally interested in the quality of
research, costs must be justified. It is argued below that i) all four
investigators are required because they have
distinct-but-necessary skills; ii) the hours per week spent on
the grant by the investigators are required; iii)  two RAs are required because of the
breadth and volume of the work proposed; iv) the duration of the
grant is appropriate given the volume of the work involved and the
fast moving nature of HoTT research; v) the travel budget
is needed to underpin the planned collaborations, workshops, and
attendance at conferences; and vi) a small consumables budget is
needed. Extra value for money arises
as Strathclyde and Nottingham 
will each fund an extra PhD student from the overheads of this grant, and
because each of our 8 collaborators are giving 4 weeks  to
this project, amounting to almost an extra person year.

\vspace{0.02in}

\noindent {\bf The Investigators:} 
All investigators have excellent track records for this research as is
detailed in the main case for support.  We are also a {\em very well
  balanced team} spanning the breadth of the proposal from theory to
practice. Having two implementors of programming languages ---
Altenkirch ($\Pi\Sigma$) and McBride (Epigram) --- will ensure robust
discussions from different perspectives which will help push the
applied research forward. Ghani, Altenkirch and Gambino's varied
theoretical expertises will similarly push forward the more
foundational research. The mathematical sophistication of HoTT, and
the intricacies of implementation, mean this depth and breadth of
knowledge is essential to the project, and makes us one of the few groups
who could conduct this research. This also partially explains why this
clearly important research hasn't been undertaken yet.

\noindent Although the investigators posses overlapping areas of expertise as
detailed above, they also uniquely posses a depth of knowledge
in a vital area of the project. This means that removing any
investigator from the project would significantly reduce the strength
of the team and the prospects for successful completion. For example,
Gambino is essential because no other investigator possesses his depth
of knowledge of the semantic foundations of HoTT or his links with the
grant of Awodey. Similarly, Altenkirch is essential as no other
investigator possesses his depth of knowledge, or preparatory work, on
the syntactic presentation of HoTT. Ghani is similarly essential because
of the depth of his knowledge of logical relations and McBride is
similarly essential for the depth of his understanding of the
implementation of functional programming languages. On the other hand,
the investigators' overlapping skills ensures that all investigators
will be involved in all areas of the research so that we can work
as one unified team rather than 4 individual investigators. This is
feasible because Gambino and Ghani have some experience with
Coq, Haskell and Agda, and McBride's facility with OTT means he has some experience with
the ideas and motivations of HoTT. Ghani's role --- apart from his
scientific contributions --- will be to use his location at
Strathclyde and knowledge of the foundational areas of the
project to knit the theoretical research at Nottingham and Leeds with
the applied research at Strathclyde. Overall, we are an ambitious
team who want to change the world we live in and believe we that,
collectively, we have the mathematical and computational skills to do
that within this ground breaking application of HoTT to programming languages.

\vspace{0.02in}

\noindent {\bf Hours per Week:} 
Each investigator will spend 20\% of their time on the project.  This
means that on average they will devote one full day per week the
project. These workloads are standard for such projects, and our
experience of research management suggests any less is totally
unrealistic given the volume of the work required and the time needed
to keep abreast of other developments in the project and elsewhere in
the world.  On the other hand, a commitment of 20\% of their time
allows the investigators to devote the significant time required by
the proposed research while still discharging their other essential
responsibilities.

\vspace{0.02in}

\noindent {\bf The RAs:} 
The volume of work required is clearly too much to be undertaken by
the investigators alone; its sophistication makes it unsuitable
for PhD students. Full-time RAs are therefore needed. We 
originally hoped one RA would suffice, but both the volume of the
work (especially within implementation, verification and the
development of inherently open-ended case studies) and the highly
distinct skills needed between the foundational and the
more practical aspects of the project, means it is totally unrealistic
to expect it to be undertaken by one RA. Thus we definitely need two
RAs and the balance of the team will be maintained by hiring one RA at
Nottingham on WP1-4, and one RA at Strathclyde on WP5-8. These RAs
need some skills in the other's area so that each can benefit from, and
contribute to, research across the project. This is vital given our
desire to work as one cohesive unit and also produces a multiplier
effect due to the benefits of collaboration and cross-fertilisation
between the RAs. Certainly, the benefits will be greater than that of
having two RAs work on two separate projects --- two 
projects is not possible anyway as each half of the project is
inextricably linked with the other. The RAs will also need skills to
work with our collaborators, to contribute to the writing of papers
and grant proposals; and, as part of their general training, to help
mentor the PhD students funded by Strathclyde and Nottingham.  A Grade
7, Point 29 salary of $\pounds$28,839 is the minimum for newly-minted
PhDs, so a Grade 7, Point 35 salary of $\pounds$35,597 will give us
the flexibility to employ the more senior RAs who have the 
skill set (outlined above) required by this project. The small extra
cost is more than outweighed by the associated benefits. Point 35
appointments are feasible because there are a number of RAs who are
working in these areas (e.g. in the Coq and Agda development teams) or
who have been following the numerous summer schools that HoTT has
sparked (e.g. at the Trimester in Certified Proof in Paris and the
Special Year on HoTT). Examples include Dr Ondrej Rypacek, Dr Ohad Kammar and Dr James
Chapman who have indicated their interest in this project.
\vspace{0.02in}

\noindent {\bf Duration of the Grant:} The appropriate duration for
this project is 4 years. We have proposed 8 very high quality work
packages and while each has an achievable deliverable
(to ensure the project can proceed as a whole), it would be a mistake
to underestimate the amount of technical work required to formalise
these concepts to the level of rigour required. On top of this, we
want to tackle the more ambitious goals in our work packages as they
will produce better overall results in terms of the programming tools
developed and the case studies we carry out. Finally, HoTT is a
fast-moving subject and four years gives us time to absorb and make use
of breakthroughs likely to arise in the near future.  Moreover, the
time required to implement and formally prove properties within a
proof system is an order of magnitude greater than for equivalent
pencil-and-paper proofs. This is the price of extra rigour. Allocating
four years to the project gives us reasonable amount of time to conclude
the work in the various packages --- and also provides sufficient time
at the end of the project to serve as a safety cushion in case any of
the work packages run into unforeseen difficulties.


\noindent Each work package is vital: WP1 provides
the semantic criteria against which the rest of our
research can be validated, WP2 is needed to turn the semantic model of
HoTT from WP1 into a syntactic presentation of HoTT, WP3 is needed
since HITs are one of the exciting new features of HoTT, and WP4 will
be used to generate impact from HITs by applying them to a significant
problem in computer science, namely the theory of effects. WP5 is
needed to machine check the correctness of our foundations,
programming language tools and case studies, while WP6,7
together form the programming language we aim to develop. The
value of them will be demonstrated within WP8.
Thus, dropping any of the work packages would clearly diminish the
project.

\vspace{0.02in}

\noindent {\bf Travel Budget:} Within CS, conference publication is
vital for assessing the quality of research, disseminating its
results, attracting feedback and keeping abreast of the related
research.  We will attend two conferences per year at an estimated
cost of $\pounds 1000$ per trip, for a total of $\pounds
48$K. This is accurate since most conferences will
be in Europe and cost approximately $\pounds 300$ for registration,
$\pounds 250$ for travel, $\pounds 200$ for subsistence, and $\pounds
250$ for accommodation. The cost of occasional conferences further
afield is offset by occasional conferences in the UK. Example
conferences are LICS, ICFP, POPL, Types, TLCA, MPC, MSFP, TFP, CSL,
and ICALP.  We have also asked for $\pounds 32$K for 4 research visits
to each of our 8 collaborators --- their active participation will
provide significant added value while the cost is minimal 
over 4 years. Each visit will be approximately one week and the costs
are standard covering travel and basic subsistence. Dr Gambino held a
highly successful UK HoTT meeting this year and Prof Ghani will host
one in Jan 2015. We propose another workshop half way through the
grant and plan to invite our collaborators as invited speakers.
Their presence during our talks will increase the quality of their
formal review. Their presence also increases the attractiveness of the
workshop and thus opportunities for dissemination to, and feedback
from, attendees. At the end of the grant, we will hold a summer school
to demonstrate and disseminate our results. Altenkirch, Ghani and
McBride have significant experience of the educational benefits of
such events through their work in the MGS. We request \pounds 6K for
the workshop and \pounds 6K for the summer school --- these are very
small costs when taken against the benefits of facilitating exchange
of ideas and generating greater impact. We also require money for
travel within the team. We request funds for 2 group meetings per year
and for 2 further trips per person per year. As these are all UK
trips, we estimate \pounds 500 per trip giving a total of \pounds
48K. Again these are minimal costs and certainly required as
face-to-face meetings are highly productive. Overall, the travel
budget is about 10\% of the entire budget which is standard.

\vspace{0.02in}

\noindent {\bf Consumables:} The RAs require a PC and also a
laptop because this grant involves an unusually large amount of travel to take
advantage of the input of our collaborators. These are not provided by
our Universities and so one each is requested for each RA. The cost is
$\pounds 4x1000$ --- an entirely standard
pricing. 

\end{document}




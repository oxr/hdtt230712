\documentclass[a4paper,11pt]{article}

\usepackage[top=2cm, bottom=2cm, left=2cm, right=2cm]{geometry}

\usepackage{mathptmx}
\usepackage{epsf}           %\input{epsf}
\usepackage{amsfonts}
\usepackage{amstext}
\usepackage{amssymb}
\usepackage{url}
\usepackage[dvips]{graphics}
\usepackage[dvips,all]{xy}
\usepackage{multicol}
\usepackage{natbib}
\setlength{\bibsep}{0.0pt}


%\newlength{\extraplusheight}
%\newlength{\extrapluswidth}
%\setlength{\extraplusheight}{4.7cm}
%\setlength{\extrapluswidth}{4.7cm}
%\addtolength{\textwidth}{\extrapluswidth}
%\addtolength{\textheight}{\extraplusheight}
%\addtolength{\oddsidemargin}{-.5\extrapluswidth}
%\addtolength{\evensidemargin}{-.5\extrapluswidth}
%\addtolength{\topmargin}{-0.5\extraplusheight}
\setlength{\parindent}{0 pt}
\setlength{\parskip}{1ex}

\newcommand{\Int}[1]{[\![ #1 ]\!]}
\newcommand{\malign}[1]{\begin{array}[t]{@{}l@{\;}l@{}l@{}} #1 \end{array}}
\newcommand{\logrel}[2]{\Delta_{#1,#2}}
\newsavebox{\fminibox}
\newenvironment{fminipage}
 {\begin{lrbox}{\fminibox}\begin{minipage}{8cm}\vspace*{-2ex}}
 {\\[-2ex]\vspace*{-2ex}\end{minipage}\end{lrbox}\noindent\centerline{\fbox{\usebox{\fminibox}}}\vspace{0.5ex}}   

%\setlength{\parindent}{0.15in}
%\setlength{\parskip}{0.3ex}

% Discourage unnecessary hyphenation.
\sloppy\hyphenpenalty 4000

\newcommand{\ra}{\rightarrow}
\newcommand{\A}{\mathcal{A}}
\newcommand{\E}{\mathcal{E}}
\newcommand{\C}{\mathcal{C}}
\newcommand{\B}{\mathcal{B}}
\newcommand{\Set}{\mbox{{\sf Set}}}
\newcommand{\Nat}{\mathit{Nat}}
\newcommand{\Alge}[1]{\mathit{Alg}_{#1}}
\newcommand{\hash}{\#}

\begin{document}

\thispagestyle{plain}
\begin{center}
  {\Large {\bf Univalent Type Theory, Programming and Verification:}}\\[1ex] 

\vspace*{-0.1in}

%  {\Large \bf Case for Support}\\[1ex]
  \rule{140mm}{.5mm}\\[2ex]
\end{center}

\noindent
{\bf \Large Part 1A: Previous Research \& Track Record}

\textbf{Professor Neil Ghani.} Neil Ghani earned his PhD in Computer
Science in 1995 from the University of Edinburgh, where he worked on
categorical models of rewriting.  After postdoctoral research at the
\'{E}cole Normale Sup\'{e}rieure in Paris, he was a Lecturer at
Leicester and then a Reader at Nottingham. In 2008, he was appointed
to a professorship at the University of Strathclyde where he founded
the Mathematically Structured Programming group. His research has
focussed on a number of topics directly related to the subject of the
current proposal, Homotopy Type Theory (HoTT). In particular: (1) his
work on data types (containers, quotient containers, indexed
containers and induction recursion) is directly related to our plans
in WP3 to develop the theory of higher inductive types; (2) his work
on logical relations is directly related to our plans in WP1 and WP2
to develop the model theory for HoTT as well as the syntactic
presentation of this model theory as a type theory; (3) his work on
the semantics of effects means he is an ideal person to develop the
application of HoTT to effects as planned in WP4; (4) his work on
Units of Measure (in collaboration with Microsoft) directly relates to
WP8 where we will apply our development of HoTT to computing in the
presence of algebraic structures such as is found in Units of
Measure. Indeed, results obtained during his previous EPSRC grants on
Containers, Induction Recursion and Logical Relations will feed into
this proposal.  More generally, Neil Ghani is a world expert on the
use of semantic structures to drive the development of type theory and
programming languages - exactly the methodology to be followed in this
proposal.

Professor Ghani serves as a member of the EPSRC College charged with
assessing the quality of grant applications. He is also a grant
assessor for the Carnegie Trust.  He was the director of the Midlands
Graduate School in the Foundations of Computer Science, is a SICSA
theme leader in Complex Systems Engineering, and was on the Steering
Committee for the British Colloquium on Theoretical Computer Science.
He has served on numerous programme committees and been the external
examiner for a number of PhD students. He is also Deputy Head of
Department and Head of Research in his home Department. He has
successfully supervised five PhD students and four RAs and has
experience in the successful management of large grants. He organises
the Scot Cats seminar series, and plans to host a meeting in the UK on
Homotopy Type Theory in November 2014, both of which puts him in touch
with other experts in the area.

For further information, see~\url{http://www.cis.strath.ac.uk/~ng}

\textbf{Host Institution: The University of Strathclyde.} The
Mathematically Structured Programming group at the University of
Strathclyde is an ideal venue for conducting this research. Led by
Neil Ghani, the group includes Dr Conor McBride and Dr Clemens Kupke,
two research associates and eight PhD students. Dr Conor McBride's
expertise is of particular value for the current proposal since it
ensures local access to a programming perspective that complements
their more foundational one, and Epigram, a state-of-the-art
dependently-typed programming system of the sort intended to
incorporate the results of the proposed research.  Finally, central
Scotland is home to vibrant theoretical computer science and
functional programming research communities, and the University of
Strathclyde is an active participant in the Scottish Informatics and
Computer Science Alliance (SICSA). The MSP group also is active in
other Scottish meetings, such as ScotCats and SPLS.



\textbf{Thorsten Altenkirch.}  Thorsten Altenkirch received his PhD in
Computer Science from the University of Edinburgh in 1993. Since 2006
he is Reader in Computer Science at the University of Nottingham,
where he founded the Functional Programming Laboratory with Graham
Hutton in 2008. Altenkirch is well known for his work on type theory
and applications of category theory in computer science, and has
published over 50 research papers (all available online via Google
scholar) which are frequently cited (h-index~$\geq 26$). During his
work at Nottingham he attracted \pounds 1M in research funding,
comprising~\pounds650,903 as PI in 4 EPSRC grants, \pounds241,075 as
CoI in 2 EPSRC grants, \pounds159,038 in 1 fellowship and 1
studentship. Especially relevant for the current project is
Observational Equality For Dependently Typed Programming
(EP/C512022/1), Theory And Applications of Induction Recursion
(EP/G03298X/1), Reusability and Dependent Types
(EP/G034109/1). Altenkirch and Ghani have alreadt collaborated
successfully on three research grants.

Altenkirch is one of the leading researchers on Homotopy Type Theory.
This is witnessed by his fellowship at the Institute of Advanced Study
in Princeton in occasion of the special year on the subject in spring
2013. There, he contributed to the standard reference on the
subject~\cite{hott-book}.  He has given invited lectures on the
subject (at HDACT in Lubljana in 2012, at the Curien-fest in Venice in
2013, at MSC in Lyon and at the Institut Henri Poincar\'e in Paris in
2014). He has also published a number of papers related to the subject
\cite{altenkirch:extSetoids,alti:ott-conf,alti:csl12,alti:tlca13-hedberg}.

For further information, see~\url{http://www.cs.nott.ac.uk/~txa}

\textbf{Host institution: The University of Nottingham.} The School of
Computer Science at the University of Nottingham is a research-led
School in one of the leading universities in the UK. The School was
ranked 8th in the last Research Assessment Exercise, and the
Functional Programming Lab within the School is one of its four major
research groups, with an international reputation for its work on
formally-based approaches to software construction and verification.
The FP lab currently comprises 4 academic staff: Professor Graham
Hutton, Dr Thorsten Altenkirch, Dr Venanzio Capretta, and Dr Henrik
Nilsson and nine PhD students.  To date, the group has
received~\pounds1.5M of EPSRC funding over 14 projects, and has 12
completed PhD students.  The Functional Programming Lab provides a
highly stimulating research environment for researchers and PhD
students with weekly research meetings and frequent seminars.
\noindent


\textbf{Dr Nicola Gambino.} Nicola Gambino received his PhD in
Computer Science from the University of Manchester in 2002. After
carrying out postdoctoral research at the University of Cambridge and
the Universit\'e du Qu\'ebec \`a Montr\'eal, he was appointed to an
Assistant Professorship in Mathematical Logic at the University of
Palermo in 2008. Since September 2012, he is Associate Professor in
Pure Mathematics at the University of Leeds. Nicola Gambino has a
consistent record of publications in leading journals and is
frequently invited to speak at international conferences. In
particular, he was a plenary invited speaker at the 2010 Logic
Colloquium in Paris and will be an invited speaker at the joint 2014
RTA-TLCA conference in Vienna. He has held several visiting positions
at prestigeous research institutions, including the Institut
Mittag-Leffler (Stockholm), the Fields Institute (Toronto) and the
Institute of Advanced Study (Princeton), where he worked in contact
with Vladimir Voevodsky on topics connected to the current
proposal. He is currently an invited researcher to the Institut Henri
Poincar\'e in occasion of the special trimester on Semantics of Proofs
and Certified Mathematics.  He serves on the editorial board of the
journal Mathematical Structures in Computer Science. He has been
funded by EPSRC via a Postdoctoral Fellowship in Mathematics, which he
held as Principal Investigator at the University of Cambridge
(2003-04).

Dr Gambino has expertise and experience in all the areas of the
proposed research, which he gained working under the mentorship of
some of the leading scholars in these areas, including Peter Aczel,
Martin Hyland and Andr\'e Joyal. His joint research with Richard
Garner obtained one of the first results relating precisely type
theory and homotopy theory, while his recent work with Steve Awodey
and Kristina Sojakova on inductive types in Homotopy Type Theory is
directly relevant for the proposed research.

For further information,  see~\url{http://www.maths.leeds.ac.uk/~pmtng}.


\textbf{Host Institution: The University of Leeds.} The University of
Leeds is one of the leading universities in the UK and provides
excellent facilities for research. The School of Mathematics, which
has a strong record of research in pure and applied mathematics, hosts
the Mathematical Logic group, which includes seven members of staff,
four research fellows and nineteen research students. The group has
established itself as one of the leading groups for research in
mathematical logic at the international level. The group has a vibrant
research activity; in particular, it organises frequently
international conferences and runs several regular activities,
including a weekly Mathematical Logic Colloquium as well as specialist
seminar series in model theory, proof theory and computability
theory.The Mathematical Logic group has been consistently funded by
EPSRC and international agencies.

The group is already active on Homotopy Type Theory. In particular,
Professor Michael Rathjen is currently PI on a 3-year EPSRC grant on
proof-theoretical aspects of the subject and we expect close
collaboration with him and the research associate employed on the
project. The group includes also another research associate (funded by
a grant from the US Air Force Office for Scientific Research) working
on Homotopy Type Theory under the direction of Dr Nicola Gambino.

%{\bf \Large Part 1A: Previous Research \& Track Record - 
%Thorsten Altenkirch}

% \vspace{0.05in}

% \noindent
% See also \url{http://www.cs.nott.ac.uk/~txa/}

% \vspace{0.05in}
% Thorsten Altenkirch received his PhD from the University of
% Edinburgh in 1993 since October 2006 he has been a Reader in
% Computer Science at the University of Nottingham and in October
% 2008 he founded  the Functional Programming Laboratory together with
% Graham Hutton.

% Altenkirch is well known for his work on Type Theory and applications
% of category theory in computer science, and has published over 50
% research papers (all available online via google scholar) which are
% frequently cited (h-index $\geq$ 26). During his work at Nottingham he attracted
% \pounds 1M in research funding, comprising \pounds 650,903 as PI in 4
% EPSRC grants, \pounds 241,075 as CoI in 2 EPSRC grants, \pounds
% 159,038 in 1 fellowship and 1 studentship. Especially relevant for the
% current project is Observational Equality For Dependently Typed Programming
% (EP/C512022/1), Theory And Applications of Induction Recursion (EP/G03298X/1),
% Reusability and Dependent Types (EP/G034109/1). Altenkirch and Ghani have been  collaborating on
% three research grants.

% Altenkirch is one of the leading researchers on the emerging subject of Homotopy Type Theory,
% this is witnessed by his fellowship at the Special Year on Homotopy
% Type Theory at the Institute of Advanced Study in Spring 2013, where he contributed to the standard reference on the subject~\cite{hott-book}. 
% He has 
% having given invited lectures on the subject (in 2012 at HDACT in
% Lubljana, 2013 at the Curien-fest in Venice, in 2014 at MSC in Lyon
% and at the IHP in Paris and published a number of papers related to the subject 
% \cite{altenkirch:extSetoids,alti:ott-conf,alti:csl12,alti:tlca13-hedberg}.

% \noindent
% {\bf \Large The Host Institution: University of Nottingham}

% \vspace*{0.05in}
% The School of Computer Science at the University of Nottingham
% is a research-led School in one of the leading Universities in
% the UK. The School was ranked 8th in the last Research Assessment
% Exercise, and the Functional Programming Lab within the School is
% one of four major research groups, with an international reputation
% for its work on formally-based approaches to software construction
% and verification.  The FP lab currently comprises 4 academic staff
% (Thorsten Altenkirch, Venanzio Capretta, Graham Hutton, and Henrik
% Nilsson) and 9 PhD students.  To date the
% group has received \pounds 1.5M of EPSRC funding over 14 projects,
% and has 12 completed PhD students.

% The Functional Programming Lab provides a highly stimulating
% research environment for researchers and PhD students with weekly
% research meetings and frequent seminars. 
% \noindent


% \pagebreak

% \noindent
% {\bf \Large Part 1B: Previous Research \& Track Record -
% Prof Neil Ghani}

% \vspace{0.05in}

% \noindent
% See also \url{http://www.cis.strath.ac.uk/~ng}

% \noindent
% {\bf \Large The Host Institution} 

% \vspace*{0.05in}

% \noindent
% The MSP group at the University of Strathclyde is an ideal venue for
% conducting this research. Dr~Johann and Prof~Ghani have a
% well-established, productive collaboration centred on using
% mathematical structures to guide programming languages research, and
% are internationally known for their work in $\lambda$-calculi,
% category theory, functional programming, and logical
% relations. Moreover,~Dr Conor McBride's membership in the MSP group
% ensures local access to a programming perspective that
% complements their more foundational one, and Epigram, a state-of-the-art
% dependently-typed programming system of the sort intended to
% incorporate the results of the proposed research. The MSP group also has
% eight PhD students and two RAs.  The strength of the group has
% been recognised by the University of Strathclyde, which is just now
% hiring a new Lecturer for it.  Finally, central Scotland is home to
% vibrant theoretical computer science and functional programming
% research communities, and the University of Strathclyde is an active
% participant in the Scottish Informatics and Computer Science Alliance
% (SICSA). The MSP group also is active in other Scottish meetings, such
% as ScotCats and SPLS.

\newpage

\noindent
{\bf \Large Part 2: The Proposed Research and Its Context}

\vspace*{-0.23in}

\begin{center}
\rule{170mm}{.5mm}
\end{center}

\vspace*{-0.3in}

\section{Introduction}\label{sec:intro}

\vspace*{-0.1in} 

%txa: added heartbleed.
%{\bf Cost of Software Failure:} The cost of software failure is truly
%staggering. Well known individual cases include the Mars Climate
%Orbiter failure ($\pounds 80$ million), Ariane Rocket disaster
%($\pounds 350$ 
%million), Pentium Chip Division failure ($\pounds 300$ million), most recently
%the heartbleed bug (upto $\pounds 250$K per server) and there are
%many, many more examples. Even worse, other software failures such as
%one in the Patriot Missile System and another in the Therac-25 radiation system
%have costs lives. More generally, a 2008 study by the US
%government estimated that faulty software costs the US economy
%$\pounds 40$ billion annually.  As a result, the human and economic
%importance of ensuring programs run without error is hard to
%over-estimate.
 

%The worldwide software market is estimated at
%\pounds 250 billion pounds every year and this figure will grow
%significantly in real terms as software becomes ever more ubiquitous
%in our lives and economy. Although the requirements for software vary
%enormously, a problem common to all software is to ensure programs run
%without error. Restarting a phone is a simple, if inconvenient task;
%restarting an aeroplane in mid-flight is not an option! Numerous
%examples of expensive software errors about from the Mars Rover to be
%recent Heartbleed Bug exposing a serious vulnerability in the popular
%OpenSSL cryptographic software library.  In a nutshell, the cost of software bugs is almost
%unbelievably staggering.

{\bf Formal Verification:} The cost of software failure is truly
staggering\footnote{see the section on National
  Importance.}. Traditional methods of software verification based
upon testing only generate partial guarantees of correctness. Stronger
guarantees of software correctness are given by mathematical proofs
but the complexity of modern software means that hand written
mathematical proofs are untrustworthy. As a result, the only way to
ensure truly secure and reliable software is the gold standard of
formally verified software, which offers machine-checked mathematical
proofs of software correctness. Several decades of pioneering work in
the UK and elsewhere have culminated in prototype languages and tools
such as Agda, Epigram, Idris and Coq and in the US NuPRL, Twelf, and
the Trellys project. These systems are beginning to make their mark,
eg Coq has just won both the ACM Software award and the SIGPLAN
Programming Languages Software award.
%~\cite{}.  -not needed, txa
The advanced
type-theoretic technology of these systems is raided by more
mainstream languages with significant industrial deployment such as
Haskell, OCaml, Scala and C\#.

{\bf The Project:} Despite these successes, such systems have a number
of short comings in crucial areas, eg when programming with quotient
types, supporting abstraction (that is, invariance under different
representations of the same structure) and extensional reasoning
(proving that programs which behave the same are the same).  {\em
  Homotopy Type Theory} (HoTT) is widely regarded as a fundamental
innovation with great potential to address these and other problems.
Delivering this potential requires i) further
development of the foundations of HoTT; ii)
a programming language and verification environment based on
these foundations; and iii) applications demonstrating the effectiveness
of this environment to the broader software development community.
Therefore we intend to pursue a three-pronged
research programme based upon a synthesis of theoretical, applied and
impact-focussed research.

%The key concept in HoTT is to introduce a very liberal
%equality which identifies structures that are replacable.

%Its core is Voevodsky's {\em Univalence Axiom} asserting
%that all provable equalities can be internalised as paths or, more
%conceptually, that computation is invariant under change of
%representation of the entities we compute with. 



\begin{itemize}
\item {\bf Theoretical Foundations:} Almost all of our understanding
  of HoTT exists within a classical framework and hence cannot be used
  to develop programming language and verification tools. Nevertheless, the recent
  work by Coquand et al. strongly suggests a constructive presentation of HoTT
  is possible. We will develop both specific constructive models
  and a general constructive model theory of HoTT, and then complement
  these models with type theoretic presentations of them.
\item {\bf Programming Languages and Verification:} The key
  deliverable of the second strand of our research will be a
  programming language which simultaneously acts as a verification
  environment based upon HoTT. This is likely to become a major
  step forward in programming languages design and influence
  the development of all current and future systems in this area.
\item {\bf Generating Impact:} Developing new and fundamentally
  better ways to construct formally verified software is not just an
  end in itself, but is also a key prerequisite for engaging others do
  do so.  To help ensure uptake of our research by the wider
  community, we will produce a number of case studies which will allow
  users to experiment with our results; at the same time, their
  practical experiences with it will feed back into our research.
\end{itemize}

{\bf Calibre and Ambition:} The foundational nature of HoTT, the
consequent potential for solving key open problems in programming
languages and formal verification research, and the resulting
potential applications to software correctness attest to the quality
and calibre of this project. Our ambition is demonstrated by the
breadth of our central belief that HoTT is not just a mathematical
foundation for type theory, but can also be turned into a programming
language and verification environment which - in its treatment of
quotients, representational invariance and equality - will become the
benchmark standard which current and future systems will seek to
emulate.


\section{Scientific and Technological Background:}

{\bf Programming Languages:} Abstraction is essential in programming
where identifying common structure is needed to ensure code is clear,
clean and concise. This has lead to the development of high level
programming languages with expressive type systems capable of closing
the {\em semantic gap} between what programmers know about
computational entities and what their types can express about them.
The current state of the art are the {\em dependently typed
  programming languages} such as Agda, Epigram, Idris and Coq where
the programmer can express within the type of his/her program a
continuum of precision from basic assertions up to a complete
specification - about a program’s behaviour. This proposal seeks to
become the first of a new breed of univalent dependently typed
programming languages which advance the state of the art by offering a
powerful, yet computationally tractable, equality and thereby bring to
reality the goal of programming up to invariance of representation.

%Coq award
%transform
%ahead of the curve

{\bf Programme Verification:} While the advantages of the certainty
afforded by mathematical proof has been recognised for centuries, it
has also been recognised that this certainty is undermined by the
capacity for humans to make mistakes in their proofs. The advent of
computers raised the possibility once more of achieving in practice
the promise of mathematical certainty. This potential is now coming to
fruition, eg systems such as Coq have been able to formally verify
both large mathematical theorems such as the 4-Colour problem, and
large software systems such as the CompCert C-compiler
However, these systems
are not {\em extensional} in that just because one can prove that
objects are behaviourally indistinguishable, one cannot conclude that
they are the same. This is a fundamental problem as it weakens the
power of the verification system. Our research will address this
issue by producing a formal verification system where objects with the
same behaviour can be proved equal thereby advancing the state of the
art here.


{\bf Type Theory:} Underlying both formal verification systems and
programming languages is the subject of type theory which grew out of
Russell's attempts to deal with paradoxes in naive set theory. A major
conceptual understanding afforded by type theory was the Curry-Howard
correspondence which observed that programs and proofs are actually
the same thing, eg proofs are just particular forms of programs. As a
result, by developing sophisticated type theories, we advance both the
field of programming languages and the field of program verification.

A major step forward within type theory was achieved by Martin-L\"of
who realised that type theory needed to be extended to cover equality
within the system and he did this by introducing the intensional
identity type in Martin-L\"of Type Theory (MLTT).
However, it soon became apparent that this notion of equality type was
too weak, eg functions that are pointwise equal cannot be proven to
be equal. To address this, Martin-L\"of then introduced extensional
MLTT which produced a strong equality but at the price of
losing decidability of type checking. However, the problem of a strong but
decidable theory of equality remained fundamentally unresolved for 40
years. 


{\bf Observational Type Theory} (OTT) is a step forward towards a
decidable type theory with a strong equality, proposed by
Altenkirch and McBride~\cite{alti:ott-conf} in 2007. Its propositional equality is designed
to fit the structure of types, and is hence extensional for functions,
but its definitional equality remains decidable. However, equality of
types themselves is rigidly structural, not up to isomorphism,
allowing a proof-irrelevant implementation but precluding the
exploitation of higher-dimensional structure. As we advance to a
proof-relevant treatment of a more refined equality, OTT offers key
insight and techniques to combine a strong equality with decidable
type checking.

%power of methods
%absolutely vital
%very clear evidence

% invariance of representation = iso == equal
%vs
% strength of propositional equality (= identity type)
% strength of definitional equality beta

% hott or univalence
% hott = tt + univalence + hits
% univalance + hdtt/hdct
% hott = int param + kan filllers + hits
% what does univalence mean

{\bf Homotopy Type Theory} (HoTT) is a revolutionary new analysis of
equality based on intuitions from homotopy theory. Types are
\emph{spaces}; terms are \emph{elements} within a space; the equality
type is the space of \emph{paths} in a space. The core of HoTT is the
new Axiom of Univalence, introduced by Fields medalist Vladimir
Voevodsky, asserting (roughly) that isomorphic types are
equal. Crucially, equality proofs carry the data necessary to program
and reason up to invariance of representation. HoTT subsumes function
extensionality but also transcends what we could previously express,
e.g.\ \emph{higher} inductive types (HITs) include the usual tree
structures, but also quotients \cite{alti:mpc04} and geometric objects such as the
torus, where the hole stops us deforming some paths into others. Such
paths contain non-trivial structural information, necessitating a
proof-relevant equality. The same phenomenon occurs in computer
science, e.g. lists are an inductive type; braids are lists with extra
paths identifying lists upto twisting of any element past its
neighbours, and bags are braids with paths-between-paths identifying
braidings which yield the same permutation.
% Finally, in combinatorics, HITs cover the
% fundamental concept of species as species are quotient
% structures. Despite the ubiquity of examples, there is however no
% systematic and complete treatment of HITs which harnesses their power
% and thereby deliver on their potential.

%txa: moved it to the end, after HoTTx
{\bf Logical Relations:} Coquand's model is closely reated to
parametricity because we can view the interpretation in cubical sets
as defining dependently typed logical relations.
%txa: replacing
% One of the key features of Coquand's model is that
% the equality relation is defined recursively over the structure of
% types using an internalised form of parametricity. 
Parametricity
itself is a fundamental technology within computer science and,
research into it is currently being funded by EPSRC at Strathclyde. We
will use the expertise at Strathclyde to see if variants of the
standard presentations of parametricity lead to better behaved
refinements of the cubical set model, and feed innovations in the use
of parametricity within HoTT back into the general parametricity
community. {\bf Needs Repair/rewrite}


%What is HoTT

%However, we do not yet have a satisfactory computational explanation of
%univalence. This is partly because almost all of our understanding of
%HoTT exists within a classical framework 

%Why us and not the United States of Awodey. Strong definitional equlity. 

Coquand's \emph{cubical} model of HoTT interprets closed expressions
as canonical denotations, demonstrating basic feasibility of computing
with univalence. However, much work lies ahead: we need to give
closed expressions normal forms \emph{within} HoTT, and to recover
definitional equalities lost by the cubical interpretation. Further, other models might
offer better computational foundations for HoTT, so practical progress
towards programming languages and verification tools based on HoTT
is best informed by a broader model theory.



% Nevertheless, Coquand's and Huber's
%work shows that producing programming languages and proof assistants
%based upon HoTT is feasible in principle and we plan to
%collaborate closely with them.


%Earleir motivation for OTT
%Grant numbers below

%Mention somewhere the prizes that Coq has been recently been awarded, as proof that this is cutting-edge technology

%Mention somewhere that HoTT is having an impact on the design of (new versions of) Coq: there is ongoing work on implementing mechanisms for higher inductive types (Barras). Also: Bauer is working on new proof assistant based on ideas of Voevodsky.

%Mention somewhere also the importance of type theory, Coq, Agda for
%computer-assisted formalization of mathematical proofs 


\section{Methodology and Research Programme:}



Our general methodology to the development of HoTT-based software
construction and verification will be to harness ideas from a number
of different sources: i) the evolving state of the art, e.g. the HoTT
book and the cubical sets construction of Coquand et al. as well as
the ongoing dialogue with our collaborators; ii) our work on OTT which
is a proof-irrelevant prototype of what we seek; iii) our work on the
implementation of dependently typed programming languages (Epigram and
$\Pi\Sigma$ \cite{alti:pisigma-new,alti:checking}); iv) our work on datatypes (containers, indexed
containers, induction-induction and induction-recursion) 
\cite{alti:fossacs03,alti:tlca03,alti:icalp04,alti:jpartial,alti:mpc04,alti:cont-tcs,alti:regular,alti:cats07,alti:jcats07,alti:lics09,
alti:catind2}
 ; v) our work
on parametricity; and vi) our work on constructing internal models of
type theory. These multiple sources will ensure that we neither
slavishly follow the hype that inevitably surrounds significant
innovation, nor are unaware of current and future advances in HoTT. We
have divided the project into the following work packages with WP1
hosted in Leeds, WP2-4 in Nottingham and WP5-8 in Strathclyde. Of course
the reality is that we will continue our established practice of
working closely together.

%Argue in justification for resources
%Explain compute below 



{\bf WP1: Semantic Foundations of HoTT.}  We seek semantic insights
into HoTT akin to those provided by Cartesian closed categories for
the simply typed $\lambda$-calculus.  A constructive model theory for
HoTT is essential because: (1) a general model theory of HoTT will
guide the design of different presentations and implementations of
HoTT; (2) models of HoTT will provide algebraic techniques to reason
about the correctness of implementations which complement syntactic
techniques, and in particular (3) specific implementations of HoTT can
be proven sound by giving a specific models of them.  These models
need to be constructive so that programs, even those using the
Univalence axiom, will compute ({i.e.} reduce to a normal form). While
the standard model of HoTT based upon simplicial sets is not
constructive, the cubical set model recently developed by Coquand et
{al.}\ is constructive and raises the question of understanding it
within a general model theory.

 We will attack this problem from the following directions: (1) we
 will analyse existing models ({e.g.} groupoids, strict
 $\omega$-groupoids, simplicial sets, cubical sets) and isolate
 exactly how they ensure constructivity or where they fail to do
 so. In the latter case, we will investigate whether it is possible to
 constructivize them; (2) on the basis of our experience with (1), we
 will develop a model theory for HoTT by both adapting known methods
 to construct new models from old (such as by the methods of slicing,
 glueing, sheaves) to the setting of HoTT.

For (2), our starting point will be the simple axiomatisation of the
notion of a model of type theory obtained by Steve
Awodey~\cite{AwodeyS:natmtt}.  On that basis and building on Joyal's
work~\cite{JoyalA:cathl}, we will require the existence of additional
structure, corresponding to the axioms of HoTT under consideration. In
terms of risk, (1) is certainly achievable since it involves only the
analysis of existing concrete models, while (2) is a more ambitious
goal. Nevertheless, our expertise on semantics of type
theories~\cite{neil2014relParamDep}, homotopical
algebra~\cite{GambinoN:idetwfs,GambinoN:homl2c} and
$\omega$-groupoids~\cite{alti:csl12} makes even this ambitious goal
feasible. Deliverables from WP1 will be a broad class of models of
HoTT which considerably deepen our understanding and map out the
design space of its syntactic presentations.

% {\bf WP1: Semantic Foundations of HoTT:}  We
% seek semantic insights into HoTT akin to those provided by Cartesian
% closed categories for the simply typed
% $\lambda$-calculus.  A constructive model theory for HoTT is essential because: i)
% specific implementations of HoTT can be proven sound by giving a specific
% models of them; ii) 
% %since we don't know a priori what the best implementation
% %of HoTT will be, 
% a general model theory of HoTT will implicitly predict, and thereby
% guide, the design space of different presentations and implementations
% of HoTT; and iii) models of HoTT will provide algebraic techniques to
% reason about the correctness of implementations which complement
% syntactic techniques. These models need to be constructive so that
% programs, even those using Univalence, will compute. While the
% standard model of HoTT based upon simplicial sets is not constructive,
% Coquand et al.'s recent cubical set model is constructive and 
% has thus opened the door to a more general model theory.

% %Joyal
% %Back ground : Qullien model structures

% We will attack this problem from the following directions: i) we will
% analyse existing models (cubical sets, groupoids, strict
% $\omega$-groupoids, simplicial sets, globular sets) to isolate exactly
% how they ensure constructivity, or (where they fail to do so) how to 
% constructivize them; and ii) informed by i),
% we will develop a model theory for HoTT by both adapting known methods to
% define Quillen model structures (such as the small object argument)
% to the constructive setting and by showing how one can build  new
% constructive model structures from old (eg by
% slicing). A promising starting point for a constructive
% version of the small object argument comes from Garner's work, where
% it is related to the construction of free monads. In terms of risk,
% i) is certainly achievable since it involves only the analysis of
% existing concrete models, while ii) is a more ambitious
% goal. Nevertheless, our expertise on semantic models of parametricity \cite{neil2014relParamDep}, model
% categories (Gambino) and $\omega$-groupoids \cite{alti:csl12} makes even this
% ambitious goal feasible. Deliverables from WP1 will be a broad class
% of models of HoTT which considerably deepen our
% understanding and map out the design space of its 
% syntactic presentations.

% back ground work: NOMINAL SETS,

%LARGE BODY OF
%WORK ON SIMPLICIAL SETS AND MODEL CATEGORIES. MANY MODELS =>
%UNIVERSALLY VALID PRINCIPLES.
% Design space of the HoTT family earlier om

%Background: current presentation (HoTT presentation and Thierry's)
%dont have canonicity.

{\bf WP2: Univalent Type Theory:} Building on WP1, we need syntactic
presentations of our models of HoTT in the form of type theories. The
challenge is to present the essential data of the model as built from
a {\em finite} collection of type and term constructors - this is
particularly difficult given the {\em arbitrary} higher dimensional
structure of HoTT. One also must prove essential properties such as strong normalization, decidability of
definitional equality and canonicity (i.e. all terms reduce to
values). It is of particular interest to establish the expressive
power of the associated equational theories, e.g. to distinguish
carefully between which computation rules hold as definitional
equalities and which as propositional equalities. Another key property
(required for WP5) is that, as a foundational theory, our type theory
ought to be expressive enough to describe its own models. These
properties will be established either directly or via the models of
WP1.




%Different models will produce different theories and we
%will analyse them in terms of their tractablility, concision and 
%meta-theoretic properties. Essential properties we require of a well
%behaved type theory are

% We will begin by developing Altenkirch's preliminary type theory for
% cubical sets, which both internalises parametricity and adds Kan
% fillers to the theory. 
Cubical sets share structure with logical relations, as Altenkirch has
recently observed. We shall exploit this connection to replace the
uniform identity type of intensional MLTT with a
higher-dimensional equality defined to fit the structure of types. The
models from WP1 will drive refinement of our design, until we have a
canonical presentation of HoTT. Our preparatory work, and prior
expertise in OTT \cite{alti:ott-conf}, normalisation by
evaluation and big-step reduction
\cite{alti:ctcs95,alti:lics96,alti:flops04,txa:jtait}, 
strengthening definitional
equality~\cite{Allais:2013:NEN:2502409.2502411}, and logical
relations~\cite{neil2014relParamDep} ensures a high probability of delivering
an effective presentation of HoTT. Our more audacious goal is to find
the generic internal
language of $\infty$-LCCCs in the same way that extensional MLTT
is the internal language of LCCCs.  {\em Deliverable: a type theory to
  underpin our programming language and formal verification environment.} {\bf
introduce  $\infty$-LCCC} 

%We follow the
%practise in WP1 of managing risk in this workpackage by first aiming
%at the moderate goal of deriving specific presentations relating to
%specific models and then aiming for the more ambitious goal of
%integrating these presentations onto a unified framework.

{\bf WP3: Higher Inductive Types:} 
Our goal is to accomodate HITs in the semantics developed in WP1 and
the syntactic framework of WP2. Our idea to achieve this is to develop
a universal HIT playing the role for HITs that W-types play for
ordinary inductive types. This is feasible as partial progress has
already been made \dots one can often reduce HITs with higher
dimensional constructors to ones with only 0- and 1-dimensional
constructors (using the hubs-and-spoke construction). 
Similarly, \cite{gylterud:thesis,kock:groupoids} has shown how quotient 
containers can be reduced to ordinary containers in a homotopical 
setting. A secondary goal
is to generate a high-level level syntax for HITs as an alternative to
the universal HIT in the same way that strictly positive types provide
a grammar for defining various W-types. This will feed into WP7.  Our
most ambitious goal is to - time permitting - investigate other
variations of HITs: coinductive HITs, mixed inductive/coinductive HITS
\cite{txa:mpc2010g}, HITs and inductive-inductive and inductive-recursive
HITs which open the door towards a more concise representation of
dependently typed syntax in type theory~\cite{chapman2009type} by
introducing constructors and definitional equalities at the same
time. We will also investigate a pattern matching syntax for HITs this
is related again to WP7. {\bf More on QCs}


The risk within the first phase of this WP seems relatively low since we
have already a good background from our previous work on data
types~\cite{alti:cont-tcs,alti:lics09,txa:cie10,alti:catind2} including EPSRC grants on Containers and Induction
Recursion and because partial results already exist showing that
results are available and also guiding the way towards fresh
ones. While its not clear we can push the results as far as we imagine
with respect to higher inductive recursive types etc, these results
are not essential to the rest of the project.


{\bf WP4: Programming with Effects:} {\bf Correctness via Types} Most
programs interact with their environments, e.g., to read and/or write
to the memory and to detect and respond to errors such as attempts to
divide by zero or to open files that don’t exist. Such programs are
called effectful programs and are known to be inherently difficult to
reason about because, for example, the result of a program might
depend upon the evaluation order. One major advance was Eugenio
Moggi’s idea that effects can be modelled semantically by monads and
Wadler later showed that monads could internalised as syntactic sugar
to structure effectful programs themselves, eg via the {\tt
  do}-notation of Haskell. More recently, Plotkin and Power extended
this analysis using Lawvere Theories to show how (all most all)
computational monads arise from effect-generating operations and
equations which these operations satisfy.

Unfortunately, in general, we cannot represent equational theories in
current programming languages such as Haskell. Therefore one is often
forced to program not using the quotient algebra as desired but using
the free algebra and then check - externally to the program - that the
program respects the quotient structure. Of course HoTT, can help us
formally verify this. But HITs allow us to go further and give an
inductive presentation of such quotient structures opening the way to
program directly on the quotient algebra and assert the correctness of
the program via type checking. Not only is this a particularly
efficient form of formal verification, but the correctness of this
program can then be used to validate the preconditions of other
programs. Executing this research program means formalising both
Lawvere Theories in HoTT (using HITs to represent effectful
computations) and also the mathematical algebra of Lawvere theories
such as tensor products which can be used to combine them. In doing
this we will set ourselves the concrete goals of both simplifying and
extending i) McBride and Andjelkovic's work on Frank from free monads
to Lawvere Theories; ii) Brady's effects library for Idris; and iii)
Bauer and Pretnar's treatments of effects in his programming language
Eff.  We will do this both within the type theory of WP2 and reflect
progress into the programming language of WP7.  This work package will
therefore act as both validation for the theoretical research done by
RA1, and also generate impact by showing how that work can be used to
tackle a major programming languages problem. Basic results should be
low risk as the fundamental ideas of treating effects via algebraic
theories and treating algebraic theories via HITs are established in
principle. However, more advanced effects such as indexed-effects
which arise in dependently typed programming, or a full-scale integration
of our results any in the language of WP7 will be more challenging.


{\bf WP5 Formalisation of Meta-Theory of HoTT.}  As we intend to use
HoTT as a formal verification system, we need the highest possible
level of trust in its correctness. This means we need to check that
the model theory of HoTT, the associated type theory and their
relationship are correct. Given that HoTT is inherently a complex and
combinatorially intricate mathematical subject because of its higher
dimensional structure, a purely paper based verification would be
doubtful.  Thus, the only way to ensure the required high level of
trust in the correctness of our work is to formally verify that this
is the case.

%Voevodsky ... fed up with errors in papers.

We will begin with the relatively low risk task of formalising the
cubical sets model of HoTT, our nascent cubical type theory and their
relationship in Agda. However this approach will be unsatisfactory in
the long run because of the limitations of Agda and hence we will
switch to formal verification in our HoTT-based type theory and formal
verification system as they are developed. Our belief is that formal
verification in HoTT will actually be easier because the syntax of
Type Theory will be more efficiently formalised as a HIT which
represents not just the type and term constructors but also the
associated definitional equality. This is low risk as there is already
some formalisation of HoTT in Agda, and more generally, because we
have significant expertise in both techniques (such as induction
recursion and induction induction) required formalise type theory and
also in the verification of properties of the formalisation~\cite{}.
As the project progresses we will consider the more ambitious goals
and risky goal of formally verifying properties of the core programming
language developed in WP6. At the end of the work package we will have
deliverables consisting of formal verification of the key properties
of HoTT. Not only will this ensure the required level of trust in our
system, but this will have a significant impact upon the formal
verification community as it will be the first instance of formal
verification in a {\em HoTT-based} formal verification
environment.~\footnote{as opposed to formal verification in current
  systems such as Agda and Coq}. {\bf Agda + non-computational
  univalence is not HoTT!}
  




% However, there is no clear
% theory or justification of the current implementation which also has
% been proven to be unsound in several instances. Our goal is to develop
% a notion of pattern matching which is consistent with HoTT and
% formally verify this fact. 

{\bf WP6: Implementing a Core Programming Language:} {\bf HoTT in
  Agda:} In order showcase the potential for HoTT based programming
languages to the wider programming languages community, and learn from
their feedback, we need to build a prototypical implementation of a
type checker and interpreter based on the type theory designed in WP2.
This system forms a proof of concept implementation lacking most if
not all bells and whistles present in modern implementations of Type
Theory. That is, while we will implement the type theory of WP2, we
will not attempt to implement a high-level syntax for datatypes,
universe polymorphism, implicit arguments or pattern
matching. However, the language will include a universal HIT based on
the work in WP3.

We are planning to use Haskell as an implementation language because
there is considerable expertise at Nottingham and Strathclyde in using
Haskell to implement type checkers for dependently typed languages
including Epigram, $\Pi\Sigma$ and Agda \cite{alti:checking,easy,alti:pisigma-new}.
In addition, our prototypical implementation of OTT will be
particularly informative as it can be viewed as a precursor of HoTT as
it has an recursively defined notion of equality. 

Coquand's and Huber's implementation of the cubical set model shows
that our approach is feasible in principle and we plan to collaborate
with them. However, we should be able to address particular issues
such as the fact that certain definitional equalities don't hold in
the cubical set model using the technology we have developed in the
context of OTT \cite{alti:ott-conf} which is the subject of current
work at Strathclyde.

The main difficulty within this WP is that the implementation of
higher dimensional type theories is a new area --- nevertheless our
experience in the implementation of dependent type theories leads us
to believe this is still of low risk. Having said that, there is some
risk that the implementation takes much longer than expected: we will
manage this by keeping the scope of the language small. {\bf Play up OTT and Parametricity (Johann)}

{\bf WP7: A High Level Programming Language:}
The aim of this WP is to make the language developed in the previous
WPs usable in practice. This relies on a high-level syntax for
datatypes including higher inductive types and on integrating known
technology such as implicit arguments but also compilation and
interfaces to other languages. A central question which needs to be
answered is wether it is preferable to integrate our ideas with an
existing system such as Agda, Coq or Idris or wether it is preferable
to start from scratch. The advantages of the former approach is that
we connect with significant user communities, can learn from their
experience and avoid duplication of work. However, it is currently
not clear wether this is feasible since it would affect the very
core of these systems. Either way, we plan to collaborate closely with
the developers of these systems to maximise compatibility and impact.

The technical challenges we face are to restrict pattern matching so
that it is compatible with HoTT --- we plan to build on recent work
by Coeckx \cite{coeckx-without-k}. There are other issues
related to the termination checker which need to be adressed, 
see recent discussions on the Agda and Coq mailing lists.
%\cite{coq-agda-issue-w-termination}. 
We also want to integrate pattern
matching with HITs --- this is currently an open problem. {\bf The
deliverables with consist of language extensions and some programs}

This is the most ambitious of our work packages because, if successful,
we would have produced a new state of the art programming language for
software construction and verification. Given the current interest in
HoTT, it would immediately attract attention of significant
numbers. However, the volume of work required to develop a practical
language makes this also the most risky WP. Nevertheless, if all we
produce are proof-of-concept implementations of some high level
features, leaving significant amounts of the implementation of a
practical language to future work, then the project as a whole will
still be a massive success because both the foundational, practical
and engineering groundwork for a homotopical programming language will
have been done. {\bf Libraries for doing HoTT in Agda}


% In Agda pattern matching is one of the main devices to support
% efective program construction ofr depndent types. However, unlimited
% pattern matching is incompatible with HoTT. Currently, in Akgda there
% is an adhoc implementation of a check that pattern matching is
% restricted so that UIP is not derivable. However, there is no clear
% theory or justification of the current implementation which also has
% been proven to be unsound in several instances. Our goal is to develop
% a notion of pattern matching which is consistent with HoTT and
% formally verify this fact. 

% In HoTT there are two orthogonal hierarchies of types indexed by size
% (i.e. universe level) and dimension (i.e. truncation level or h-level). We often
% want to quantify over types with a certain size or a certain dimension
% and we also want to be able to implement construction parametric in
% both. On the other hand we want to minimize bureaucracy and in
% particular automatically infer subtyping relations bewteen different
% levels. We will explore this new area which is essential to make HoTT
% usable in practice.



{\bf WP8: Generating Impact Through Case Studies:} The ultimate goal
of the proposed research is to support software construction and
verification via HoTT. Our final work package comes full circle to our
original motivation: we will apply our programming language to a
number of real-world programming problems thereby demonstrating the
impact that HoTT can have. By abstracting recurring and effective
patterns that arise, we will also develop new methodology to
complement the new expressivity of programming in HoTT.

Firstly, HoTT's ability to ``work upto isomorphism'' opens the way to
program correctly using simple and straightforward reference
presentations of data structures and then replace these presentations
with more efficient equivalents at runtime. For instance, we shall
deliver treatments of \emph{numbers} (simple unary representations $\cong$
efficient binary representations), \emph{sequences} (cons-lists $\cong$
finger-trees) and \emph{matrices} (vector-of-vectors $\cong$ sparse
encodings). %More generally, this opens the way for HoTT technology to
%be applied to situations where methodologies such as {\em views} and
%{\em worker-wrapper transformations} thereby influencing their
%development too. 
A similar phenomenon occurs with data structures which store redundant
information to improve access time, e.g., databases with indices to
cut search, and records with cached values to avoid
recomputation. {\bf ornaments}. The technical challenge will be to
ensure that the efficiency savings are not dominated by the cost of
computing with isomorphisms at runtime. Our idea is to use fusion to
minimise the number of isomorphisms present at runtime, and to enable
the compiler to work with intermediate representations to give fine
grain control of the cost of the isomorphisms involved.  This will
ensure that we maximize the regions within which we use the efficient
representations, converting data only at the boundaries.  In effect,
we will have improved on \emph{data abstraction}, the state-of-the-art
tool for managing the craft of implementation, by supporting the
refinement of concrete computational models of data.

Secondly, HoTT's richer notion of equality ensures that different
representations of a value can be exploited for efficiency purposes
but cannot yield inconsistency. For example, whilst treelike
structures can be given a canonical form, data such as individual graphs, cycles
and multisets often have multiple representatives which should be
treated the same by operations. Today's technology presents the
dilemma of whether to expose the representation and risk inconsistent
treatment or to hide behind an abstraction barrier which offers a
fixed repertoire of consistent operations but inhibits us from
exploiting the representation to develop unforeseen operations
efficiently. At last HoTT offers us a precise deal: we can work with
representatives, but we must work up to equality. 

\section{Quality, Management, and Planning}

\vspace*{-0.1in}


{\bf Relevance to Beneficiaries:} This, perhaps more than many
projects, is an ambitious project which - by qualitatively advancing
the very foundations of programming languages - has the potential to
have a very significant impact on a large number of researchers. While
theoretical computer scientists, e.g. category theorists, type
theorists and logicians will be interested in the fundamental nature
of Homotopy Type Theory, the most significant long term impact will be
on those who program and those who verify programs as they will be
interested is the critical mass of example code we will develop and
verify as this will provide them with a gateway to HoTT. The potential
impact of this research can also be gauged by the adventure,
timeliness and novelty:

{\bf Calibre, Ambition, and Adventure:} The potential of HoTT to solve
one of the deepest problems in programming languages and verification
- namely the efficient computational treatment of equality - and the
quality of our ideas for brining this potential to fruition all attest
to the calibre of the proposed research. Our ambition is demonstrated
by our desire to transform HoTT from an idea into a new
benchmark in the development of programming languages. The proposed
research certainly is not incremental! The adventurous nature of the
proposed research is demonstrated by its scope, which ranges from
fundamental research (WP1-WP3), to programming languages and
verification (WP5-WP7), and to impact generation via case studies (WP4 and
WP 8). The proposed research is also ...

\noindent {\bf ...timely:} This is an extremely timely moment to
embark on the proposed research. Not only do we have our own results
to draw on, but there have also been significant recent advances in
directly related areas, such as Coquand's cubical model of HoTT.
Moreover, the US government has just funded a \$7.5m complementary
project applying HoTT to the foundations of mathematics, thereby
further increasing timeliness.

\noindent {\bf ...novel:} 
Our goal of turning cutting-edge developments in type theory straight
into state-of-the-art programming language and verification techniques
distinguishes our approach to HoTT from many others which typically
focus on one or the other. We take this as our goal because we believe
that severing the link between foundational understanding and
practical application diminishes both. This overall novelty is
complemented at a more detailed level by our background in logical
relations, containers, and OTT means we also bring technical novelty
to the study of HoTT. This is crucial ... while we benefit from being
part of a worldwide community working on HoTT, we also have great
distinctiveness in our cummulative competancies and expriences leading to 
significant novelties in our approach.
 
\vspace*{0.02in}

{\bf Nationally Important:} The software market is estimated at \$500
billion per year, and this figure is likely to grow 
significantly in real terms 
as software becomes ever more
ubiquitous. It is thus essential to the UK's national interest to have 
strong presence in this market. One crucial aspect of software is that
it is correct, i.e., does what's intended and does not go wrong.  Even
failures of everyday devices like iPods and mobile phones are
inconvenient,
%it is inconvenient when everyday devices like toasters and mobile phones 
%fail to work properly, 
% software crashing is inconvenient,
%might be tolerable, 
but software leaking voting records,
compromising %for %an aeroplane crashing definitely is not.
the global financial sector, or launching 
nuclear weapons without authorisation 
can lead to unprecedented and clearly unacceptable global uncertainties.

While testing of programs has dominated the last 50 years of software
development,
%of software engineering, 
the next 50 years are likely to see an increasing
demand for provably correct software. This is partly because testing
is by its very nature only a partial guarantee, and partly because
programming language technology is finally advancing to the stage
where it is feasible to formally verify critical programs.
%prove that programs 
%%are correct in that they
%do what they are intended to do and nothing else. 
%EPSRC considers 
Both programming languages and program verification are identified in
EPSRC's portfolio as areas of vital 
%national importance
importance for cybersecurity, and EPSRC
thus intends to grow them.
This proposal 
%focusses precisely on 
%%This proposal fits exactly into the area of 
uses ideas from mathematics  to enhance 
programming languages and  program
verification, so lies squarely in their intersection. % of these areas.
%but also helps secure important 
%contributes to this important 
%national interests.
The UK is a world leader
in these areas, but continued investment is required to maintain that
status in a rapidly changing world and a rapidly evolving field.

Within programming languages and program verification, we are aiming
high.  The current state of the art in programming language design is
limited by the lack of a clear understanding of how strong equality
ought to be within a programming language. Our research will provide a
step change in perogramming languages research where the current
ad-hoc treatments of equality will be replaced by one with a well
understood foundation. Thus we expect the results of our research to
become the cornerstone for the next generation of high level
programming languages cited by both
theoreticians and practitioners well into the future. If successful, 
%the success can
the proposed research can
be expected to have great impact on programming languages and program
verification over the next 10 to 50 years, and perhaps even beyond. 
%After all, good research is timeless!

\vspace*{0.02in}

{\bf Feasibile:} We are {\em very well-positioned} to conduct the
proposed research. Drs Gambino and Altenkirch both attended the IAS
Special Year on Univalent Foundations in Princeton and have published
influential papers on HoTT as well as coauthoring the HoTT book. Prof
Ghani is an expert on logical relations (he is currently PI on an
EPSRC grant on the subject) and Dr McBride is a world expert on
programming languages. All members of the team have significant
experience in the key areas of category theory, type theory and
programming languages which are the pillars upon which this project is
built. In a nutshell, the team is world leading in all areas of the
project.

\vspace*{0.02in}

{\bf Success criteria:} We have clearly demonstrated success criteria
for each workpakage in the form of a clearly delineated deliverable
and have argued why these deliverables are of great value individually
but also are essential to the overall goal of the project. As for the
overall project, its criteria for success will be threefold: i) new
syntactic and semantic foundations for HoTT; ii) programmiung language
and verification tools for computing with HoTT; and iii) a code base
of programs written in, and verified by, these tools thereby showing
applied researchers how HoTT can be used in their daily practice.

% The success of the foundational phase of the proposed
% research will be demonstrated by developing thccess criteria e syntax (WP2) and
% semantics (WP1) 
% of a type theory (programming languages?) based upon the principles of HoTT
% (validating univalence), extending this to cover HITs (WP3) and 
% verifying the required properties of the language.
% The success of the languages and verification phase will be
% demonstrated by the implementation of a programming language (WP 5, 6, 7)  whose
% correctness has been formally verified and within which - for the
% first time - HoTT programs can be run. Finally, the success of the impact phase
% will be demonstrated by developing proof-of-concept applications of
% our results to problems of interest to the wider programming languages
% and program verification community.

\vspace*{0.02in}

{\bf Management and Planning:} Apart from clearly planning the content
and methodology within individual work packages, their success
criteria, the inherent balance between risk and ambition within them,
and fall back positions to ensure the project will proceed even if
unexpected difficulties arise, we have also carefully planned how the
workpackages fit togther, who will lead them and who will be secondary
contributors. This is detailed in the associated Gannt chart.
%Work on WP1 precedes work on the othered 
%work packages because it develops the fundamental techniques to be
%extended, applied, and implemented. Work on WP2 precedes work on WP3
%since morphisms between Lawvere fibrations are needed to estalish
%universal properties of constructions on logical relations. WP4, WP5,
%and WP6 are independent, but WP1, WP2, and WP3 all feed into them. WP7
%will be integrated with the others to the greatest extent possible by
%starting work on it as soon as results from WP1 are available. 
Risk {\em between} work packages is minimised since

Overall a significant positive outcome is virtually assured as the
constructive nature of Coquand's model forms a proof of concept that
programming with HoTT is possible, and we have sanity checked the
ideas we will use within each workpackage. If any difficulty arises we
have the expertise of our collaborators to supplement our own aswell
as that of the highly active HoTT community. Indeed, the biggest risk
is the huge amount of work required in the {\em engineering} of a
programming language and verification environment - for this reason we
have separated the ambitious goal of a fully fledged system (WP7) from
the relatively risk free goal of a core system (WP6).

{\bf The RA and Their Training:} We will seek two RAs: one with
expertise in some of category theory, type theory, semantics of
programming languages; and another with epxerience in the
implementation of functional programming languages, software
development and formal proof within modern systems such as Agda and
Coq. This is feasible: we know of several highly-qualified researchers
in these areas and recent events such as the trimester in certified
proof in Paris and the special year on HoTT in Princeton has created a
signficant pool of young PhD students and postdocs with the required
background. Nevertheless we will advertise widely to recruit the best RAs
possible. %Weekly
FOP group, {\bf Team Nicola}, and MSP group meetings
%To integrate the RA into the project, we will hold weekly MSP group
%meetings to 
provide regular opportunities to report on research 
progress and generate new ideas, and will help integrate the RA into
the project and the highly conducive local environments.
We plan on reading groups for discussing research
papers 
related to the project to also help train the RAs. The RAs will have
the opportunity to write papers and grant proposals, lead research,
and help mentor PhD students.  At the end of the project the RAs will
possess highly-desirable knowledge and skills, and be well-positioned
to lead future research and/or development efforts. This is important
since there is more work to be done in the research and development of
next generation programming languages than the active workforce can
handle. Overall, the proposed project will have a high impact in terms
of training.

\vspace*{0.02in}

{\bf Collaboration:} In carrying out the proposed research we will
collaborate with internationally leading researchers so as to maximise
its potential for impact. See our {\em Pathways to
  Impact} statement for details.

%We are keen to see the foundations we develop
%deployed in real language implementations, so are fortunate to have
%Dr~Conor McBride, architect of Epigram, in our
%research group. Dr~McBride is an expert in the design and
%implementation of dependently typed languages, and is a valuable
%source of information about what users expect from real systems.  
%%We will also consult our project partners: Prof~Alex Simpson and
%%Drs~Robert Atkey, Nick Benton, Andrew Kennedy, and Carsten
%%Sch\"urmann.  This will increase not only the quality of the proposed
%%research, its dissemination into the wider community, too.
%The expert guidance of our project partners --- Prof~Alex Simpson and
%Drs~Robert Atkey, Nick Benton, Andrew Kennedy, and Carsten Sch\"urmann
%--- will increase the quality of the proposed research and
%ensure that it solves important problems in key application
%areas.

%\vspace{-0.15in}

%{%\small

\newpage

\begin{footnotesize}
\begin{twocolumn}
\bibliographystyle{plain}
%\bibliographystyle{abbrv}
%\bibliographystyle{plainnat}
\bibliography{proposal,alti}
\end{twocolumn}
\end{footnotesize}

% \begin{multicols}{2}
% \bibliographystyle{plain}
% \bibliography{proposal,alti}
% \end{multicols}{2}

\end{document}

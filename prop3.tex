\documentclass[a4paper,11pt]{article}

\usepackage[top=2cm, bottom=2cm, left=2cm, right=2cm]{geometry}

\usepackage{mathptmx}
\usepackage{epsf}           %\input{epsf}
\usepackage{amsfonts}
\usepackage{amstext}
\usepackage{amssymb}
\usepackage{url}
\usepackage[dvips]{graphics}
\usepackage[dvips,all]{xy}
\usepackage{multicol}

%\newlength{\extraplusheight}
%\newlength{\extrapluswidth}
%\setlength{\extraplusheight}{4.7cm}
%\setlength{\extrapluswidth}{4.7cm}
%\addtolength{\textwidth}{\extrapluswidth}
%\addtolength{\textheight}{\extraplusheight}
%\addtolength{\oddsidemargin}{-.5\extrapluswidth}
%\addtolength{\evensidemargin}{-.5\extrapluswidth}
%\addtolength{\topmargin}{-0.5\extraplusheight}
\setlength{\parindent}{0 pt}
\setlength{\parskip}{1ex}

\newcommand{\Int}[1]{[\![ #1 ]\!]}
\newcommand{\malign}[1]{\begin{array}[t]{@{}l@{\;}l@{}l@{}} #1 \end{array}}
\newcommand{\logrel}[2]{\Delta_{#1,#2}}
\newsavebox{\fminibox}
\newenvironment{fminipage}
 {\begin{lrbox}{\fminibox}\begin{minipage}{8cm}\vspace*{-2ex}}
 {\\[-2ex]\vspace*{-2ex}\end{minipage}\end{lrbox}\noindent\centerline{\fbox{\usebox{\fminibox}}}\vspace{0.5ex}}   

%\setlength{\parindent}{0.15in}
%\setlength{\parskip}{0.3ex}

% Discourage unnecessary hyphenation.
\sloppy\hyphenpenalty 4000

\newcommand{\ra}{\rightarrow}
\newcommand{\A}{\mathcal{A}}
\newcommand{\E}{\mathcal{E}}
\newcommand{\C}{\mathcal{C}}
\newcommand{\B}{\mathcal{B}}
\newcommand{\Set}{\mbox{{\sf Set}}}
\newcommand{\Nat}{\mathit{Nat}}
\newcommand{\Alge}[1]{\mathit{Alg}_{#1}}
\newcommand{\hash}{\#}

\begin{document}

\thispagestyle{plain}
\begin{center}
  {\Large {\bf Univalent Type Theory, Programming and Verification:}}\\[1ex] 

\vspace*{-0.1in}

%  {\Large \bf Case for Support}\\[1ex]
  \rule{140mm}{.5mm}\\[2ex]
\end{center}

\noindent
{\bf \Large Part 1A: Previous Research \& Track Record - 
Thorsten Altenkirch}

\vspace{0.05in}

%\vspace{0.05in}

\noindent
See also \url{http://www.cs.nott.ac.uk/~txa/}

\vspace{0.05in}
Thorsten Altenkirch received his PhD from the University of
Edinburgh in 1993 since October 2006 he has been a Reader in
Computer Science at the University of Nottingham and in October
2008 he founded  the Functional Programming Laboratory together with
Graham Hutton.

Altenkirch is well known for his work on Type Theory and applications
of category theory in computer science, and has published over 50
research papers (all available online via google scholar) which are
frequently cited (h-index $\geq$ 26). During his work at Nottingham
\pounds 1M in research funding, comprising \pounds 650,903 as PI in 4
EPSRC grants, \pounds 241,075 as CoI in 2 EPSRC grants, \pounds
159,038 in 1 fellowship and 1 studentship. Especially relevant for the
current project is Observational Equality For Dependently Typed Programming
(EP/C512022/1), Theory And Applications of Induction Recursion (EP/G03298X/1),
Reusability and Dependent Types (EP/G034109/1). Altenkirch and Ghani have been  collaborating on
three research grants.

Altenkirch is one of the leading researchers on the emerging subject of Homotopy Type Theory,
this is witnessed by his fellowship at the Secidal Year on Homotopy
Type Theory at the Institute of Advanced Study in Spring 2013, where he contributed to the standard reference on the subject \cite{hott}. 
He has 
having given invited lectures on the subject (in 2012 at HDACT in
Lubljana, 2013 at the Curien-fest in Venice, in 2014 at MSC in Lyon
and at the IHP in Paris and published a number o papers related to the subject 
\cite{alti:lics99,alti:ott-conf,alti:csl12,alti:tlca13-hedberg}.

\noindent
{\bf \Large The Host Institution: University of Nottingham}

\vspace*{0.05in}
The School of Computer Science at the University of Nottingham
is a research-led School in one of the leading Universities in
the UK. The School was ranked 8th in the last Research Assessment
Exercise, and the Functional Programming Lab within the School is
one of four major research groups, with an international reputation
for its work on formally-based approaches to software construction
and verification.  The FP lab currently comprises 4 academic staff
(Thorsten Altenkirch, Venanzio Capretta, Graham Hutton, and Henrik
Nilsson) and 9 PhD students.  To date the
group has received \pounds 1.5M of EPSRC funding over 14 projects,
and has 12 completed PhD students.

The Functional Programming Lab provides a highly stimulating
research environment for researchers and PhD students with weekly
research meetings and frequent seminars. 
\noindent


\pagebreak

\noindent
{\bf \Large Part 1B: Previous Research \& Track Record -
Prof Neil Ghani}

\vspace{0.05in}

\noindent
See also \url{http://www.cis.strath.ac.uk/~ng}

\noindent
{\bf \Large The Host Institution} 

\vspace*{0.05in}

\noindent
The MSP group at the University of Strathclyde is an ideal venue for
conducting this research. Dr~Johann and Prof~Ghani have a
well-established, productive collaboration centred on using
mathematical structures to guide programming languages research, and
are internationally known for their work in $\lambda$-calculi,
category theory, functional programming, and logical
relations. Moreover,~Dr Conor McBride's membership in the MSP group
ensures local access to a programming perspective that
complements their more foundational one, and Epigram, a state-of-the-art
dependently-typed programming system of the sort intended to
incorporate the results of the proposed research. The MSP group also has
eight PhD students and two RAs.  The strength of the group has
been recognised by the University of Strathclyde, which is just now
hiring a new Lecturer for it.  Finally, central Scotland is home to
vibrant theoretical computer science and functional programming
research communities, and the University of Strathclyde is an active
participant in the Scottish Informatics and Computer Science Alliance
(SICSA). The MSP group also is active in other Scottish meetings, such
as ScotCats and SPLS.



\newpage

\noindent
{\bf \Large Part 2: The Proposed Research and Its Context}

\vspace*{-0.23in}

\begin{center}
\rule{170mm}{.5mm}
\end{center}

\vspace*{-0.3in}

\section{Introduction}\label{sec:intro}

\vspace*{-0.1in} 

%txa: added heartbleed.
{\bf Cost of Software Failure:} The cost of software failure is truly
staggering. Well known individual cases include the Mars Climate
Orbiter failure (\$125 million), Ariane Rocket disaster (\$500
million), Pentium Chip Division failure (\$475 million), most recently
the heartbleed bug (upto US\$400K per server) and there are
many, many more examples. Even worse, other software failures such as
one in the Patriot Missile System and the Therac-25 radiation system
have costs lives. More generally, programmers make 15 to 50 errors per
1,000 lines of code, and a 2008 study by the US government estimated
that faulty software costs the US economy \$59.5bn annually.  As a
result, the need to ensure programs run without error is becoming
increasingly critical to modern software development.

%The worldwide software market is estimated at
%\pounds 250 billion pounds every year and this figure will grow
%significantly in real terms as software becomes ever more ubiquitous
%in our lives and economy. Although the requirements for software vary
%enormously, a problem common to all software is to ensure programs run
%without error. Restarting a phone is a simple, if inconvenient task;
%restarting an aeroplane in mid-flight is not an option! Numerous
%examples of expensive software errors about from the Mars Rover to be
%recent Heartbleed Bug exposing a serious vulnerability in the popular
%OpenSSL cryptographic software library.  In a nutshell, the cost of software bugs is almost
%unbelievably staggering.

{\bf Type Systems:} 
\begin{quote}
  Static type systems are the world's most widely applied formal
  method, in daily use by millions of programmers.\\
  \emph{Simon Peyton Jones (Principal Researcher at
    Microsoft)}
\end{quote}
Type based approaches to software development support a range of
guarantees from basic data consistency to certified compliance with a
formal specification. Research in the UK and Europe has produced 
prototype languages and tools such as Agda, Epigram, Idris and Coq
and in the US NUPRL, Twelf,  and the Trellys project. These systems are
beginning to make their mark in safety critical application. Their
advanced type technology is raided by more mainstream languages 
such as Haskell, OCaml, Scala and C\# with significant industrial
employment. 

% Traditional methods of software verification based
% upon testing only generate partial guarantees of correctness. Stronger
% guarantees of software correctness are given by mathematical proofs
% but the complexity of modern software means that hand written
% mathematical proofs are untrustworthy. As a result, the only way to
% ensure truly secure and reliable software is the gold standard of
% formally verified software which offers machine-checked mathematical
% proofs of software correctness.

%  Several decades of pioneering work in
% the UK and elsewhere have culminated in the development of
% sophisticated and expressive type systems within which software
% construction and formal verification can take place. State of the art
% examples which are beginning to have industrial deployment are
% Haskell, OCaml, Agda, Idris, Coq, and the NSF funded Trellys project. 
% %txa: changed next sentence.
One of the most pressing and fundamental
open problems which all current systems face is to support abstraction
to achieve modular replaceability and hence reusability
but at the same time preserve the computational nature of the language.

 % which equalities
% should we compute with. If the strength of equality is too weak, then
% we cannot compute effectively while if it is too strong then type
% checking becomes undecidable.

%The lack of an answer to this
%question underpins the difficulties all current systems have with
%extensionality, quotients, abstract data types and our ability to
%compute with different representations of the same structure.

{\bf The Project:} {\em Homotopy Type Theory} (HoTT) is a
revolutionary new approach to this problem which, although  widely regarded as
a fundamental innovation with great potential, has yet to 
significantly impact on programming languages and verification.
% txa: added below
The key concept in HoTT is to introduce a very liberal equality which
identifies structures that are replacable. 
%Its core is Voevodsky's {\em Univalence Axiom} asserting
%that all provable equalities can be internalised as paths or, more
%conceptually, that computation is invariant under change of
%representation of the entities we compute with. 
% txa: deleted we believe that
% txa: changed par below.
Delivering this potential requires i) further
development of the foundations of HoTT; ii)
a programming language and verification environment based on
these foundations; and iii) applications demonstrating the effectiveness
of this environment to the broader software development community.
% programming languages and formal
% verification community. 
Therefore we intend to pursue a three-pronged
research programme based upon a synthesis of theoretical, applied and
impact-focussed research.

\begin{itemize}
\item {\bf Theoretical Foundations:} Almost
  all of our understanding of HoTT exists within a classical
  framework and hence cannot be used for programming
  languages and verification. Nevertheless, the recent work by Coquand et al. suggests a
  constructive presentation of HoTT might be found. We will develop
  both specific constructive models and a general constructive model
  theory of HoTT, and then complement these models with type theoretic
  presentations of them. 
\item {\bf Programming Languages and Verification:} The key
  deliverable of the second strand of our research will be a
  programming language which simultaneously acts as a verification
  environment based upon HoTT. This is likely to become a major
  step forward in programming languages design and influence
  the development of all current and future systems in this area.
\item {\bf Generating Impact:} Developing new and fundamentally
  better ways to construct formally verified software is not just an
  end in itself, but is also a key prerequisite for engaging others do
  do so.  To help ensure uptake of our research by the wider
  community, we will produce a number of case studies which will allow
  users to experiment with our results; at the same time, their
  practical experiences with it will feed back into our research.
\end{itemize}

{\bf Calibre and Ambition:} The foundational nature of HoTT, the
consequent potential for solving one of the key open problems in
programming languages and formal verification research, and the
potential applications to software correctness attest to the quality
and calibre of this project. Our ambition is demonstrated by our
central belief that HoTT can be turned into a programming language and
verification environment which - in its treatment of equality and
representational invariance - will become the benchmark standard which
current and future systems will seek to emulate.

\section{Scientific and Technological Background:}

{\bf Programming Languages:} Abstraction is essential in programming
where identifying common structure is needed to ensure code is clear,
clean and concise. This has lead to the development of high level
programming languages with expressive type systems capable of closing
the {\em semantic gap} between what programmers know about
computational entities and what their types can express about them.
The current state of the art are the {\em dependently typed
  programming languages} such as Agda, Epigram, Idris and Coq where
the programmer can express within the type of his/her program a
continuum of precision from basic assertions up to a complete
specification - about a program’s behaviour. This proposal seeks to
become the first of a new breed of univalent dependently typed
programming languages which advance the state of the art by offering a
powerful, yet computationally tractable, equality and thereby bring to
reality the goal of programming up to invariance of representation.

%Coq award
%transform
%ahead of the curve

{\bf Programme Verification:} While the advantages of the certainty
afforded by mathematical proof has been recognised for centuries, it
has also been recognised that this certainty is undermined by the
capacity for humans to make mistakes in their proofs. The advent of
computers raised the possibility once more of achieving in practice
the promise of mathematical certainty. This potential is now coming to
fruition, eg systems such as Coq have been able to formally verify
both large mathematical theorems such as the 4-Colour problem, and
large software systems such as the CompCert C-compiler
However, these systems
are not {\em extensional} in that just because one can prove that
objects are behaviourally indistinguishable, one cannot conclude that
they are the same. This is a fundamental problem as it weakens the
power of the verification system. Our research will address this
issue by producing a formal verification system where objects with the
same behaviour can be proved equal thereby advancing the state of the
art here.


{\bf Type Theory:} Underlying both formal verification systems and
programming languages is the subject of type theory which grew out of
Russell's attempts to deal with paradoxes in naive set theory. A major
conceptual understanding afforded by type theory was the Curry-Howard
correspondence which observed that programs and proofs are actually
the same thing, eg proofs are just particular forms of programs. As a
result, by developing sophisticated type theories, we advance both the
field of programming languages and the field of program verification.

A major step forward within type theory was achieved by Martin L\"of
who realised that type theory needed to be extended to cover equality
within the system and he did this by introducing the intensional identity type.
However, it soon became apparent that this notion of equality type was
too weak, eg functions that are pointwise equal cannot be proven to
be equal. To address this, Martin-L\"of then introduced extensional
Martin-L\" which produced a strong equality but at the price of
loosing decidability of type checking. However, the problem of a strong but
decidable theory of equality remained fundamentally unresolved for 40
years. 
%power of methods
%absolutely vital
%very clear evidence

% invariance of representation = iso == equal
%vs
% strength of propositional equality (= identity type)
% strength of definitional equality beta

% hott or univalence
% hott = tt + univalence + hits
% univalance + hdtt/hdct
% hott = int param + kan filllers + hits
% what does univalence mean

{\bf HoTT:} Homotopy Type Theory (HoTT) is a revolutionary new
approach to the problem which seeks to use intuitions from homotopy
theory to generate a new analysis of equality. In particular, we think
of types as spaces, terms as elements within the space and the
identity type as the space of paths in a space. At the core of HoTT,
is the new Axiom of Univalence as introduced by Fields medalist
Vladimir Voevodsky which – roughly – asserts that isomorphic types can
be treated as equal. In particular, we can program and reason up to
invariance of representation. Furthermore, univalence implies
functions which behave the same on every input are in fact equal - a
key property required in formal reasoning. HoTT also extends what we
could previously do, eg the higher inductive types (HITs) include not
only the usual inductive types, but also quotients and geometric
objects such as spheres or circles.

%What is HoTT

However, we do not yet have a satisfactory computational explanation of
univalence. This is partly because almost all of our understanding of
HoTT exists within a classical framework although this has changed
recently with a significant advance in the form of Coquand's cubical
model of HoTT together with an interpretation of Martin L\"of type
theory which validates univalence. However, much remains to be done:
there is not yet a computational interpretation of univalence in that
terms currently only compute to denotations in the model rather than
other terms. More generally, its not clear that this model is the
right model in that it only validates a weak version of the
computation rule for the identity type. Further, other models might
provide better foundations for developing HoTT, while a proper
development of the subject almost certainly requires a proper model
theory for HoTT. Finally, no programming language or verification tools
have been built for HoTT so far. Nevertheless, Coquand's and Huber's
work shows that producing programming languages and proof assistants
based upon HoTT is feasible in principle and we plan to
collaborate closely with them.

{\bf Parametricity:} One of the key features of Coquand's model is that
the equality relation is defined recursively over the structure of
types using an internalised form of parametricity. Parametricity
itself is a fundamental technology within computer science and,
research into it is currently being funded by EPSRC at Strathclyde. We
will use the expertise at Strathclyde to see if variants of the
standard presentations of parametricity lead to better behaved
refinements of the cubical set model, and feed innovations in the use
of parametricity within HoTT back into the general parametricity
community.



\section{Methodology and Research Programme:}
Our general methodology to the development of HoTT-based software
construction and verification will be to follow ideas from a number of
different sources
\begin{itemize}
\item The existing literature of HoTT es exemplified in the HoTT book
\item Our own experience in the definition and implementation of OTT as
  this can be seen as a proof-irrelevant version of what we seek
\item The work of Coquand et al. which represents the current state of
  the art in constructive developments of HoTT.
\item Our experience and those of others in parametricity which is
  fundamental in the cubical set model of HoTT.
\item The diverse experience, perspectives and motivations of our collaborators.
\end{itemize}
These multiple sources will ensure that we neither slavishly follow
the hype that inevitably surrounds significant innovation, nor are
unaware of current and future advances in HoTT. We have divided the
project into the following work packages with WP1-4 being hosted in
Nottingham and WP5-8 in Strathclyde. Of course the reality will be
that we all work very closely together.
 
{\bf WP1: Semantic Foundations of HoTT:} 
A constructive model theory for HoTT is essential because: i)
specific implementations of HoTT can be proven sound by giving a specific
model of it; ii) 
%since we don't know a priori what the best implementation
%of HoTT will be, 
a general model theory of HoTT will implicitly predict, and thereby
guide, the design space of different presentations and implementations
of HoTT --- compare for example the intimate relationship between the
simply typed lambda calculus and Cartesian closed categories; iii)
models of HoTT will provide algebraic techniques to reason about the
correctness of implementations which complement syntactic
techniques. These models need to be constructive to ensure that
programs, even those using the Univalence Axiom, can be given
computational behaviour. Unfortunately the standard model of HoTT
based upon simplicial sets is not constructive, and while Coquand et
al.\ have produced a specific constructive model of HoTT based upon
cubical sets, a general constructive model theory of HoTT does not yet
exist. {\bf Joyal, J-beta in Coquand. Curry-Howard}

We will attack this problem from the following directions: i) we will
analyse current models (groupoids, globular sets, simplicial sets,
cubical sets etc.) to isolate exactly where they fail to be
constructive {\bf small object:} or how they ensure constructivity,
and thereby develop techniques to constructivize specific models; and
ii) using intuitions from (i), we will define a notion of
$\infty$-LCCCs {\bf Shulman 2012} which provide a general model theory for HoTT 
by adapting known methods for
constructing Quillen model structures to a constructive setting.  
Principally this will involve extending the use of logical relations
(the core of the cubical set model) to other models.  In terms of
risk, i) is certainly achievable as it involves only the analysis of
already existing concrete models while ii) is a more ambitious
goal. Nevertheless, our expertise on semantic models of logical
relations (Ghani), {\bf what} (Gambino) and {\bf what} (Altenkirch)
makes even this ambitious goal feasible. Deliverables from WP1 will be
a broad class of models of HoTT which will not only considerably
deepen our understanding of HoTT, but which individually can be used
as the basis for a syntactic presentation of HoTT and which
collectively describe the design space of such presentations.  {\bf
  Nominals, large body of work on simplicial sets and model
  categories. Many models $=>$ universally valid principles}


{\bf WP2: Univalent Type Theory:} Next, we need a syntactic
presentation of the models of HoTT derived in WP1 in the form of a
type theory. This is difficult because one has to present all the data
in the model as built from a finite collection of type and term
constructors and then prove essential properties: strong normalisation
and decidability of definitional equality, canonicity in that all
terms reduce to values, and that the associated equational theory is
strong enough, e.g.\ that the computation rule for identity types is
valid definitionally in these theories. Another key property (required
by WP5) is that, as a foundational theory, our type theory ought to be
expressive enough to describe itself. These properties will be established
either directly or via the models of WP1.

%Different models will produce different theories and we
%will analyse them in terms of their tractablility, concision and 
%meta-theoretic properties. Essential properties we require of a well
%behaved type theory are

We will begin by developing Altenkirch's preliminary type theory for
cubical sets, which both internalises parametricity and adds Kan
fillers to the theory. This will produce a variant of intentional
Martin-L\"of Type Theory which replaces the global identity type with
a recursively defined higher-dimensional equality. We will then use
the various models from WP1 to either justify that this is a canonical
type theory for HoTT, or to refine the type theory further. {\bf Is
  WP1's influence clear?} Our preliminary work, and our expertise in
topics such as {\em Observational Type Theory} (Altenkirch,
McBride)~\footnote{ Observational Type Theory can be thought of as a
  proof-irrelevant version of the desired type theory.},
normalisation by evaluation and big-step reduction (Altenkirch,
McBride), the strengthening of definitional equality (McBride),
parametricity (Ghani) means that there is a high probability in
producing a type theory for HoTT. Our more ambitious, and hence more
risky goal, is to produce a type theory which acts as the internal
language of $\infty$-LCCCs in the same way that the extensional Martin
L\"of Type Theory arises the internal language of LCCCs.  This
work package's deliverable will be a type theory which will form the
basis of our programming language and formal verification
environment. {\bf Check TXA, Parametricity, AP-logic}.  {\bf Quillen
  Models, Quillen model Structures etc. Coherence}

%We follow the
%practise in WP1 of managing risk in this workpackage by first aiming
%at the moderate goal of deriving specific presentations relating to
%specific models and then aiming for the more ambitious goal of
%integrating these presentations onto a unified framework.

{\bf WP3: Higher Inductive Types:} In mathematics and computer science
structures are often presented via operations and equations leading to
a two-phase analysis: first one constructs a free algebra and then one
constructs the quotient algebra. In computer science, while reasoning
about free algebras is easy, reasoning about quotient structures is
difficult. Higher inductive types (HITs) offer a new an exciting
possibility ... one replaces equations by operations at higher
dimensions. This has already had exciting consequences in that
geometric objects such as spheres, the real numbers and the
cummulative hierarchy of sets can all be seen as HITs.  In Computer
Science, HITs cover datatypes with permutable positions such as
multisets, circular lists as well as the more general quotient
containers \cite{abottAltenGhaniMcB:quotientContainers, EPSRC
  Containers}. Finally, in combinatorics, HITs cover the fundamental
concept of species as species are quotient structures. Despite the
ubiquity of examples, there is however no systematic and complete
treatment of HITs which harnesses their power and thereby deliver on
their potential.

Our goal is to accomodate HITs in the semantics developed in WP1 and
the syntactic framework of WP2. Our idea to achieve this is to develop
a universal HIT playing the role for HITs that W-types play for
ordinary inductive types. This is feasible as partial progress has
already been made ... one can often reduce HITs with higher
dimensional constructors to ones with only 0- and 1-dimensional
constructors (using the hubs-and-spoke construction). 
Similarly, \cite{gylterud:thesis,kock:groupoids} has shown how quotient 
containers can be reduced to ordinary containers in a homotopical 
setting. A secondary goal
is to generate a high-level level syntax for HITs as an alternative to
the universal HIT in the same way that strictly positive types provide
a grammar for defining various W-types. This will feed into WP7.  Our
most ambitious goal is to - time permitting - investigate other
variations of HITs: coinductive HITs, mixed inductive/coinductive HITS
\cite{txa-nisse}, HITs and inductive-inductive and inductive-recursive
HITs which open the door towards a more concise representation of
dependently typed syntax in type theory \cite{chapman2009type} by
introducing constructors and definitional equalities at the same
time. We will also investigate a pattern matching syntax for HITs this
is related again to WP7. {\bf More on QCs}


The risk within the first phase of this WP seems relatively low since we
have already a good background from our previous work on data
types~\cite{II etc} including EPSRC grants on Containers and Induction
Recursion and because partial results already exist showing that
results are available and also guiding the way towards fresh
ones. While its not clear we can push the results as far as we imagine
with respect to higher inductive recursive types etc, these results
are not essential to the rest of the project.


{\bf WP4: Programming with Effects:} {\bf Correctness via Types} Most
programs interact with their environments, e.g., to read and/or write
to the memory and to detect and respond to errors such as attempts to
divide by zero or to open files that don’t exist. Such programs are
called effectful programs and are known to be inherently difficult to
reason about because, for example, the result of a program might
depend upon the evaluation order. One major advance was Eugenio
Moggi’s idea that effects can be modelled semantically by monads and
Wadler later showed that monads could internalised as syntactic sugar
to structure effectful programs themselves, eg via the {\tt
  do}-notation of Haskell. More recently, Plotkin and Power extended
this analysis using Lawvere Theories to show how (all most all)
computational monads arise from effect-generating operations and
equations which these operations satisfy.

Unfortunately, in general, we cannot represent equational theories in
current programming languages such as Haskell. Therefore one is often
forced to program not using the quotient algebra as desired but using
the free algebra and then check - externally to the program - that the
program respects the quotient structure. Of course HoTT, can help us
formally verify this. But HITs allow us to go further and give an
inductive presentation of such quotient structures opening the way to
program directly on the quotient algebra and assert the correctness of
the program via type checking. Not only is this a particularly
efficient form of formal verification, but the correctness of this
program can then be used to validate the preconditions of other
programs. Executing this research program means formalising both
Lawvere Theories in HoTT (using HITs to represent effectful
computations) and also the mathematical algebra of Lawvere theories
such as tensor products which can be used to combine them. In doing
this we will set ourselves the concrete goals of both simplifying and
extending i) McBride and Andjelkovic's work on Frank from free monads
to Lawvere Theories; ii) Brady's effects library for Idris; and iii)
Bauer and Pretnar's treatments of effects in his programming language
Eff.  We will do this both within the type theory of WP2 and reflect
progress into the programming language of WP7.  This work package will
therefore act as both validation for the theoretical research done by
RA1, and also generate impact by showing how that work can be used to
tackle a major programming languages problem. Basic results should be
low risk as the fundamental ideas of treating effects via algebraic
theories and treating algebraic theories via HITs are established in
principle. However, more advanced effects such as indexed-effects
which arise in dependently typed programming, or a full-scale integration
of our results any in the language of WP7 will be more challenging.


{\bf WP5 Formalisation of Meta-Theory of HoTT.}  As we intend to use
HoTT as a formal verification system, we need the highest possible
level of trust in its correctness. This means we need to check that
the model theory of HoTT, the associated type theory and their
relationship are correct. Given that HoTT is inherently a complex and
combinatorially intricate mathematical subject because of its higher
dimensional structure, a purely paper based verification would be
doubtful.  Thus, the only way to ensure the required high level of
trust in the correctness of our work is to formally verify that this
is the case.

%Voevodsky ... fed up with errors in papers.

We will begin with the relatively low risk task of formalising the
cubical sets model of HoTT, our nascent cubical type theory and their
relationship in Agda. However this approach will be unsatisfactory in
the long run because of the limitations of Agda and hence we will
switch to formal verification in our HoTT-based type theory and formal
verification system as they are developed. Our belief is that formal
verification in HoTT will actually be easier because the syntax of
Type Theory will be more efficiently formalised as a HIT which
represents not just the type and term constructors but also the
associated definitional equality. This is low risk as there is already
some formalisation of HoTT in Agda, and more generally, because we
have significant expertise in both techniques (such as induction
recursion and induction induction) required formalise type theory and
also in the verification of properties of the formalisation~\cite{}.
As the project progresses we will consider the more ambitious goals
and risky goal of formally verifying properties of the core programming
language developed in WP6. At the end of the work package we will have
deliverables consisting of formal verification of the key properties
of HoTT. Not only will this ensure the required level of trust in our
system, but this will have a significant impact upon the formal
verification community as it will be the first instance of formal
verification in a {\em HoTT-based} formal verification
environment.~\footnote{as opposed to formal verification in current
  systems such as Agda and Coq}. {\bf Agda + non-computational
  univalence is not HoTT!}
  




% However, there is no clear
% theory or justification of the current implementation which also has
% been proven to be unsound in several instances. Our goal is to develop
% a notion of pattern matching which is consistent with HoTT and
% formally verify this fact. 

{\bf WP6: Implementing a Core Programming Language:} {\bf HoTT in
  Agda:} In order showcase the potential for HoTT based programming
languages to the wider programming languages community, and learn from
their feedback, we need to build a prototypical implementation of a
type checker and interpreter based on the type theory designed in WP2.
This system forms a proof of concept implementation lacking most if
not all bells and whistles present in modern implementations of Type
Theory. That is, while we will implement the type theory of WP2, we
will not attempt to implement a high-level syntax for datatypes,
universe polymorphism, implicit arguments or pattern
matching. However, the language will include a universal HIT based on
the work in WP3.

We are planning to use Haskell as an implementation language because
there is considerable expertise at Nottingham and Strathclyde in using
Haskell to implement type checkers for dependently typed languages
including Epigram, $\Pi\Sigma$ and Agda \cite{epigram,easy,pisigma}.
In addition, our prototypical implementation of OTT will be
particularly informative as it can be viewed as a precursor of HoTT as
it has an recursively defined notion of equality. 

Coquand's and Huber's implementation of the cubical set model shows
that our approach is feasible in principle and we plan to collaborate
with them. However, we should be able to address particular issues
such as the fact that certain definitional equalities don't hold in
the cubical set model using the technology we have developed in the
context of OTT \cite{alti:ott-conf} which is the subject of current
work at Strathclyde.

The main difficulty within this WP is that the implementation of
higher dimensional type theories is a new area --- nevertheless our
experience in the implementation of dependent type theories leads us
to believe this is still of low risk. Having said that, there is some
risk that the implementation takes much longer than expected: we will
manage this by keeping the scope of the language small. {\bf Play up OTT and Parametricity (Johann)}

{\bf WP7: A High Level Programming Language:}
The aim of this WP is to make the language developed in the previous
WPs usable in practice. This relies on a high-level syntax for
datatypes including higher inductive types and on integrating known
technology such as implicit arguments but also compilation and
interfaces to other languages. A central question which needs to be
answered is wether it is preferable to integrate our ideas with an
existing system such as Agda, Coq or Idris or wether it is preferable
to start from scratch. The advantages of the former approach is that
we connect with significant user communities, can learn from their
experience and avoid duplication of work. However, it is currently
not clear wether this is feasible since it would affect the very
core of these systems. Either way, we plan to collaborate closely with
the developers of these systems to maximise compatibility and impact.

The technical challenges we face are to restrict pattern matching so
that it is compatible with HoTT --- we plan to build on recent work
by Coeckx \cite{coeckx-without-k}. There are other issues
related to the termination checker which need to be adressed
\cite{coq-agda-issue-w-termination}. We also want to integrate pattern
matching with HITs --- this is currently an open problem. {\bf The
deliverables with consist of language extensions and some programs}

This is the most ambitious of our work packages because, if successful,
we would have produced a new state of the art programming language for
software construction and verification. Given the current interest in
HoTT, it would immediately attract attention of significant
numbers. However, the volume of work required to develop a practical
language makes this also the most risky WP. Nevertheless, if all we
produce are proof-of-concept implementations of some high level
features, leaving significant amounts of the implementation of a
practical language to future work, then the project as a whole will
still be a massive success because both the foundational, practical
and engineering groundwork for a homotopical programming language will
have been done. {\bf Libraries for doing HoTT in Agda}


% In Agda pattern matching is one of the main devices to support
% efective program construction ofr depndent types. However, unlimited
% pattern matching is incompatible with HoTT. Currently, in Akgda there
% is an adhoc implementation of a check that pattern matching is
% restricted so that UIP is not derivable. However, there is no clear
% theory or justification of the current implementation which also has
% been proven to be unsound in several instances. Our goal is to develop
% a notion of pattern matching which is consistent with HoTT and
% formally verify this fact. 

% In HoTT there are two orthogonal hierarchies of types indexed by size
% (i.e. universe level) and dimension (i.e. truncation level or h-level). We often
% want to quantify over types with a certain size or a certain dimension
% and we also want to be able to implement construction parametric in
% both. On the other hand we want to minimize bureaucracy and in
% particular automatically infer subtyping relations bewteen different
% levels. We will explore this new area which is essential to make HoTT
% usable in practice.



{\bf WP8: Generating Impact Through Case Studies:} The ultimate goal
of the proposed research is to support software construction and
verification via HoTT. Our final work package comes full circle to our
original motivation: we will apply our programming language to a
number of real-world programming problems thereby demonstrating the
impact that HoTT can have. By abstracting recurring and effective
patterns that arise, we will also develop new methodology to
complement the new expressivity of programming in HoTT.

Firstly, HoTT's ability to ``work upto isomorphism'' opens the way to
program correctly using simple and straightforward reference
presentations of data structures and then replace these presentations
with more efficient equivalents at runtime. For instance, we shall
deliver treatments of \emph{numbers} (simple unary representations $\cong$
efficient binary representations), \emph{sequences} (cons-lists $\cong$
finger-trees) and \emph{matrices} (vector-of-vectors $\cong$ sparse
encodings). %More generally, this opens the way for HoTT technology to
%be applied to situations where methodologies such as {\em views} and
%{\em worker-wrapper transformations} thereby influencing their
%development too. 
A similar phenomenon occurs with data structures which store redundant
information to improve access time, e.g., databases with indices to
cut search, and records with cached values to avoid
recomputation. {\bf ornaments}. The technical challenge will be to
ensure that the efficiency savings are not dominated by the cost of
computing with isomorphisms at runtime. Our idea is to use fusion to
minimise the number of isomorphisms present at runtime, and to enable
the compiler to work with intermediate representations to give fine
grain control of the cost of the isomorphisms involved.  This will
ensure that we maximize the regions within which we use the efficient
representations, converting data only at the boundaries.  In effect,
we will have improved on \emph{data abstraction}, the state-of-the-art
tool for managing the craft of implementation, by supporting the
refinement of concrete computational models of data.

Secondly, HoTT's richer notion of equality ensures that different
representations of a value can be exploited for efficiency purposes
but cannot yield inconsistency. For example, whilst treelike
structures can be given a canonical form, data such as individual graphs, cycles
and multisets often have multiple representatives which should be
treated the same by operations. Today's technology presents the
dilemma of whether to expose the representation and risk inconsistent
treatment or to hide behind an abstraction barrier which offers a
fixed repertoire of consistent operations but inhibits us from
exploiting the representation to develop unforeseen operations
efficiently. At last HoTT offers us a precise deal: we can work with
representatives, but we must work up to equality. 


\section{Quality, Management, and Planning}

\vspace*{-0.1in}


{\bf Relevance to Beneficiaries:} This, perhaps more than many
projects, is an ambitious project which has the potential to have
a significant impact on a large number of researchers. Theoretical
computer scientists, e.g. category theorists, type theorists and
logicians will be interested in the fundamental nature of Homotopy
Type Theory. Further, programmers will be interested because of the
critical mass of programming language examples that Homotopy Type
Theory can be applied to --- they will particularly appreciate both
the languages we develop and the code we will provide and verify. The
potential impact of this research can also be gauged by the adventure,
timeliness and novelty which we now discuss.

{\bf Calibre, Ambition, and Adventure:} The potential of HoTT to solve
one of the deepest problems in programming languages and verification,
and the quality of our ideas for brining this potential to fruition
all attest to the calibre of the proposed research. Our ambition is
demonstrated by our desire to transform HoTT
from an idea with potential to a benchmark in the development of
programming languages. The proposed
research certainly is not incremental! The adventurous nature of the
proposed research is demonstrated by its scope, which ranges from
fundamental research (WP1-WP3) to programming languages and verification
(WP5-WP7) to impact generation via case studies (WP4 and WP 8). The
proposed research is also

\noindent {\bf ...timely:} This is an extremely timely moment to
embark on the proposed research. Not only do we have our own results
to draw on, but there have also been significant recent advances in
directly related areas, such as Coquand's cubical model of HoTT.
Moreover, programming
languages have advanced to the stage where, for the first time, the
results of this research both the implemented on their own and also
feed into existing languages.

\noindent {\bf ...novel:} {\bf  Fill In:}
Our goal of
turning cutting-edge developments in type theory straight into state-of-the-art
programming language and  verification 
techniques distinguishes our approach to HoTT from many others which
focus on one or the other. We take this as our goal because
we believe that severing the link between foundational understanding
and practical application diminishes both.
 
\vspace*{0.02in}

{\bf National Importance:} The software market is estimated at \$500
billion per year, and this figure is likely to grow 
significantly in real terms 
as software becomes ever more
ubiquitous. It is thus essential to the UK's national interest to have a
strong presence in this market. One crucial aspect of software is that
it is correct, i.e., does what's intended and does not go wrong.  Even
failures of everyday devices like iPods and mobile phones are
inconvenient,
%it is inconvenient when everyday devices like toasters and mobile phones 
%fail to work properly, 
% software crashing is inconvenient,
%might be tolerable, 
but software leaking voting records,
compromising %for %an aeroplane crashing definitely is not.
the global financial sector, or launching 
nuclear weapons without authorisation 
can lead to unprecedented and clearly unacceptable global uncertainties.

While testing of programs has dominated the last 50 years of software
development,
%of software engineering, 
the next 50 years are likely to see an increasing
demand for provably correct software. This is partly because testing
is by its very nature only a partial guarantee, and partly because
programming language technology is finally advancing to the stage
where it is feasible to formally verify critical programs.
%prove that programs 
%%are correct in that they
%do what they are intended to do and nothing else. 
%EPSRC considers 
Both programming languages and program verification are identified in
EPSRC's portfolio as areas of vital 
%national importance
importance for cybersecurity, and EPSRC
thus intends to grow them.
This proposal 
%focusses precisely on 
%%This proposal fits exactly into the area of 
uses ideas from mathematics  to enhance 
programming languages and  program
verification, so lies squarely in their intersection. % of these areas.
%but also helps secure important 
%contributes to this important 
%national interests.
The UK is a world leader
in these areas, but continued investment is required to maintain that
status in a rapidly changing world and a rapidly evolving field.

Within programming languages and program verification, we are aiming
high.  The current state of the art in programming language design is
limited by the lack of a clear understanding of how strong equality
ought to be within a programming language. Our research will provide a
step change in perogramming languages research where the current
ad-hoc treatments of equality wiull be replaced by one with a well
understood foundation. Thus we expect the results of our research to
become the cornerstone for the next generation of high level
programming languages cited by both
theoreticians and practitioners well into the future. If successful, 
%the success can
the proposed research can
be expected to have great impact on programming languages and program
verification over the next 10 to 50 years, and perhaps even beyond. 
%After all, good research is timeless!

\vspace*{0.02in}

{\bf Feasibility:} We are {\em uniquely well-positioned} to conduct
the proposed research. Drs Gambino and Altenkirch both attended the
IAS Special Year on Univalent Foundations in Princeton and hace
published influential works on HoTT. Prof Ghani is an expert on
parametricity (he is currently PI on an EPSRC grant on parametricity)
and Dr McBride is a world expert on programming languages. All members
of the team have significant experience in the key areas of category
theory, type theory and programming languages which are the three
pillars upon which this project is built.

\vspace*{0.02in}

{\bf Success criteria:} The success of the foundational phase of the proposed
research will be demonstrated by developing the syntax (WP2) and
semantics (WP1) 
of a type theory (programming languages?) based upon the principles of HoTT
(validating univalence), extending this to cover HiTs (WP3) and 
verifying the required properties of the language.
The success of the languages and verification phase will be
demonstrated by the implementation of a programming language (WP 5, 6, 7)  whose
correctness has been formally verified and within which - for the
first time - HoTT programs can be run. Finally, the success of the impact phase
will be demonstrated by developing proof-of-concept applications of
our results to problems of interest to the wider programming languages
and program verification community.

\vspace*{0.02in}

{\bf Management and Planning:} {\bf The rest was for a previous
  grant. IGNORE ... NG to rewrite:} Work on WP1 precedes work on the other
work packages because it develops the fundamental techniques to be
extended, applied, and implemented. Work on WP2 precedes work on WP3
since morphisms between Lawvere fibrations are needed to establish
universal properties of constructions on logical relations. WP4, WP5,
and WP6 are independent, but WP1, WP2, and WP3 all feed into them. WP7
will be integrated with the others to the greatest extent possible by
starting work on it as soon as results from WP1 are available. Risk
{\em between} work packages is minimised since WP4, WP5, WP6, and WP7
can influence one another, but lack of progress on one will not
inhibit progress on others. WP1, WP2, and WP3
%Risk {\em between} the work packages is minimised by
%organising them so that each can influence the others, but lack of
%progress on one will not prohibit progress on others. The core phase
%is 
are relatively risk free: %it builds 
they build on our previous work, and we 
already have ideas how to proceed and fallback positions
should our first approaches fail. Risk {\em within} work packages
is minimised by 
%ensuring that 
applying 
%bringing 
the right expertise %is brought 
to %bear on 
each. %The core research in 
WP1, WP2, and WP3 will be %jointly
led by Dr~Johann and Prof~Ghani; Prof~Ghani will lead WP4; and
Dr~Johann will lead WP5 and WP6. The RA will focus on WP7
but %, depending upon their theoretical expertise, may 
may
%be integrated into 
contribute to other %aspects of the proposed research as well.
work packages as well if they are able.

\vspace*{0.02in}

{\bf The RA and Their Training:} We will seek two RAs: one with
expertise in some of category theory, type theory, semantics of
programming languages; and anopther with epxerience in the
implementation of functional programming languages and software
development and formal proof within modern systems such as Agda and
Coq. This is feasible: we know of several highly-qualified researchers
--- e.g., Fredrik Forsberg, Cl\'ement Fumex, Barbara Petit, and Noam
Zeilberger --- due to finish PhDs or postdoctoral positions within the
next year, and we will advertise to recruit the best RA
possible. %Weekly
FOP group and MSP group meetings
%To integrate the RA into the project, we will hold weekly MSP group
%meetings to 
provide opportunities to report on research 
progress and generate new ideas, and will help integrate the RA into
the project. Our
%A weekly 
reading group for discussing research
papers %from the current literature
related to the project will also help train the RA. The RA will have
the opportunity to write papers and grant proposals, lead research,
and help mentor PhD students.  At the end of the project the RA will
possess highly-desirable knowledge and skills, and be well-positioned
to lead future research and/or development efforts. This is important
since there is more work to be done in the research and development of
next generation programming languages than the active workforce can
handle. Overall, the proposed project will have a high impact in terms
of training.

\vspace*{0.02in}

{\bf Collaboration:} In carrying out the proposed research we will
collaborate with internationally leading researchers so as to maximise
its potential for impact. See our {\em Pathways to
  Impact} statement for details.

%We are keen to see the foundations we develop
%deployed in real language implementations, so are fortunate to have
%Dr~Conor McBride, architect of Epigram, in our
%research group. Dr~McBride is an expert in the design and
%implementation of dependently typed languages, and is a valuable
%source of information about what users expect from real systems.  
%%We will also consult our project partners: Prof~Alex Simpson and
%%Drs~Robert Atkey, Nick Benton, Andrew Kennedy, and Carsten
%%Sch\"urmann.  This will increase not only the quality of the proposed
%%research, its dissemination into the wider community, too.
%The expert guidance of our project partners --- Prof~Alex Simpson and
%Drs~Robert Atkey, Nick Benton, Andrew Kennedy, and Carsten Sch\"urmann
%--- will increase the quality of the proposed research and
%ensure that it solves important problems in key application
%areas.

\vspace{-0.15in}

{%\small

\bibliographystyle{plain}
\begin{thebibliography}{}

\vspace*{-0.25in}

\setlength{\parindent}{0in}
\setlength{\columnsep}{0.3in}
\setlength{\parskip}{-0.1ex}

\begin{multicols}{2}

%M. Bezen and T. Coquand, ???

%M. Bezem and T. Coquand and S. Huber, A cubical set model of type theory, Preprint available from Thierry Coquand's web page, March 25th 2014.

%A. Pitts, Book on nominal sets.



%\bibitem{acc93}
%M. Abadi, L. Cardelli, and P.-L. Curien. Formal Parametric
%Polymorphism. {\em Theoretical Computer Science} 121(1-2),
%pp. 9~--~58, 1993.

%\bibitem{ab08}
%A. Ahmed and M. Blume. Typed Closure Conversion Preserves
%Observational Equivalence. Proceedings, Inernational Conference on
%Functional Programming, pp. 157~--~168, 2008.

%\bibitem{adr09} 
%A. Ahmed, D. Dreyer, and A. Rossberg. State-dependent
%Representation Independence. Proceedings, Principles of Programming
%Languages, pp. 340~--~353, 2009.

\bibitem{ajk}
R. Atkey, P. Johann, A. Kennedy. Abstraction and Invariance for
Algebraically-Indexed Types. Submitted, 2012.

%\bibitem{atk09}
%R. Atkey. Syntax for Free: Representing Syntax with Binding Using
%Parametricity. Proceedings, Typed Lambda Calculi and Applications,
%pp. 35~--~49, 2009.

%\bibitem{atk12} R. Atkey. Relational Parametricity for Higher
%  Kinds. Accepted for Presentation, Computer Science Logic, 2012.

%\bibitem{agjj12}
%R. Atkey, N. Ghani, B. Jabcobs, and P.
%Johann. Fibrational Induction Meets Effects. Proceedings,
%Foundations of Software Science and Computation
%Structures, pp. 42~--~57, 2012.

\bibitem{bar91}
H. P. Barendregt. Introduction to Generalized Type Systems. {\em
  Journal of Functional Programming} 1(2), 1991.

\bibitem{ber11}
J.-P. Bernardy. A Theory of Parametric Polymorphism and an
Application. PhD thesis, Chalmers University of Technology, 2011.

%\bibitem{bjp12}
%J.-P. Bernardy, P. Jansson, and R. Paterson. Proofs for Free --
%Parametricity for Dependent Types. {\em Journal of Functional
%  Programming} 22(2), pp. 107~--~152, 2012.

\bibitem{bfss90}
E. S. Bainbridge, P. J. Freyd, A. Scedrov, P. J. Scott. Functorial
Polymorphism. {\em Theoretical Computer Science} 70(1), pp. 35-64,
1990. 

%\bibitem{bh09} 
%N. Benton and C.-K. Hur.  Biorthogonality, step-indexing and compiler
%correctness.  Proceedings, International Conference on Functional
%Programming, pp. 97~--~108, 2009.

\bibitem{bkhb06} N. Benton, A. Kennedy, M. Hofmann, 
  L. Beringer. Reading, Writing and Relations. APLAS'96,
%Asian Symposium on Programming Languages and Systems, 
pp. 114-130. %, 1996.

\bibitem{bm05}
L. Birkedal, R. M{\o}gelberg. Categorical Models for Abadi and
Plotkin's Logic for Parametricity. {\em Mathematical Structures in
  Computer Science} 15(4), pp. 709-772, 2005.

%\bibitem{bmp06}
%L. Birkedal, R.M{\o}gelberg, and R. Petersen. Linear Abadi and Plotkin
%Logic. {\em Logical Methods in Computer Science} 2(5), pp. 1~--~48, 2006.

%\bibitem{cnet08}
%Total economic cost of insecure software: \$180 billion a year in the
%U.S. 2008. Available at {\tt
%http://news.cnet.com/8301-13846$\_$3-9978812-62.}\\
%{\tt html}

%\bibitem{dnb10}
%D. Dreyer, G. Neis, L. Birkedal. The Impact of Higher-order State and
%Control Effects on Local Relational Reasoning. ICFP'10, pp. 143-156.

\bibitem{dr04}
B. Dunphy, U.~S. Reddy. Parametric Limits. LICS'04,
%Symposium on Logic in Computer Science, 
pp. 242-251. %, 2004.

%\bibitem{ep03}
%S. Eijiro and B. C. Pierce. Logical relation for encryption. {\em 
%Journal of Computer Security} 11(14) (2003), pp. 521~--~554.

\bibitem{fgj11}
C. Fumex, N. Ghani, P. Johann. Indexed Induction and Coinduction,
Fibrationally. CALCO'11,
%Conference on Algebra and Coalgebra on Computer Science, 
pp. 176-191. %, 2011.

%\bibitem{galorath11}
%Cost of Independent Software Verification \&
%Validation. 2011. Available at 
%{\tt http://www.galorath.com/wp/}\\
%{\tt cost-of-independent-software-verification-}\\
%{\tt validation-ivv.php}

\bibitem{gjf12}
N. Ghani, P. Johann, C. Fumex. Generic Fibrational Induction.
{\em Logical Methods in Computer Science} 8(2), 2012.

%\bibitem{hd11}
%C.-K. Hur and D. Dreyer. A Kripke Logical Relation Between ML and
%Assemply. Proceedings, Principles of Programming Languages,
%pp. 133~--~146, 2011.

\bibitem{hj98}
C.\ Hermida, B.\ Jacobs. Structural Induction and Coinduction in a
Fibrational Setting. {\em Information and Computation} 145 (2),
pp. 107-152, 1998. 

%\bibitem{hd11} C.-K. Hur and D. Dreyer.  A kripke logical relation
%  between ML and assembly.  Proceedings, Principles of Programming
%  Languages, pp. 133~--~146, 2011.

%\bibitem{nist02}
%The Economic Impacts of Inadequate Infrastructure for Software
%Testing. 2002. Prepared by RTI for NIST. Available at {\tt
%  http://www.nist.gov/director/prog-ofc/report02-3.pdf}

%\bibitem{jac93}
%B. Jacobs. Comprehension Categories and the Semantics of Type
%Dependency. {\em Theoretical Computer Science} 107, pp. 169~--~207, 1993.

\bibitem{jac99}
B. Jacobs. {\em Categorical Logic and Type Theory}. Elsevier, 1999.

%\bibitem{jg07}
%P. Johann and N. Ghani. Initial Algebra Semantics is Enough!
%Proceedings, Typed Lambda Calculi and Applications, pp. 207~--~222, 2007.

%\bibitem{jg08}
%P. Johann, N. Ghani. Foundations for Structured Programming with
%GADTs. POPL'08, %Principles of Programming Languages,
%pp. 297-308. %, 2008.

\bibitem{jsv10}
P. Johann, A. Simpson, J. Voigtl\"ander. A Generic Operational
Metatheory for Algebraic Effects. LICS'10,
%Logic in Computer Science, 
pp. 209-218. %, 2010.

%\bibitem{kat11}
%S. Katsumata. Relating Computational Effects by
%$\top\top$-Lifting. Proceedings, International Conference on Automata,
%Languages, and Programming, pp. 174~--~185, 2011.

%\bibitem{kd}
%N. Krishnaswami and D. Dreyer. A Relationally Parametric Model of the
%Calculus of Constructions. Submitted, 2012.

\bibitem{ken97}
A. Kennedy. Relational Parametricity and Units of
Measure. POPL'97, %Principles of Programming Languages,
pp. 442-455. %, 1997.

%\bibitem{mr92}
%Q. Ma and J. Reynolds. Types, abstraction, and parametric
%polymorphism, part 2. Proceedings, Mathematical Foundations of
%Programming Semantics, pp. 1~--~40, 1992.

\bibitem{ms07}
R. M{\o}gelberg, A. Simpson. Relational Parametricity for
Comptuational Effects. LICS'07, pp. 436-355.

%\bibitem{mv05}
%P. Melli\`es and J. Vouillon. Recursive Polymorphic Types and
%Parametricity in an Operational Framework. Proceedings, Logic in
%Computer Science, pp. 82~--~91, 2005.

%\bibitem{pit87}
%A. Pitts. Polymorphism is Set-theoretic, Constructively. Proceedings,
%Category Theory and Computer Science, pp. 12~--~39, 1987.

\bibitem{pit96}
A. Pitts. Relational Properties of Domains. {\em Information and
  Computation} 127, pp. 66-90, 1996.

\bibitem{pa93}
G. Plotkin, M. Abadi. A Logic for Parametric
Polymorphism. TLCA'93, %Typed Lambda Calculi and Applications,
pp. 361-375. %, 1993.

\bibitem{pp02} 
G. D. Plotkin, J. Power. Notions of Computation Determine
Monads. FOSSACS'02,
%Foundations of Software Science and Computation Structures, 
pp. 342-356. %, 2002.

%\bibitem{ps98}
%A. Pitts and I. Stark. Operational Reasoning for Functions with Local
%State. {\em Higher-order Operational Techniques and Semantics}, 1998.

\bibitem{rey83} J. C. Reynolds. Types, Abstraction and Parametric
  Polymorphism. {\em Information Processing} 83(1) (1983), pp. 513-523.

%\bibitem{rey84}
%J. Reynolds. Polymorphism is not Set-theoretic. Proceedings, Semantics
%of Data Types, pp. 145~--156, 1984.

%\bibitem{rp10}
%J.\ Reed and B. C. Pierce. Distance makes the types grow stronger: a
%  calculus for differential privacy. Proceedings, International
%  Conference on Functional Programming, pp. 157~--~168, 2010.

%\bibitem{ss01}
%A. Sabelfeld and D. Sands. A PER Model of Secure Information Flow in
%Sequential Programs. {\em Higher-Order and Symbolic Computation} 14
%(1), pp. 59~--~91, 2001.  

%\bibitem{ts04}
%S. Tse and S. Zdancewic. Translating Dependency into
%Parametricity. Proceedings, International Conference on Functional
%Programming, pp. 115~--~125, 2004.  

%\bibitem{vw10}
%D. Vytiniotis and S. Weirich. Parametricity, Type Equality and
%Higher-order Polymorphism. {\em Journal of Functional Programming}
%20(2), pp. 175~--~210, 2010.

%\bibitem{wp99}
%Mystery of Orbiter Crash Solved. Available at
%{\tt
%  http://www.tysknews.com/Depts/Metrication/mystery$\_$of$\_$orbiter$\_$crash$\_$solved.htm} 


%\bibitem{stat85} R. Statman. Logical relations and the typed
%  $\lambda$-calculus. {\em Information and Control} 65 (1985),
%  pp. 85–-97.

\end{multicols}
\end{thebibliography}

}

\end{document}

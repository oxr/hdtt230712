\documentclass[a4paper]{article}

\usepackage{enumerate}
\usepackage{multibib}
\usepackage[margin=2.5cm]{geometry}

% \oddsidemargin 0in
% \evensidemargin 0in
% \textwidth=6.0in
\itemsep=0in
\parsep=0in


\newcites{pap,oth}{List of All Papers,Other References}

\newcommand{\project}{Homotopy Type Theory : Programming and Verification}
\newcommand{\mltt}{MLTT}

\begin{document}
\begin{center}\begin{Large}\textsc{CV : Ond\v{r}ej
      Ryp\'a\v{c}ek}\end{Large}\\
  \begin{large}
    Research Statement
  \end{large}

\end{center}

\subsection*{Personal data}
\begin{tabbing}
  \hspace{4cm}\= \\
  Address:\>Flat 12 Building 46, Marlborough Rd, London SE18 6TA\\
  Nationality:\>Czech\\
  Degrees:\>Phd, 2009, Computer Science, University of Nottingham\\
  \>MSc, 2005, Theoretical Computer Science, Charles University, Prague
\end{tabbing}
\subsection*{Academic Employment}
\begin{tabbing}
  \hspace{4cm}\=\\
  Oct 2012-present\>Research Assistant\\
  University of Oxford,\>Unified Model of Compositional and
  Distributional Semantics (EP/I03808X/1)\\
  Computer Science\>
\end{tabbing}
% \begin{tabbing}
%   \hspace{4cm}\=\\
%   May 2012 - Sept 2012\>Research Associate\\
%   University of Sheffield,\>Higher order Refinement Techniques for Model
%   Driven Architecture\\
%   Computer Science\>(EP/G031711/1)
% \end{tabbing}
\begin{tabbing}
  \hspace{4cm}\=\\
  Jan 2010 - Sept 2012\>Research Associate\\
  King's College London,\>Higher order Refinement Techniques for Model
  Driven Architecture\\
  Computer Science\>(EP/G03012X/1, EP/G031711/1)
\end{tabbing}



\subsection*{Research Statement}
I am not only the ideal candidate for the theory RA (based in Nottingham) on the {\project} project, but I also believe that it is not possible to find a better candidate by open advertiseing. 
The depth a breadth of knowledge and experience required from a potential candidate spans from pure higher category theory, over Martin-L\"of Type Theory ({\mltt}) and Homotopy Type Theory (HoTT), both at a theoretical and practical level, to functional programming and indexed containers. I have experience supported by a track record of publications in precisely these areas. It is unlikely that anyone else at a simalar level of seniority would.

The following are the facts attesting to my claim. 


%evidence
\paragraph{I have a publication track record in the project's key research area (WP2).}
I have already published in HoTT \citepap{AltenkirchRypacek12, AltenkirchLiRypacek14}, and in particular the very area the project is planning to explore: computational content of HoTT and formalisation of Type Theory (TT) in TT. 
The papers seek a formalisation of weak $\omega$-groupoids within \mltt, and implement it in the theorem prover Agda. 
%
Weak $\omega$-groupoids are a higher dimensional analogs of groupoids, i.e. categories where all morphisms are isomorphisms, where all coherence laws are weakened from equational axioms to higher cells. They are fundamental in categorical semantics of HoTT.
%
In \citepap{AltenkirchRypacek12} a syntactical presentation of weak $\omega$-groupoids is devised which is suitable for formalisation in Agda. In \citepap{AltenkirchLiRypacek14}, a major simplification of the formulation is achieved by following Brunerie's reformulation of Grothendieck's $\omega$-groupoids, which adheres to the same syntactical approach but simplifies the formulation of coherence.

Both formulations of weak $\omega$-groupoids amount to a definiton of a Type Theory with Identity Types with special introduction rules for coherence. I have played a major role in formalisation of the definition devised in \citepap{AltenkirchRypacek12,AltenkirchLiRypacek14} in a theorem prover (Agda). My other experience with formalisation in theorem proves includes my postdoc position at King's College, where most of the work was carried out formally in Coq and later in Agda. 

This attests to my readiness to contribute to WP2, which aims at developing a univalent type theory where univalence has computational content, and its formalisation in a theorem prover. It also attest my ability to contribute to other work packages that deal with implementation, WPs 5 and 6. 


\paragraph{I have knowledge and experience in all key research areas of the theory RA.} 
The work plan for WPs 2-4 spans a broad range of topics. These are precisely the areas I'm experienced with. 

WP3 intends to develop a foundational theory of HITs and also indexed containers in the setting of HoTT. I have background and experience in this area.  During my postdoc at King's College I have applied indexed containers to formalise datastructres stored on the head. The work has been completely formalised in Agda \citepap{FosterRypacekStruth12}.  In \citepap{JaskelioffRypacek12} we investigate traversability of datatypes as distributivity over a finitary container.

WP4 indents to study monads and Lawvere theories in HoTT in order to study effects in the setting of HoTT.  My thesis \citepap{RypacekThesis} is on 2-categorical monads and Lawvere theories in the context of mathematics of program construction so I have knowledge and experience in the area. 

% In particular the connection of Gray's tensor product of 2-categories and distributive laws is investigated. 

It is unlikely to find another RA who has published in all the key areas of the project . 


\paragraph{I have already established a working relationship with the members of the research team.} I have coathored two papers with Dr Thorsten Altenkirch, the PI on the type theory research branch of the project. This shows we have already established a good working relationship, that we can work and publish together. I have known and worked with Prof. Neil Ghani and Dr Conor McBride since my PhD in Nottingham; I've met Dr Nicola Gambino at the Thematic Trimester of Formalised Proofs at IHP in Paris, 2014. I worked with the whole team during the preparation of the grant proposal. 

This attests to me already having a productive working relationship with all members of the research team. Any freely hired RA would have yet to establish that. 

\paragraph{I have been involved in the project preparation} from the beginning therefore I know and have helped devise the work plan, the goals, methods. 
\begin{quote}{\small{I'm not sure it's good to say this:}}
  \end{quote}
%  
I've been involved in all the previous versions (last 2 years). It attests to my comittement to the project. 




\paragraph{I am well connected with the HoTT community}
I have spent June 2014 at the thematic trimester of Semantics of Proofs and Certified Mathematics at IHP, Paris, attending workshops, talks and discussing HoTT.
%
I attended the HDACT workshop in Ljubljana, June 2012. 
%
I regularly give presentations in seminars at my host universities and
meetings related to HoTT to raise awareness of the topic in general
and disseminate my work, e.g. ScotCats \citeoth{scotcats}, The
Wessex Theory Seminar \citeoth{wessex}, or Domains \citeoth{domains}.

This demonstrates that I'm an active and well connected member of
the community which will serve the project goals on making HoTT and in
particular the programming language developed within widely used.

\paragraph{I have ideas how to do the work on WP2-4}
t.b.d.

% \paragraph{Industrial Experience}
% Before starting my PhD in Nottingham I have worked, part-time and
% full-time in minor a major software development companies witnessing
% first-hand the current engineering practice based on trial and error

% Since starting my PhD I'm committed to contributing to a next generation of software technology which is based on mathematical rigour rather than trial and error.

% This project seeks to the theoretical advances of HoTT all the way to real world case studies. I believe my industrial experience will be useful in this area too.


\phantom{
\citepap{RypacekMsc}
\citepap{RypacekThesis}
\citepap{RypacekBackhouseNilsson06}
\citepap{LammelRypacek08}
\citepap{JaskelioffRypacek12}
\citepap{FosterRypacekStruth12}
\citepap{AltenkirchRypacek12}
\citepap{AltenkirchLiRypacek14}
\citepap{RypacekSadrzadeh14}}

\newpage
%\bibliographystylepap{alpha}
%\bibliographypap{cv-hdtt2}
%%%
\renewcommand{\refname}{{\large List of All Papers}}
\begin{thebibliography}{RBN06}
\bibitem[ALR14]{AltenkirchLiRypacek14}
Thorsten Altenkirch, Nuo Li, and Ond\v{r}ej Ryp\'{a}\v{c}ek.
\newblock Some constructions on omega-groupoids.
\newblock In {\em Proceedings of the 2014 International Workshop on Logical
  Frameworks and Meta-languages: Theory and Practice}, LFMTP '14, pages
  4:1--4:8, New York, NY, USA, 2014. ACM.

\bibitem[AR12]{AltenkirchRypacek12}
Thorsten Altenkirch and Ond\v{r}ej Ryp\'a\v{c}ek.
\newblock A syntactical approach to weak omega-groupoids.
\newblock In Patrick C{\'e}gielski and Arnaud Durand, editors, {\em CSL},
  volume~16 of {\em LIPIcs}, pages 16--30. Schloss Dagstuhl - Leibniz-Zentrum
  fuer Informatik, 2012.

\bibitem[RS14]{RypacekSadrzadeh14}
Ond\v{r}ej Ryp\'a\v{c}ek and Mehrnoosh Sadrzadeh.
\newblock A low-level treatment of generalized quantifiers in categorical
  compositional distributional semantics.
\newblock In {\em Second Workshop on Natural Language and Computer Science},
  Vienna, July 2014.

\bibitem[FRS12]{FosterRypacekStruth12}
Simon Foster, Ond\v{r}ej Ryp\'a\v{c}ek, and Georg Struth.
\newblock Correctness of object oriented models by extended type inference.
\newblock In Abhik Roychoudhury and Meenakshi D'Souza, editors, {\em ICTAC},
  volume 7521 of {\em Lecture Notes in Computer Science}, pages 46--60.
  Springer, 2012.

\bibitem[JR12]{JaskelioffRypacek12}
Mauro Jaskelioff and Ond\v{r}ej Ryp\'a\v{c}ek.
\newblock An investigation of the laws of traversals.
\newblock In James Chapman and Paul~Blain Levy, editors, {\em MSFP}, volume~76
  of {\em EPTCS}, pages 40--49, 2012.

\bibitem[Ryp10]{RypacekThesis}
Ond\v{r}ej Ryp\'{a}\v{c}ek.
\newblock {\em Distributive laws in programming structures}.
\newblock PhD thesis, University of Nottingham, 2010.


\bibitem[LR08]{LammelRypacek08}
Ralf L{\"a}mmel and Ond\v{r}ej Ryp\'a\v{c}ek.
\newblock The expression lemma.
\newblock In Philippe Audebaud and Christine Paulin-Mohring, editors, {\em
  MPC}, volume 5133 of {\em Lecture Notes in Computer Science}, pages 193--219.
  Springer, 2008.

\bibitem[RBN06]{RypacekBackhouseNilsson06}
Ond\v{r}ej Ryp\'a\v{c}ek, Roland~Carl Backhouse, and Henrik Nilsson.
\newblock Type-theoretic design patterns.
\newblock In Ralf Hinze, editor, {\em ICFP-WGP}, pages 13--22. ACM, 2006.



\bibitem[Ryp03]{RypacekMsc}
Ond\v{r}ej Ryp\'a\v{c}ek.
\newblock Polytypic programming in the context of logic programming.
\newblock Master's thesis, Charles University, Prague, 2003.


\end{thebibliography}
%%

\bibliographystyleoth{plain}
\bibliographyoth{or-cv}




\end{document}

\documentclass[a4paper]{article}

\usepackage{enumerate}
\usepackage{multibib}
\usepackage[margin=2.5cm]{geometry}

% \oddsidemargin 0in
% \evensidemargin 0in
% \textwidth=6.0in
\itemsep=0in
\parsep=0in


\newcites{pap,oth}{List of All Papers,Other References}

\newcommand{\project}{Homotopy Type Theory : Programming and Verification}
\newcommand{\mltt}{MLTT}

\begin{document}
\begin{center}\begin{Large}\textsc{CV : Ond\v{r}ej
      Ryp\'a\v{c}ek}\end{Large}\\
  \begin{large}
    Research Statement
  \end{large}

\end{center}

\subsection*{Personal data}
\begin{tabbing}
  \hspace{4cm}\= \\
  Address:\>Flat 12 Building 46, Marlborough Rd, London SE18 6TA\\
  Nationality:\>Czech\\
  Degrees:\>Phd, 2009, Computer Science, University of Nottingham\\
  \>MSc, 2005, Theoretical Computer Science, Charles University, Prague
\end{tabbing}
\subsection*{Academic Employment}
\begin{tabbing}
  \hspace{4cm}\=\\
  Oct 2012-present\>Research Assistant\\
  University of Oxford,\>Unified Model of Compositional and
  Distributional Semantics (EP/I03808X/1)\\
  Computer Science\>
\end{tabbing}
% \begin{tabbing}
%   \hspace{4cm}\=\\
%   May 2012 - Sept 2012\>Research Associate\\
%   University of Sheffield,\>Higher order Refinement Techniques for Model
%   Driven Architecture\\
%   Computer Science\>(EP/G031711/1)
% \end{tabbing}
\begin{tabbing}
  \hspace{4cm}\=\\
  Jan 2010 - Sept 2012\>Research Associate\\
  King's College London,\>Higher order Refinement Techniques for Model
  Driven Architecture\\
  Computer Science\>(EP/G03012X/1, EP/G031711/1)
\end{tabbing}



\subsection*{Research Statement}
I believe that I am not only the ideal candidate for the theory RA
(based in Nottingham) on the {\project} project, but I also believe
that it is not possible to find a better candidate by open
advertising. The depth and breadth of knowledge and experience
required from a potential candidate ranges from pure higher category
theory, to Martin-L\"of Type Theory ({\mltt}) and Homotopy Type Theory
(HoTT) (both at a theoretical and practical level) to functional
programming and containers. I have experience supported by a track
record of publications in all of these areas. Further, I have
experience in the more applied areas of the project which will allow me
to interface and collaborate well with the more applied RA in
Strathclyde - a vital capability which many theoreticians working in
HoTT do not have. This experience takes the form of significant use of
formal verification systems such as Agda and Coq. Even more, my
industrial experience means that I also am excited by - and have
something to contribute to - the impact generation parts of the
proposal. \\[1ex]

\noindent In more detail:


%evidence
\paragraph{I have a publication track record in all the relevant research areas}
\begin{itemize}
\item {\bf WP1:} I have a track record in higher dimensional category
  theory as demonstrated by my thesis~\cite{RypacekThesis} and in
  higher-categorical models of HoTT, as demonstrated by
  \citepap{AltenkirchRypacek12} and \citepap{AltenkirchLiRypacek14}
  which formalises our understanding of weak $\omega$-groupoids within
  \mltt. Weak $\omega$-groupoids are a higher dimensional analogs of
  groupoids, i.e. categories where all morphisms are isomorphisms,
  where all coherence laws are weakened from equational axioms to
  higher cells. They are fundamental in categorical semantics of HoTT.

\item {\bf WP2:} Both formulations of weak $\omega$-groupoids in
  \citepap{AltenkirchRypacek12,AltenkirchLiRypacek14} amount to a
  definition of a type theory with identity types and introduction and
  elimination rules for coherence. A syntax for the type theory is
  given in terms of contexts, types in context formed from a basic
  type by formation of identity types, and terms of types ---
  variables, \emph{refl} terms, and coherence terms. In
  \citepap{AltenkirchLiRypacek14}, a major simplification of the
  formulation is achieved by following Brunerie's reformulation of
  Grothendieck's $\omega$-groupoids, which adheres to the same
  syntactical approach but simplifies the formulation of coherence. In
  summary, I have already taken first steps in trying to present HoTT
  as a type theory.

\item {\bf WP3} intends to develop a foundational theory of HITs based
  on the theory of indexed containers in the setting of HoTT. I have
  experience and track record also in this area --- during my postdoc
  at King's College I have applied indexed containers to formalise
  data structures stored on the heap as their coalgebras {\bf citation required}. In
  \citepap{JaskelioffRypacek12} we investigate traversability of
  datatypes as distributivity over a finitary container. As containers
  lie at the heart of the methodology of WP3, this puts me in an ideal
  position to carry out the work described there. 

\item {\bf WP4} intends to study monads and Lawvere theories in HoTT
  in order to study effects in the setting of HoTT.  My thesis
  \citepap{RypacekThesis} is on 2-categorical monads and Lawvere
  theories in the context of mathematics of program construction so I
  have knowledge and experience in the motivations, techniques and
  applications of the research to be conducted in WP4.
\end{itemize}

\paragraph{I have experience with formalisation verification in Agda and
  CoQ.} This is crucial in enabling the Nottingham RA to collaborate
successfully with the Strathclyde team, especially the Strathclyde
RA. In particular, te work in
\citepap{AltenkirchRypacek12,AltenkirchLiRypacek14} has been almost
entirely formalised in Agda. I have played the major role in this
formalisation. This was a nontrivial task as the definition of the syntax of
the type theory defined therein involves a complicated mutually
recursive, inductive-recursive and inductive-inductive definition. The
complexity of the formalisation brings the Agda compiler to its limits
and motivated extension of Agda with more liberal form of mutual
definition. In addition, the work in \citepap{FosterRypacekStruth12} has been completely
formalised in both Coq and Agda showing my knowledge of formal
verification spreads across several platforms which, in turn, allows
me to understand the strengths and weaknesses of each system.
  

\paragraph{I have already established a working relationship with the
  members of the research team.} Beyond my technical expertise, there
are other reasons which make me an ideal choice to the position of the
Nottingham RA. For example, I have coauthored two papers with Dr
Thorsten Altenkirch, the PI on the Nottingham grant. This shows we have
already established a good working relationship, that we can work and
publish together. I have known and informally worked with Prof. Neil
Ghani and Dr Conor McBride since meeting them while they both worked
at Nottingham during my PhD. Finally, I've met Dr Nicola Gambino at the
Thematic Trimester of Formalised Proofs at IHP in Paris, 2014 where we
had several fruitful discussions. I have also worked with the whole team during
the preparation of the grant proposal. Moreover, I have  been involved
in developing all areas of the proposal which means I need no time to
settle into the project but can rather start producing results on day one.
Finally, I am well connected with the HoTT community, e.g.
I have spent June 2014 at the thematic trimester of Semantics of
Proofs and Certified Mathematics at IHP, Paris, attending workshops,
talks and discussing HoTT and I attended the HDACT workshop in Ljubljana, June 2012. 
I also regularly give presentations in seminars at my host universities and
meetings related to HoTT to raise awareness of the topic in general
and disseminate my work, e.g. ScotCats \citeoth{scotcats}, The
Wessex Theory Seminar \citeoth{wessex}, or Domains \citeoth{domains}.
This demonstrates that I'm an active and well connected member of
the community which will serve the project goals on making HoTT and in
particular the programming language developed within widely used.



\phantom{
\citepap{RypacekMsc}
\citepap{RypacekThesis}
\citepap{RypacekBackhouseNilsson06}
\citepap{LammelRypacek08}
\citepap{JaskelioffRypacek12}
\citepap{FosterRypacekStruth12}
\citepap{AltenkirchRypacek12}
\citepap{AltenkirchLiRypacek14}
\citepap{RypacekSadrzadeh14}}

%\bibliographystylepap{alpha}
%\bibliographypap{cv-hdtt2}
%%%
\renewcommand{\refname}{{\large List of All Papers}}
\begin{thebibliography}{RBN06}
\bibitem[ALR14]{AltenkirchLiRypacek14}
Thorsten Altenkirch, Nuo Li, and Ond\v{r}ej Ryp\'{a}\v{c}ek.
\newblock Some constructions on omega-groupoids.
\newblock In {\em Proceedings of the 2014 International Workshop on Logical
  Frameworks and Meta-languages: Theory and Practice}, LFMTP '14, pages
  4:1--4:8, New York, NY, USA, 2014. ACM.

\bibitem[AR12]{AltenkirchRypacek12}
Thorsten Altenkirch and Ond\v{r}ej Ryp\'a\v{c}ek.
\newblock A syntactical approach to weak omega-groupoids.
\newblock In Patrick C{\'e}gielski and Arnaud Durand, editors, {\em CSL},
  volume~16 of {\em LIPIcs}, pages 16--30. Schloss Dagstuhl - Leibniz-Zentrum
  fuer Informatik, 2012.

\bibitem[RS14]{RypacekSadrzadeh14}
Ond\v{r}ej Ryp\'a\v{c}ek and Mehrnoosh Sadrzadeh.
\newblock A low-level treatment of generalized quantifiers in categorical
  compositional distributional semantics.
\newblock In {\em Second Workshop on Natural Language and Computer Science},
  Vienna, July 2014.

\bibitem[FRS12]{FosterRypacekStruth12}
Simon Foster, Ond\v{r}ej Ryp\'a\v{c}ek, and Georg Struth.
\newblock Correctness of object oriented models by extended type inference.
\newblock In Abhik Roychoudhury and Meenakshi D'Souza, editors, {\em ICTAC},
  volume 7521 of {\em Lecture Notes in Computer Science}, pages 46--60.
  Springer, 2012.

\bibitem[JR12]{JaskelioffRypacek12}
Mauro Jaskelioff and Ond\v{r}ej Ryp\'a\v{c}ek.
\newblock An investigation of the laws of traversals.
\newblock In James Chapman and Paul~Blain Levy, editors, {\em MSFP}, volume~76
  of {\em EPTCS}, pages 40--49, 2012.

\bibitem[Ryp10]{RypacekThesis}
Ond\v{r}ej Ryp\'{a}\v{c}ek.
\newblock {\em Distributive laws in programming structures}.
\newblock PhD thesis, University of Nottingham, 2010.


\bibitem[LR08]{LammelRypacek08}
Ralf L{\"a}mmel and Ond\v{r}ej Ryp\'a\v{c}ek.
\newblock The expression lemma.
\newblock In Philippe Audebaud and Christine Paulin-Mohring, editors, {\em
  MPC}, volume 5133 of {\em Lecture Notes in Computer Science}, pages 193--219.
  Springer, 2008.

\bibitem[RBN06]{RypacekBackhouseNilsson06}
Ond\v{r}ej Ryp\'a\v{c}ek, Roland~Carl Backhouse, and Henrik Nilsson.
\newblock Type-theoretic design patterns.
\newblock In Ralf Hinze, editor, {\em ICFP-WGP}, pages 13--22. ACM, 2006.



\bibitem[Ryp03]{RypacekMsc}
Ond\v{r}ej Ryp\'a\v{c}ek.
\newblock Polytypic programming in the context of logic programming.
\newblock Master's thesis, Charles University, Prague, 2003.


\end{thebibliography}
%%

\bibliographystyleoth{plain}
\bibliographyoth{or-cv}




\end{document}

Training